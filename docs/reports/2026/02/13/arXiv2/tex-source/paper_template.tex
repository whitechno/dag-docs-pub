%%%%%%%%%%%%% ML-SYS
%%%%%%%% mlsys 2024 EXAMPLE LATEX SUBMISSION FILE %%%%%%%%%%%%%%%%%

\documentclass{article}

% Recommended, but optional, packages for figures and better typesetting:
\usepackage{microtype}
\usepackage{graphicx}
% \usepackage{subfigure}
\usepackage{booktabs} % for professional tables

%%%%%%%%%%%%% PACKAGE
\usepackage{tabularx}
\usepackage{array}
\setlength{\tabcolsep}{3pt} % Adjust horizontal padding if needed
\usepackage{makecell}

\usepackage{tcolorbox}
\tcbuselibrary{listingsutf8}

\usepackage{url}
\def\UrlBreaks{\do\/\do-}
\usepackage{color}
\usepackage{listings} % Make sure this is included
% \usepackage{minted}
% \usepackage[finalizecache,cachedir=.]{minted}
% \usepackage[frozencache,cachedir=.]{minted}

% \usepackage[finalizecache,cachedir=minted-cache]{minted}
\usepackage[frozencache,cachedir=minted-cache]{minted}
% \usepackage[noend]{algorithmic}

\usepackage{mdframed}
%\usepackage{array}
% \usepackage{sectsty}
% \usepackage{algorithm}
% \usepackage{algpseudocode}

\usepackage{eqparbox}
% subfigures
\usepackage{caption}
\usepackage{subcaption}
\usepackage{tikz}
\usepackage{amsmath}
\usepackage{amsthm,thmtools}
\usepackage{multirow}
%\usepackage{tabularx}
\newcommand*\circled[1]{\tikz[baseline=(char.base)]{
            \node[shape=circle,draw,inner sep=1pt] (char) {#1};}}

\declaretheoremstyle[
  headfont=\bfseries,
  bodyfont=\itshape,
]{myplain}
\declaretheoremstyle[
  headfont=\bfseries,
  bodyfont=\normalfont,
]{mydefinition}
\declaretheoremstyle[
  headfont=\itshape,
  bodyfont=\normalfont,
]{myremark}

\declaretheorem[
  style=myplain,
  name=Theorem,
]{theorem}
\declaretheorem[
  style=mydefinition,
  name=Problem,
]{problem}
\declaretheorem[
  style=myremark,
  unnumbered,
  name=Solution,
]{solution}

\setminted[sql]{breaklines, framesep=3mm, fontsize=\footnotesize, numbersep=5pt}

% Define your colors
\definecolor{codegreen}{rgb}{0,0.6,0}
\definecolor{codegray}{rgb}{0.5,0.5,0.5}
\definecolor{codepurple}{rgb}{0.58,0,0.82}
\definecolor{backcolour}{rgb}{0.95,0.95,0.92}

% Define the listing style
\lstdefinestyle{mystyle}{
    backgroundcolor=\color{backcolour},
    commentstyle=\color{codegreen},
    keywordstyle=\color{magenta},
    numberstyle=\tiny\color{codegray},
    stringstyle=\color{codepurple},
    basicstyle=\ttfamily\footnotesize,
    breakatwhitespace=false,
    breaklines=true,
    captionpos=b,
    keepspaces=true,
    numbers=left,
    numbersep=5pt,
    showspaces=false,
    showstringspaces=false,
    showtabs=false,
    tabsize=2,
    lineskip=-1pt
}



%% The following content must be adapted for the final version
% paper-specific

\newif\ifcomments
% comment below to remove comments, uncomment to add comments
% \commentstrue
\ifcomments
    % add name and specify color to have individual comment command
    \providecommand{\ion}[1]{{\color{teal}{[ion: #1]}}}
    \providecommand{\matei}[1]{{\color{green}{[matei: #1]}}}
    \providecommand{\joey}[1]{{\color{magenta}{[joey: #1]}}}
    \providecommand{\shu}[1]{{\color{cyan}{[shu: #1]}}}
    \providecommand{\souj}[1]{{\color{red}{[souj: #1]}}}
    \providecommand{\asim}[1]{{\color{blue}{[asim: #1]}}}
    \providecommand{\simon}[1]{{\color{orange}{[simon: #1]}}}
    \providecommand{\accheng}[1]{{\color{purple}{[accheng: #1]}}}
    \providecommand{\amog}[1]{{\color{orange}{[amog: #1]}}}
    \newcommand{\todo}[1]{\textcolor{red}{TODO: #1}}
\else
    \providecommand{\ion}[1]{}
    \providecommand{\joey}[1]{}
    \providecommand{\shu}[1]{}
    \providecommand{\asim}[1]{}
    \providecommand{\simon}[1]{}
    \providecommand{\accheng}[1]{}
\fi
\newcommand{\SYS}{}
\def\SYS/{SYS}
\newcommand{\sys}{SYS }

\usepackage{xspace}
\newcommand{\optimal}{OPHR\xspace}
\newcommand{\greedy}{GGR\xspace}
\newcommand{\pluseq}{\mathrel{+}=}

% hyperref makes hyperlinks in the resulting PDF.
% If your build breaks (sometimes temporarily if a hyperlink spans a page)
% please comment out the following usepackage line and replace
% \usepackage{mlsys2024} with \usepackage[nohyperref]{mlsys2024} above
\usepackage{hyperref}

% Attempt to make hyperref and algorithmic work together better:
\newcommand{\theHalgorithm}{\arabic{algorithm}}

% Use the following line for the initial blind version submitted for review:
% \usepackage{mlsys2025}
% \pagestyle{empty}


% If accepted, instead use the following line for the camera-ready submission:
\usepackage[accepted]{mlsys2025}
\fancyfoot[C]{\raisebox{-20pt}{\thepage}} % to show page number at bottom center without affecting the text layout

% The \mlsystitle you define below is probably too long as a header.
% Therefore, a short form for the running title is supplied here:
% \mlsystitlerunning{Submission and Formatting Instructions for MLSys 2025}

\begin{document}
\lstset{style=mystyle}

% \makeatletter
% \algnewcommand{\LineComment}[1]{\Statex \hskip\ALG@thistlm \(\triangleright\) #1}
% \makeatother

\twocolumn[
\mlsystitle{Optimizing LLM Queries in Relational Data Analytics Workloads}

% It is OKAY to include author information, even for blind
% submissions: the style file will automatically remove it for you
% unless you've provided the [accepted] option to the mlsys2024
% package.

% List of affiliations: The first argument should be a (short)
% identifier you will use later to specify author affiliations
% Academic affiliations should list Department, University, City, Region, Country
% Industry affiliations should list Company, City, Region, Country

% You can specify symbols, otherwise they are numbered in order.
% Ideally, you should not use this facility. Affiliations will be numbered
% in order of appearance and this is the preferred way.
\mlsyssetsymbol{equal}{*}

\begin{mlsysauthorlist}
\mlsysauthor{Shu Liu}{equal,to}
\mlsysauthor{Asim Biswal}{equal,to}
\mlsysauthor{Amog Kamsetty}{to}
\mlsysauthor{Audrey Cheng}{to}
\mlsysauthor{Luis Gaspar Schroeder}{to,goo}
\end{mlsysauthorlist}
\begin{mlsysauthorlist}
\mlsysauthor{Liana Patel}{ed}
\mlsysauthor{Shiyi Cao}{to}
\mlsysauthor{Xiangxi Mo}{to}
\mlsysauthor{Ion Stoica}{to}
\mlsysauthor{Joseph E. Gonzalez}{to}
\mlsysauthor{Matei Zaharia}{to}
\end{mlsysauthorlist}

\mlsysaffiliation{to}{UC Berkeley}
\mlsysaffiliation{goo}{Technical University of Munich}
\mlsysaffiliation{ed}{Stanford University}

\mlsyscorrespondingauthor{Shu Liu}{lshu@berkeley.edu}
% \mlsyscorrespondingauthor{Eee Pppp}{ep@eden.co.uk}

% You may provide any keywords that you
% find helpful for describing your paper; these are used to populate
% the "keywords" metadata in the PDF but will not be shown in the document
\mlsyskeywords{Machine Learning, MLSys}

\vskip 0.3in

% V2
\begin{abstract}

Batch data analytics is a growing application for Large Language Models (LLMs).
LLMs enable users to perform a wide range of natural language tasks, such as classification, entity extraction, and translation, over large datasets.
However, LLM inference is highly costly and slow:
for example, an NVIDIA L4 GPU running Llama3-8B can only process 6 KB of text per second, taking about a day to handle 15 GB of data; processing a similar amount of data costs around \$10K on OpenAI's GPT-4o.
In this paper, we propose novel techniques that can significantly reduce the cost of LLM calls for relational data analytics workloads.
Our key contribution is developing efficient algorithms for reordering the rows and the fields within each row of an input table to maximize key-value (KV) cache reuse when performing LLM serving.
As such, our approach can be easily applied to existing analytics systems and serving platforms. Our evaluation shows that our solution can yield up to 3.4$\times$ improvement in job completion time on a benchmark of diverse LLM-based queries using Llama 3 models. Our solution also achieves a 32\% cost savings under OpenAI and Anthropic pricing models.

\end{abstract}

]



% this must go after the closing bracket ] following \twocolumn[ ...

% This command actually creates the footnote in the first column
% listing the affiliations and the copyright notice.
% The command takes one argument, which is text to display at the start of the footnote.
% The \mlsysEqualContribution command is standard text for equal contribution.
% Remove it (just {}) if you do not need this facility.

% \printAffiliationsAndNotice{}  % leave blank if no need to mention equal contribution
\printAffiliationsAndNotice{\mlsysEqualContribution} % otherwise use the standard text.


%\section{Introduction}
% Shu 
% Start with MLSys example, one of the most impactful applications of batch inferece is data analysis with LLMs, All the major analytical databases; make it easier to . batch analytics --> large amount of data, for example, database vendors; dataframe. Using this with the new API , users can write a SQL query of the following 

% One of the main use cases (data analysis apps), and say this pattern is not too optimized, even tho we build great servers take the batch and do batching ,aslo need analytical system to do something to optimize the hit rate 

% SQL thing should be shown as an example, with this API, write the query as the following 

% Batch data analytics worklaods
% Widely used applications for LLMs (to LLM, MLSYs people, one of your applications) 

% A lot of parts still talk about SQL, pick a consistent naming and use it throughout 

% Present it as an insight (what level should we reorder, discover that it is important to do it per row) 
%%%%%%%%%%%%%%%%%%%%%%%%%
% \asim{i feel like the first sentence could use some work}
% Large Language Models (LLMs) are changing the landscape of textual data analysis.

One of the most popular applications of large language model (LLM) batch inference is data analytics. 
A growing number of analytics platforms now support LLM invocations for complex analytical tasks. 
For instance, leading database vendors, such as 
    AWS Redshift~\cite{aws-redshift-llm}, Databricks~\cite{databricks-ai-functions}, and Google BigQuery~\cite{google-bigquery-llm}, have integrated LLM functionality into their SQL APIs. 
    Similarly, DataFrame libraries and programming frameworks~\cite{langchain, lotus} offer LLM support for querying relational (table-based) data. 
With these new APIs, users can write queries like the following: 
% \souj{You could consider starting with - Many leading database vendors...}

\vspace{2pt}
\begin{mdframed}[linecolor=black, linewidth=.5pt]
\begin{minted}[fontsize=\small]{sql}
SELECT user_id, request, support_response, 
  LLM('Did {support_response} address {request}?', support_response, request) AS success
FROM customer_tickets 
WHERE support_response <> NULL
\end{minted}
\end{mdframed}
\vspace{-1pt}
where the LLM is invoked for each row in the customer ticket table to analyze whether the customer service requests are effectively addressed. 
Increasingly, analysts wish to leverage LLMs in such queries for tasks including classification, entity extraction, summarization, and translation~\cite{databricks-ai-functions}. Going forward, we will refer to queries that invoke LLMs over relational data as \textit{LLM queries}.

Unfortunately, applying LLMs to real-world datasets (which can contain millions of rows) incurs significant computational and monetary costs.       
% \souj{Is this dollar cost for 1M or 15K rows? High latency low cost vs low latency but high cost tradeoff isn't coming through} NOT SURE how to deal with this yet
Accordingly, there has been growing research on LLM inference optimization.
In particular, recent work~\cite{vllm, sglang, cascade-inference, hydragen, promptcache} leverages prompt caching, a technique that stores the attention states of frequently reused prompt segments in GPU memory, known as key-value (KV) cache ~\cite{attention-is-all-you-need}. Reusing cached state whenever a similar \textit{prefix} of prompts appears again can significantly reduce inference latency~\cite{sglang}. 
In addition, prompt reuse also brings economic benefits.
Recently, providers like OpenAI, Anthropic, and Google Gemini~\cite{openai-pricing, anthropicpromptcaching, gemini} have introduced prompt caching as a service, charging 2--10$\times$ less for cached prompts.  
Therefore, maximizing \textit{prefix hits} in the prompt KV cache is crucial for reducing both LLM request time and monetary costs.
% Reusing prefixes in the cache has also been shown to have an outsized impact on performance.

However, simply invoking LLMs over relational data within analytical engines and connecting to a backend inference server with a prompt cache often results in low cache hit rates. This approach fails to exploit relational workloads to fully maximize cache reuse. 
% \shu{@joey: it's not about memory resource only; even with infinite memory, changing field-order improve hit rates; need to address both (columns) to get }

% However, existing LLM inference systems are mostly optimized for online serving workloads---they process requests as soon as they arrive, typically in a first-in, first-out (FIFO) order. Thus, they miss out on opportunities to improve performance by potentially re-ordering requests and taking advantage of the offline relational workloads information. \shu{have batch APIs, vague, change something in the analytical system to maximize cache reuse; take DB that makes LLM requests and connect it with a server with prompt cache, might still get a very low hit rate, in this paper, we show how modifying the analytical engine can greatly improve the hitrate}

% We introduce various techniques that reduce end-to-end query latency, dominated by LLM request latency. 
In this work, we identify and present solutions to optimize relational data analytics workloads for offline LLM inference.
In particular, given an LLM query, we propose \textbf{request reordering} at the row and field granularity of the relational data. 
Our key insight is that, with oracular knowledge of all requests to be sent to the LLM, we can reorder both the requests and the fields inside each request to increase the number of cache hits. 
In real datasets, there can be many sharing opportunities across rows and fields. For example, joining feature tables, referencing popular items, or repeating similar context in RAG queries~\cite{retrieval-augmented-generation}.
These common patterns lead to repeating values in different fields, leaving rooms for significantly improving cache hit rates by optimizing request order and format. 

% creating more chances for prompt prefix cache hits when requests are reordered. Optimizing request order and format in these cases can significantly boost cache hit rates.

% Tables with joined user reviews and item metadata are common and include many shared values across metadata fields for each unique review. 
% In typical datasets, numerous opportunities for sharing across rows and fields naturally arise due to common use cases, such as joining with feature tables, referencing popular items, or retrieving similar contexts in retrieval-augmented generation (RAG) queries. 
% These settings often include repeating values across fields or rows, such as shared metadata for frequently queried items, which increases the potential for prompt prefix KV cache hits when request order and format are optimized.
% Changing the order and format of requests will markedly increase the prompt prefix KV cache hit rate in these cases. 



% For instance, two requests that share the same prefix (which may be non-consecutive under FIFO ordering) should be passed to the LLM together so that the latter can experience a cache hit. 
% Likewise, in requests that input multiple fields of data (e.g., product name and review) into the LLM, the fields should be ordered to maximize the number of shared prefixes.
% In real datasets, there can be many sharing opportunities across columns and rows, so changing the order and format of requests will markedly increase the prompt prefix KV cache hit rate.

% 
% Finding the optimal ordering of requests is challenging due to the exponential number of choices to order the fields and rows of data in a query. For a table with $n$ rows and $m$ fields, there are $n \times m!$ potential orderings. 
Finding the optimal ordering of requests is challenging due to the exponential number of choices to order the fields and rows of data in a query. For a table with $n$ rows and $m$ fields, there are $n! \times (m!)^n$ potential orderings. 
One way to reduce this search space is to apply the same field ordering across all rows. However, as we show in Sec~\ref{subsec:casestudy}, this can reduce the prefix hit count by up to a factor of $m$ compared to reordering fields on a more fine-grained, per-row basis.
% While one could dramatically reduce this number by using the same field reordering for all rows, this is a poor option, as it can reduce the hit rate by as much as $m\times$ compared to reordering the fields per-row basis. 
% \ion{This is hard to understand. An example would be nice.} \shu{we are going to show it can be as worst as} 
%While a fixed field ordering for all rows is straightforward, we show that it can yield as much as $m$ times fewer cache hits than tailored field orderings for different rows.
To support per-row field reordering, we introduce \textbf{Optimal Prefix Hit Recursion (\optimal)}, an algorithm that divides the table into smaller subtables and reorders each subtable to maximize the prefix hits. While \optimal achieves high hit rates, its complexity is exponential, which makes it impractical for large datasets. To address this challenge, we propose \textbf{Greedy Group Recursion (\greedy)}, an approximate algorithm that leverages functional dependencies (such as primary and foreign key relationships from the data schema) and table statistics, which are readily available in many databases and analytics systems, to reduce the search space.
In particular, functional dependencies help identify correlated fields, reducing the number of fields that need to be reordered at each step, thus decreasing the solver runtime.
% In particular, we use functional dependencies to identify which fields will be correlated early in the recursive algorithm, thereby reducing the solver runtime. 
%if we decide on the ordering early 
% This way, \greedy narrows down the columns it needs to process at each step, reducing runtime while providing close-to-optimal performance. 
In addition, \greedy leverages the cardinality and length statistics to efficiently approximate the greedy objective. 
% \souj{Priorities are not introduced and the last line seems too abrupt or too low-level for an intro?} it is talked about in the previous two sentences, table statistics + func dependency 

% , including Amazon Product Review~\cite{amazon-product-review-dataset}, Rotten Tomatoes Movie Dataset~\cite{rotten-tomatoes-movies-dataset}, Stanford Question Answering Dataset (SQuAD)\cite{squad-dataset}, and FEVER Dataset~\cite{}
We implement our techniques in Apache Spark~\cite{zaharia2012resilient} and use vLLM~\cite{vllm} as the model serving backend. 
Due to the lack of standard workloads in this area, we build a benchmark suite of 16 LLM queries of different types, spanning selection, projection, multi-LLM invocations, and retrieval-augmented generation (RAG) queries~\cite{retrieval-augmented-generation}. We evaluate these queries on recommendation and question-answering datasets such as Amazon Product Reviews, Rotten Tomatoes Movies, BIRD, Stanford Question Answering Dataset, Public Domain MusicXML, RateBeer Reviews, and Fact Extraction and VERification datasets~\cite{amazon-product-review-dataset, rotten-tomatoes-movies-dataset, li2024can, squad-dataset, pdmx, fever}. Our techniques show 1.5 -- 3.4$\times$ speed-up in end-to-end query latency and reduce costs by up to 32\% on proprietary model APIs, while preserving query semantics. In summary, our contributions are as follows: 
\vspace{-0.5em}
\begin{itemize}
    \item We identify significant opportunities to speed up LLM-based batch data analytics through reordering rows and fields of input tables.  
    \item We introduce an optimal reordering algorithm (\optimal) that maximizes prefix sharing but with exponential complexity. 
    We propose an efficient greedy algorithm (\greedy) that approximates \optimal by leveraging functional dependencies and table statistics. We show that a fixed field ordering can yield as much as $m$ (number of fields) times worse cache hits than our solution.
    \item We present an LLM query benchmark consisting of 16 queries and 7 real-world datasets to represent a range of retrieval and processing tasks. Our evaluation with Llama3-8B and 70B shows up to a 3.4$\times$ speedup in end-to-end query latency compared to naive orderings. With OpenAI and Anthropic prefix cache pricing models, our techniques reduce costs by up to 32\%.
\end{itemize}

% \section{Introduction}
Large Language Models (LLMs) are changing the landscape of textual data analysis.
% making it dramatically easier to analyze textual data. 
In fact, a number of analytical database vendors, including AWS Redshift~\cite{aws-redshift-llm}, Databricks~\cite{databricks-ai-functions}, and Google BigQuery~\cite{google-bigquery-llm}, have already added LLM invocation functions to their SQL APIs. \shu{Matei: any user data to add here? Shall we remove this example}
As an example, consider the following SQL query: 

\vspace{4pt}
\begin{mdframed}[linecolor=black, linewidth=.5pt]
\begin{minted}[fontsize=\small]{sql}
SELECT user_id, request, support_response, 
  LLM('Did {support_response} address {request}?', support_response, request) AS success
FROM customer_tickets 
WHERE support_response <> NULL
\end{minted}
\end{mdframed}
\vspace{4pt}
where the LLM is invoked to analyze whether the customer service requests are effectively addressed.
Increasingly, analysts wish to leverage LLMs in such queries for tasks, including classification, entity extraction, summarization, and translation~\cite{databricks-ai-functions}. Going forward, we will refer to SQL queries that invoke LLMs as \textit{LLM queries}.

Unfortunately, applying LLMs this way to real-world datasets (which can contain millions of rows) has significant computational and economic costs. 
For example, question answering with 22K facts from Fact Extraction and VERification (FEVER) dataset~\cite{fever} takes 30 minutes on an NVIDIA L4 GPU instance with a Llama3-8B model~\cite{llama2}. Using OpenAI GPT-4o model, running a query over this single table along takes \$70. 
% For example, classifying the sentiment of 15K user reviews from Amazon Product Recommendation dataset~\cite{amazon-product-review-dataset} takes 30 minutes on an NVIDIA L4 GPU instance with a Llama3-8B model~\cite{llama2}. 

% Unfortunately, applying LLMs this way to real-world datasets (which can contain millions of rows) has significant computational and economic costs. 
% For example, classifying 15K rows of user reviews from Amazon Product Recommendation dataset~\cite{amazon-product-review-dataset} takes 30 minutes on an NVIDIA L4 GPU instance with a Llama3-8B model~\cite{llama2}. 
% On a similar sized instance, an analytical database, such as DuckDB~\cite{duckdb}, can process more than 100GB of data per second in the TPC-DS benchmark~\cite{duckdb-tpcds-benchmark}. 
% Processing the equivalent amount of data via the same LLM would take 96 days, more than 8 million times longer! Thus, minimizing the cost of LLM invocations is the critical objective for LLM queries. 
% Later in this paper, we demonstrate novel optimizations that can reduce LLM runtime by 5$\times$. 


There has been growing research on how to optimize this process. 
Notably, OpenAI, Anthropic, and the other model serving platform has deployed Prompt Caching~\cite{openai-pricing,vllm,cascade-inference,hydragen} \shu{cite anthropic and prompt cache paper}. This technique involves caching the attention states of commonly revisited prompt segments in key-value (KV) cache in memory~\cite{attention-is-all-you-need}, allowing for efficient reuse whenever similar segments of prompts reappear, which helps minimize latency. Reusing prefixes in the cache has also been shown to have an outsized impact on performance~\cite{sglang}. Accordingly, existing inference systems aim to maximize prefix hits in the KV cache to reduce LLM request time and monetary costs.

% At a high level, the output of each query is generated sequentially, as each output token depends on all previous tokens. To improve throughput, modern LLM serving engines batch multiple requests to process them in parallel.
% However, this requires storing all requests in a batch in memory while they are being processed. 
% As a result, efficient memory management is critical for LLM inference performance. LLM inference engines store intermediate states for past prompts and generations, or \textit{prefixes} of these requests, in a key-value (KV) cache~\cite{attention-is-all-you-need,vllm}. Reusing prefixes (e.g., between requests that share the same prompt) in the cache has been shown to have an outsized impact on performance~\cite{sglang,vllm,cascade-inference,hydragen}. Accordingly, existing inference systems aim to maximize prefix hits in the KV cache.

Existing LLM inference systems~\cite{vllm,sglang} are mostly optimized for online serving workloads---they process requests as soon as they arrive, typically in a first-in, first-out (FIFO) order, to minimize the latency. Thus, they miss out on opportunities to improve performance by potentially re-ordering requests to take advantage of the offline analytical workload information. 

In this work, we address the problem of optimizing inference for LLM queries. We introduce various techniques that reduce end-to-end query latency, dominated LLM request latency. To optimize for relational queries using LLMs, we propose dynamic \textbf{request reordering} at the row and column granularity to improve the KV cache hit rate. Our key insight is that, with oracular knowledge of all requests to be sent to the LLM, we can reorder both the requests and the fields inside each request to increase the number of cache hits. For instance, two requests that share the same prefix (which may be non-consecutive under FIFO ordering) should be passed to the LLM together so that the latter can experience a cache hit. 
Likewise, in requests that input multiple fields of data (e.g., a product name and review) into the LLM, the fields should be ordered to maximize the number of shared prefixes.
In real datasets, there can be many shared prefixes across both columns and rows of data, so changing the order and format of requests will markedly increase the prefix KV cache hit rate.


However, finding the optimal order and format of requests is challenging because there are an exponential number of ways to order the columns and rows of data in a query. For example, for a table with $n$ rows and $m$ columns, there are $n \times m!$ possible orderings. First, we introduce an optimal algorithm, \textbf{Optimal Prefix Hit Recursion (\optimal)}, which maximizes the prefix hit rate by recursively dividing the table in optimal subtables. However, \optimal is impractical for large datasets due to its exponential complexity. To approximate the optimal solution efficiently, we present the \textbf{Greedy Group Recursion (\greedy)} algorithm. Our key insight is to leverage functional dependencies and table statistics to approximate the optimal solution efficiently. In particular, we use functional dependencies to identify which fields will be correlated early in the recursive algorithm, thereby reducing the need for backtracking. %if we decide on the ordering early 
This way, \greedy narrows down the columns it needs to process at each step, reducing runtime while providing close-to-optimal performance. In addition, \greedy also leverages table statistics, such as the frequency and size of values, to approximate column order priority. 


% In addition to our request reordering techniques, we present two optimizations to further reduce the computational costs of LLMs in relational queries. First, we observe that many real-world workloads have duplicates in textual data that lead to redundant LLM invocations. With deduplication, we can minimize the number of LLM calls without affecting the accuracy of the overall query. 
% Second, we estimate LLM operator costs within query expressions. This optimization allows for the strategic reordering of operations by considering the significant expense associated with LLM operators. 


We implement our techniques in Apache Spark~\cite{spark-sql} with vLLM~\cite{vllm} and SGLang~\cite{sglang} as the model serving backend. 
Given the lack of standard workloads in this area, we build a diverse benchmark suite of LLM queries on multiple real-world datasets. We construct a wide range of query types, such as selection, projection, multi-LLM invocations, and retrieval-augmented generation (RAG)~\cite{retrieval-augmented-generation} queries, across a variety of recommendation and question-answering datasets, including Amazon Product Review \cite{amazon-product-review-dataset}, the Rotten Tomatoes Movie Dataset \cite{rotten-tomatoes-movies-dataset}, the Stanford Question Answering Dataset (SQuAD), and the FEVER Dataset \cite{squad-dataset}. We find that our techniques provide 2.1 -- 5.7$\times$ improvements in end-to-end query latency while preserving query semantics. In summary, our contributions are as follows: 
% \vspace{-1.6em}
\begin{itemize}
    \item We identify significant opportunities to speed up LLM queries through reordering rows and columns.
    \item We present an optimal algorithm (\optimal) that maximizes prefix sharing but with exponential complexity. We propose a greedy algorithm (\greedy) that approximates \optimal by leveraging functional dependencies and table statistics for efficiency.
    \item We present a set of LLM query benchmarks using real-world data to represent a range of retrieval and processing tasks. Our evaluation using vLLM and SGLang shows up to a 5.7$\times$ speedup in end-to-end query latency compared to naive baselines.
\end{itemize}

% \section{Background and Motivation}

% \accheng{change to not say only offline, FIFO but there should be some info we use}
% \shu{to be fixed}
% LLMs are a powerful tool for programmatic analysis of text data and are being rapidly incorporated into major analytical DBMSes~\cite{google-bigquery-llm,aws-redshift-llm,databricks-ai-functions}.
% LLM inference has several unique characteristics that have significant implications for performance. 
This section provides a brief overview of the inference process and the key components of the LLM architecture.
%to provide context. 
% We then present opportunities to improve performance for relational analytics.
% While there is prior work on improving model serving with data systems~\cite{velox}, these do not focus on LLMs. In particular, this research does not address impact of data dependencies between requests and also does not optimize for the tradeoff between memory and compute present in LLMs.

\textbf{LLM inference.} 
% LLMs have seen growing usage in a range of application to comprehend and generate text. 
LLMs are made up of autoregressive Transformer models~\cite{attention-is-all-you-need}, which generate words as tokens, one at a time, until a termination token is generated or a token limit is reached. The inference process for LLMs occurs in two stages: (i) the prefill stage, in which the model processes all the input prompts at once, and (ii) the decoding stage, during which output generation occurs. 
Generation for each request proceeds sequentially since the process depends on (the product of the conditional probabilities of) all previous tokens. 
% Generation for each request proceeds sequentially (i.e., new tokens must be generated one by one) since the process depends on (the product of the conditional probabilities of) all previous tokens.
% This process continues until the model outputs a termination token. 
% LLMs are made up of autoregressive Transformer models~\cite{attention-is-all-you-need}, which generate words as tokens, one at a time, based on a given prompt and the existing sequence of tokens that have already been outputted. A \textit{token} is a concise representation, typically Byte-Paired Encoding \cite{byte-pair-encoding}, of a chunk of characters.

An LLM inference engine (e.g., vLLM \cite{vllm}, TGI \cite{tgi}, TensorRT-LLM \cite{trt-llm}) runs the transformer models and schedules the prefill and decode stages. The LLM inference engine batches multiple requests continuously \cite{orca-continous-batching} together to improve throughput. 
During the inference process, the intermediate computed state for all tokens involved is stored in memory.
This token state is cached as key and value vectors in the \textit{key-value (KV) cache}. Each token can take up to 800KB for a 13B Model \cite{vllm}, so an average request (involving 2,000 tokens) can take up to 1.6 GB of space. Furthermore, even with batching (for online workloads, batch sizes of up to 32 requests are used~\cite{orca-continous-batching}), LLM inference is computationally expensive and is currently limited to a processing speed of ~2,000 tokens/s on a single GPU. 
As such, LLM performance is currently the bottleneck in many analytical tasks.

% \joey{Don't forget to fill in [X]}

% \accheng{
% - what is a token? abstraction for string 
% - how to convert token to KV? tokens need to be computed b/c they in the attention step}


% \accheng{kv vecs are by-product

% since the new tokens are conditioned on the old tokens, you need to store the old ones in cache (an some computation is already done on them)
% }
\vspace{-0.5em}
\textbf{Prompt KV cache.}
A crucial factor for LLM serving throughput is memory management within the KV cache.
% Consequently, managing memory usage in the KV cache is crucial for LLM serving throughput. 
% This cache has several unique characteristics: it can dynamically grow and shrink over time as the model generates new tokens, and its lifetime and size are not known a priori. 
To enable maximum cache utilization (i.e., hit rate), recent work proposes sharing tokens across requests. Since many user prompts share the same \textit{prefix}, or input prompt, the KV cache can store one copy of the prefix in advance to reduce redundant computation during the prefill phase. For instance, if two requests share a prefix, the first request will have already performed some computation on the input tokens and loaded these results in the KV cache. The subsequent request can directly reuse these values for further inference without having to recompute the shared tokens. 

% An example of prefix sharing across requests to enhance KV cache efficiency is shown in Figure~\ref{fig:simple-sharing}. Reqs 2 and 3 share some prefix tokens with Req 1.
% \accheng{if not in cache, need to recompute KV for tokens}

% Existing research confirms that effective KV cache management is critical to end-to-end latency and throughput \cite{vllm, sglang}. 
% focuses only on online inference in which the LLM assumes no knowledge about future requests \cite{vllm}.
\vspace{-0.5em}
% \subsection{Optimization Opportunities in Analytics}
% \shu{this needs to be fixed - ignore now}
% In this paper, we optimize LLM inference in the context of relational analytics. We describe new opportunities to improve performance by utilizing the structure and semantics of SQL workloads. \shu{this sounds repetitive}

\textbf{Improving KV cache hit rate}.
% Existing LLM online inference engines make no assumptions about future requests and execute queries as they arrive, or first-in, first-out (FIFO) order. As a result, they miss out on opportunities to improve cache hit rate by leveraging the workload information present in relational analytics.
For data analytics, we observe particularly high opportunities for \textit{prefix KV cache} sharing. 
Given information about the structure and data of the full set of requests and, critically, the ability to rearrange the requests before execution, requests can be arranged to maximize prefix KV cache reuse during inference. 
% For instance, two requests that share a prefix may execute non-consecutively in the original data ordering, and both would lead to cache misses. Instead, if we ordered these requests together, we could ensure that the latter results in a cache hit. 
% Since all prefixes are known in batch queries, we can group shared prefixes together to increase the KV cache hit rate.
Overall, we want to maximize the \textit{prefix hit count}, or the sum of the length of prefixes that can be shared in the KV cache. 


\textit{Our Approach: Request Reordering.} We leverage workload information to enhance the KV cache hit rate. Specifically, we introduce algorithms that reorder requests and fields within each request to maximize prefix sharing across requests and enable efficient memory usage. Our algorithm leverages functional dependencies and table statistics to reduce runtime while finding near-optimal orderings that maximize prefix hit count.

%
%\vspace{-1em}

\section{Background and Motivation}
\label{sec:motivation}
This section provides a brief overview of the inference process and the key components of the LLM architecture.

\textbf{LLM inference.} 
LLMs are made up of autoregressive Transformer models~\cite{attention-is-all-you-need}, which generate text token by token until a termination token or a length limit is reached. LLM inference consists of two stages: (i) the prefill stage, where the model processes the input prompts, and (ii) the decoding stage, where it generates output sequentially, as each token depends on all previously generated tokens through a chain of conditional probabilities.
LLM inference engines (e.g., vLLM \cite{vllm}, TGI \cite{tgi}, TensorRT-LLM \cite{trt-llm}) typically batch requests continuously \cite{orca-continous-batching} to improve throughput. 
The intermediate computed state for all tokens involved is stored in memory.
This token state is cached as key and value vectors in the \textit{key-value (KV) cache}, consuming up to 800KB per token for a 13B Model \cite{vllm}. 
A typical request (involving 2,000 tokens) can require up to 1.6 GB of memory. 
Despite batching (up to 32 requests), inference remains compute-intensive, with current speeds limited to ~2,000 tokens/s per GPU, making LLM performance a bottleneck for many analytical tasks.


\textbf{Prompt KV cache.}
Efficient KV cache management is critical for high LLM serving throughput.
Recent work improves cache utilization by reusing tokens across requests with shared prefixes~\cite{sglang}.
For example, if two requests share a \textit{prefix} in prompts, the first will already have performed some computation on the input tokens and cached results in the KV cache during the prefill phase. 
The subsequent request can then reuse these cached values, avoiding redundant computation of the shared tokens.
% The subsequent request can thus directly reuse these values for further inference without having to recompute the shared tokens. \amog{should RadixAttention be cited here for prompt caching?}


\textbf{Improving KV cache hit for analytics workloads}.
Real-world relational databases often exhibit diverse repetitive data patterns. 
Columnar storage systems like C-Store and Parquet~\cite{stonebraker2018c} exploit repeated values across fields for compression, while techniques like run-length encoding (RLE), multi-relational data mining, and correlation analysis~\cite{lemire2011reordering, multirelation,correlation} leverage diverse data relationships to optimize query execution. 
Relational queries also create data groupings based on access patterns. 
Techniques such as database cracking and multi-dimensional clustering (MDC)~\cite{craking,mdc}, including Delta Lake Z-order~\cite{deltalake}, reorganize data based on query patterns to optimize performance.

These structural repetitive patterns present an opportunity for \textit{prefix KV cache} sharing in an LLM query.
In our setting, an LLM is invoked once per row in a relational table, resulting in a batch of model requests from a single LLM query. Since the full table structure and content are known in advance, we can reorder these requests to maximize shared prefixes and reduce redundant computation during inference. Our goal is to maximize the \textit{prefix hit count} -- the sum of the length of token prefixes reused from the KV cache. 

% Given an LLM query (where an LLM is applied row-wise over a relational table), and the full structure and content of the table are known, we can reorder the requests associated with this query to maximize shared prefixes and reduce redundant computation. 
% Given an LLM query where an LLM is invoked row-wise (each as one request) over a relational table, and the full table content is known in advance, we can reorder the requests to maximize prefix reuse and reduce redundant computation.
% Our goal is to maximize the \textit{prefix hit count} -- the sum of the length of token prefixes reused from the KV cache. 

% Given an LLM query where LLM can be invoked multiple times over each row of the table, and the structure and data of the table it touches, we can rearrange the requests associated with this query to maximize shared prefixes. Overall, our goal is to maximize the \textit{prefix hit count}, or the sum of the length of prefixes that can be shared in the KV cache. 

% Similarly, multi-relational data mining~\cite{} and correlation analysis~\cite{} leverage data relationships to optimize query execution. 
% Additionally, relational queries often filter or access different fields, leading to implicit data groupings based on access patterns. Database cracking~\cite{} and multi-dimensional clustering (MDC)~\cite{}, including Delta Lake Z-order~\cite{}, reorganize data dynamically to improve query performance. 

\textit{Our Approach: Request Reordering.} We leverage table information to enhance the KV cache hit rate. Specifically, we introduce algorithms that reorder requests of an LLM query and fields within each request to maximize prefix sharing. Our algorithm leverages functional dependencies and table statistics to reduce runtime while finding near-optimal orderings that maximize prefix hit count.
% Given an LLM query that invokes and that we know information about the structure and data of the table it touches, we can rearrange requests to maximize shared prefixes. Overall, our goal is to maximize the \textit{prefix hit count}, or the sum of the length of prefixes that can be shared in the KV cache. 


% Given information about the structure and data of the full set of requests and, critically, the ability to rearrange the requests before execution, requests can be arranged to maximize prefix KV cache reuse during inference. 
% Overall, we want to maximize the \textit{prefix hit count}, or the sum of the length of prefixes that can be shared in the KV cache. 

%
\vspace{-0.8em}
\section{Problem Setup}
% \textbf{LLM Request Structure.} 
This section introduces the problem setup of maximizing prefix hits in the prompt cache (Sec~\ref{subsec:setup}) and highlights cases where naive fixed field ordering can result in significantly lower hit rates (Sec~\ref{subsec:casestudy}).


\begin{figure*}[tbp]
     \centering
     \begin{subfigure}[b]{0.48\textwidth}
        \centering
        % \includegraphics[width=\textwidth]{figures/movies_runtimes_e2e.pdf}
        \includegraphics[width=\textwidth]{figures/MLSys_Figures/casestudy1.pdf}
        \caption{Distinct Values in the First Field}
        \label{fig:cases-1}
    \end{subfigure}
    \hfill
    \begin{subfigure}[b]{0.48\textwidth}
        \centering
        % \includegraphics[width=\textwidth]{figures/products_runtimes_e2e.pdf}
        \includegraphics[width=\textwidth]{figures/MLSys_Figures/casestudy.pdf}
        \caption{Group of Identical Values in each Field, $m = 3$}
        \label{fig:cases-2}
    \end{subfigure}

    \vspace{-0.5em}
    \caption{\textbf{Case Study of Fixed Field Ordering:} Comparing the PHC of a fixed field ordering to a better ordering in two scenarios. Green boxes denote cache hits; red boxes indicate cache misses. A box labeled $G_{i}$ signifies consecutive rows share the same values in Field $i$; otherwise, assume values are distinct. Fig~\ref{fig:cases-1} shows fixed field ordering can be $(n-1)(m-1)$ worse in terms of PHC compared to an optimized ordering. Fig~\ref{fig:cases-2} shows fixed field ordering can be $m$ times worse in PHC compared to an optimized ordering, where $m = 3$.}
    \label{fig:runtimes}
    \vspace{-1.5em}
\end{figure*}
\vspace{-0.5em}
\subsection{Setup and Objective}
\label{subsec:setup}
% \accheng{@shu, is json format common for existing apps/workloads?}
In this work, we consider a generic LLM operator that takes the text of the prompt as well as a \emph{set} of expressions listing one or more fields $\{T.a, T.b, T.c\}$ or $\{T.*\}$ of the table $T$. This simple design can be easily implemented in most analytics systems and enables us to dynamically reorder fields within these expressions to optimize for cache efficiency. Consider the following example query:
% Since cache hits occur only for the request prefixes, determining the right order of columns in SQL queries can significantly impact performance. 

\vspace{-0.13em}
\begin{mdframed}[linecolor=black, linewidth=.5pt]
\begin{minted}[fontsize=\small]{sql}
SELECT LLM("Summarize: ", pr.*) 
FROM (
    SELECT review, rating, description
    FROM reviews r JOIN product p ON r.asin = p.asin
) AS pr
\end{minted}
\end{mdframed} 
\vspace{-0.2em}

This query sends a list of rows, each with fields \textit{review}, \textit{rating}, and \textit{description} from table \textit{pr} to the LLM for a summarization task. 

% Starting with \textit{review} is inefficient due to its many distinct values, which reduces prefix sharing. Placing \textit{description} first increases shared prefixes, as more reviews link to the same product. Effective reordering must balance prefix frequency and length to optimize cache reuse.


% \textbf{Objective} 
\textbf{Objective} The goal of request scheduling is to \textbf{maximize} the \textit{prefix hit count} by optimizing the order of fields and rows of an input table with $n$ rows and $m$ fields. 
Each row may have a different field order. 
We represent a request schedule as a list of tuples $L$, where each tuple in $L$ represents a row in the table, and the tuple elements contain the field values. 
We adjust the row order by rearranging the tuples in $L$, and adjust the field order for that row by rearranging the elements within each tuple. 
We pass each tuple alongside the user question to form an input request to the LLM. %input requests and a prefix prompt to the LLM, where each row serves as an input request. 
% Each row is encoded in standard JSON format, with field names prepended to values to indicate their corresponding fields.


% Each tuple in $L$ corresponds to a row, and the elements of the tuple correspond to the values within the columns. 

We define the \textit{prefix hit count} (PHC) of $L$ as the number of consecutive field cell values shared with the previous row starting from the first cell, summing over all $n$ rows. 
Each cell value must exactly match the corresponding cell of the previous row (cannot be a substring), and cell values past the first must match consecutively (must be a prefix). 
% We assume that the previously seen prefix must match the entire value of a previous cell, and cannot be a substring. 
Formally, a cell in the list of tuples is denoted as $L[r][f]$, indicating the value in tuple $r$ at position $f$. Then, the PHC for a list of tuples $L$ with $n$ rows and $m$ fields is given by: %\asim{it's a little confusing to me that we use r and c for the indexing but then say n rows and m fields}
% $\text{PHC}(T) = \sum_{i=1}^{n} \max_{1 \leq j \leq m} \textit{hit}(T_{i,j})
% $. 
\vspace{-1.5em}

\begin{equation}
\text{PHC}(L) = \sum_{r=1}^{n} \textit{hit}(L, r)
\label{eq:phc}
\end{equation}
\vspace{-1.5em}

Here, the function $\textit{hit}(L, r)$ represents the prefix hit count for a single row $r$ in $L$. For simplicity, we assume that the input list is sorted. For each row $r$, the function checks if the value in each field $f$ matches any previously seen value in the same field of the previous row $r-1$. If all previous fields match, the hit count is the sum of the squares of the lengths of the values in those fields until a mismatch occurs. The squared lengths reflect the quadratic complexity of token processing in LLM inference, where each token computation depends on every preceding token and increases computational cost quadratically with input length. 
% \shu{make it clear about the Tuple index, tuple has order that represents column ordering; have to sort to make things work, previous rows (overlap), simplify by looking at, assuming at least one row fits into KV cache; n-1 (how much overlap, assume cache is large enough to reuse for previous few rows)}

% It is defined as the length of the concatenated string of cell values for row $i$ from column $1$ to $j$, if and only if all the concatenated cell values exactly match the previously seen concatenated cell values in the same columns.

% Defined over all previous rows 
% \begin{equation}
% \textit{hit}(L, r) = \max_{0 \leq c < m}
% \begin{cases} 
% \sum_{t=0}^{c-1} \text{len}(L[r][t])^2 & \text{if } \FORall t \leq c, L[r][t] = L[x][t] \\
% & \quad \exists x \text{ s.t., }  0 \leq x < r \\
% %& \quad \max\left(0, r -q \right) \leq x < r \\
% 0 & \text{otherwise}
% \end{cases}
% \label{eq:hit}
% \end{equation}
\vspace{-2em}

\begin{equation}
\textit{hit}(L, r) = \max_{0 \leq c < m}
\begin{cases} 
\sum_{f=1}^{c} \text{len}(L[r][f])^2 & \text{if } \forall f \leq c, \\ & L[r][f]= \\ & L[r-1][f] \\

%& \quad \max\left(0, r -q \right) \leq x < r \\
0 & \text{otherwise}
\end{cases}
\label{eq:hit}
\end{equation}
\vspace{-1.5em}


To simplify the design, we make two assumptions. 
First, we make a common assumption that at least one tuple (row) can fit into the KV cache to allow reuse.
Second, we assume that a cell value only counts as a hit if it exactly matches a previously seen value -- substring matches are not allowed. 
% Second, we assume that cell values do not have substring hits: a cell value must match the entire value of a previously seen cell to count as a hit. 
This is a reasonable assumption in relational databases, where exact value repetition is common and extensively leveraged by storage optimization techniques like run-length encoding~\cite{lemire2011reordering}. Column-oriented storage systems such as C-Store and Parquet~\cite{stonebraker2018c} also benefit from many exact repetitions in columnar data.
These assumptions simplify design and, as shown in Sec.~\ref{sec:evaluation}, demonstrate good real-world performance.


% This simplifies the design. 
% \shu{what are the implications of these assumptions on performance? for this one, can we say we simplify both prefix checks and the structure of hits, in reality this achieves good token hit rate improvements as well even within the field?}
%\shu{is this enough for justifying just look at previous row and sort?}
% \shu{pessimistic framing}



\subsection{Case Study: Fixed Field Ordering}
\label{subsec:casestudy}
% \shu{say at the beginning result up front, next we are going to show: can increase the hit rate by $m$ x of hit rate}
% \begin{figure}
%     \vspace{-1em}
%     \centering
%     % \includegraphics[width=0.49\textwidth]{figures/SIGMODfigures/prefix_hit_maximization.pdf}
%     \includegraphics[width=0.49\textwidth]{figures/MLSys_Figures/casestudy.pdf}
%     \vspace{-1.5em}
%     \caption{\textbf{Example of Fixed Field Ordering:} Comparing the PHC of fixed ordering to a better ordering in a $3x \times 3$ table. The green box represents cache hits of $x$ rows; the red box represents cache misses. The box marked with $G_{i}$ means $x$ rows contain the same values, otherwise assume the values are all distinct.}
%     \label{fig:optimal}
% \end{figure}

% For example, columnar storage systems like C-Store and Parquet~\cite{} leverage diverse repetitions in columnar data for compression. Run-length encoding (RLE) methods exploit complex table reordering to group data for storage efficiency~\cite{}. Multi-relational data mining~\cite{} and correlation analysis of tables~\cite{} highlight the existence of diverse data correlations and exploit them for query optimization.
% Additionally, fields may be filtered or searched differently by queries, leading to different data groupings based on access patterns. Database cracking [Idreos, 2007] and multi-dimensional clustering (MDC) [Chen, 2012], including Delta Lake Z-order [Armbrust, 2020], reorganize data based on workloads to improve query performance.

% Relational data typically has a fixed field order across rows, which can lead to lower hit rates in real-world relational databases that exhibit diversity in data patterns (described in Sec~\ref{sec:motivation}). 
Relational data typically uses a fixed field order across rows, which can lead to lower hit rates in real-world databases with diverse data patterns (Sec~\ref{sec:motivation}).
In fact, we show that using a fixed order can reduce the hit rate by up to $m$ times compared to a per-row field reordering.
To illustrate this, we begin with a simple example and extend it to show the potential impact of a naive fixed field ordering on prefix hit counts (PHC). 
% we analyze a case study showing 
% \ion{Say the result upfront, e.g., we show that with fixed re-ordering the hit rate can reduce by $m\times$ as compared to per-row field reordering; then, the rest should be about how you arrive to this result.}
First, consider a table $T$ with $n$ rows and $m$ fields arranged in an arbitrary (default) order. 
For simplicity, we assume each value is of length one.
In many cases, certain fields of an input table may contain highly unique values, like timestamps or IDs. 
In the worst case, suppose the first field of the table contains only unique values (Fig~\ref{fig:cases-1}), and the remaining $m-1$ fields contain the same value across all rows. 
This ordering yields $0$ PHC. 
A more optimized ordering (Fig~\ref{fig:cases-1}) will place the other $m-1$ fields first, yielding a PHC of $(n-1) \times (m-1)$. Each of the $n-1$ rows has a hit after the initial cold miss, and the length of each hit is $m-1$. 

% Suppose there are $k$ groups $G_1, ..., G_{k}$ in field $i$, each with $x$ rows containing the same value where $x < n$.
% For instance, selecting $G_{1,{i}}$ as the leading field for $x$ rows, and then $G_{2,{i+1}}$ or other groups for subsequent sections can raise the achievable PHC to $m \times (x-1)$. 
Now consider a scenario where the table contains groups of consecutive rows with identical values (not necessarily in the same field). 
Suppose each field $i$ has one such group with $x$ consecutive rows of the same value, with other $n-x$ rows having distinct values, where $n$ is the number of rows. 
We denote the group appearing in the $\text{Field}_i$ as $G_i$, so we have $G1, ..., G_{m}$ groups, where $m$ is the number of fields. 
Now, consider a scenario where groups in consecutive fields are non-overlapping across rows, as shown in Fig~\ref{fig:cases-2}. 
With fixed field reordering, the PHC of this structure is limited to $x-1$ no matter which field is prioritized. 
By contrast, a better ordering would rearrange the field order for different rows to prioritize groups with shared values. 
Fig~\ref{fig:cases-2} references a table with $3x$ rows and 3 fields. A naive fixed field ordering for all rows will result in misses on two groups, each with $x$ rows in $\text{Field}_{2}$ and $\text{Field}_{3}$. However, a better ordering will pick different $\text{Field}_{j}$ to prioritize for different rows, resulting in a 3 times higher hit rate of $3(x-1)$.

In the above scenario, PHC improvements from optimized field ordering can reach $m$ times that of a fixed field ordering. For example, there can be multiple (instead of just one) such groups in each field. 
% with equal length and frequency and groups in consecutive columns are non-overlapping rows
If each field contains roughly the same number of such groups, dynamic reordering for different rows can achieve as much as an $m$-fold improvement in PHC over fixed field ordering. 
Under the OpenAI pricing model, which charges half price for cached prompts, optimizing field order for a table with nine fields could yield 42\% in cost savings compared to fixed field ordering, assuming fixed ordering has a 10\% hit rate (e.g., $\frac{(x-1)}{n} = 10$). 
% If we have an input table of 9 fields, the fixed field ordering can be $9$ times worse than a better ordering. If fixed ordering has 10\%, then there is an ordering of hit rate 90\% which translate to 42\% potential cost savings under the OpenAI pricing models.
This example highlights the benefits of a more complex field reordering mechanism for different rows on PHC.

% This example underscores the performance gains achievable by adapting field order to align with shared prefixes, highlighting the impact of a more complex field reordering mechanism on PHC.
% Consider a pricing model that charges 2$\times$ less on cached prompt, the cost ratio difference will be $\frac{\frac{x}{2}+(1-x)}{\frac{mx}{2}+(1-mx)}$, which is approximately $1 + \frac{x(1-m)}{2}$. Consider a simple scenario of 



% Now imagine for example, there are some groups of values in the default ordering, first column, that does have some values. Assume that there are $i$ such groups from $G_1, ..., G_{i}$, each group has $x$ rows where $x < n$. We can always construct a case where if $i == 1$, and each column contains such group with non-overlapping rows compared to the previous or next column, then the PHC of such fixed column orderings is $x - 1$, where $-1$ account for the cold miss of such column. However, an obvious better ordering is to have different column orderings for different set of rows. For example, choosing $G1$ of column $1$ to be first, then for the next set of $x$ rows pick $G3$ of column $2$ to be first. In this case, the hit rate of such ordering will be $m \times (x-1)$. Thus, the ratio of PHCs comparing default orderings and a better orderings will be as high as $m$.  


\vspace{-0.5em}
\section{Recursive Request Reordering}
\label{sec:column-reordering}

% We propose an optimal request ordering algorithm, Optimal Prefix Hit Maximization (\optimal), to maximize PHC. 
% Given the exponentially expensive computational cost of the \optimal algorithm,  we also introduce Greedy Group Recursion (\greedy), an approximation of \optimal that leverages functional dependencies and table statistics to reorder requests for large tables efficiently. 
% We leverage full workload information from analytical queries to ensure requests sharing the same prefixes are executed consecutively. 

We now introduce our algorithms that re-arrange fields to maximize prefix sharing in the KV cache. 
We present an optimal recursive reordering algorithm that maximizes PHC (Sec~\ref{sec:optimal}) and 
introduce a greedy algorithm that efficiently approximates the optimal algorithm  (Sec~\ref{sec:greedy}).

%%%%%%%%%%%%% MLSYS ALGORITHM NEEDS RECONSTURCTION 
% \begin{algorithm}[t!]
% \caption{Optimal Prefix Hit Recursion (OPHR)}
% \begin{algorithmic}[1]
% \STATE \textbf{Input:} Table $T$
% \STATE \textbf{Output:} Prefix Hit Count $S$, Reordered List of Tuples $L$

% % \newcommand{\algorithmicfunction}{\textbf{function}}
% % \newcommand{\algorithmicendfunction}{\algorithmicend\ \algorithmicfunction}

% \FUNCTION{$\textsc{HitCount} (v, c, T)$ }
%     \STATE $R_v \gets \{i \mid T[i,c] = v\}$
%     \STATE {\bfseries Return} ${\text{len}(v)}^2 \times (|R_v| - 1)$
% \ENDFUNCTION

% \item[]
% \FUNCTION{$\textsc{Recurse}$ ($T$)}
%     \IF{$|T|_{rows} = 1$}
%         \STATE return 0, $[T[1]]$
%     \ENDIF 
%     \IF{$|T|_{cols} = 1$}
%         \STATE $S \gets \sum_{v \in \text{distinct}(T[,1])} \textsc{HitCount}(v, 1, T)$ % groupby or sort, choose best group, append 
%         \STATE {\bfseries Return} $S, sort([T[i] \mid i \in 1 \dots |T|_{rows}])$
%     \ENDIF
%     \STATE $max\_phc \gets -1$, $best\_L \gets T$

%     \COMMENT{For each distinct value $v$ in each column $c$}
%     \FOR{$c \in \text{columns}(T)$, $v \in \text{distinct}(T[,c])$} 
%         % \STATE $\text{distinct\_values} \gets \{T[i,c] \mid i \in 1 \dots |T|_{rows}\}$
%         % \FOR{$v \in \text{distinct\_values}$}
%         \STATE $R_v \gets \{i \mid T[i,c] = v\}$
%         \STATE $A\_HC, L_A \gets \textsc{Recurse}(T[\text{rows} \setminus R_v, \text{cols}])$
%         \STATE $B\_HC, L_B \gets \textsc{Recurse}(T[R_v, \text{cols} \setminus \{c\}])$
%         \STATE $C\_HC \gets \textsc{HitCount}(v, c, T)$
%         \STATE $phc \gets A\_HC + B\_HC + C\_HC$
%         \IF{$phc > max\_phc$}
%             \STATE $\max\_phc = phc$, 
%             \STATE $best\_L = [[v] + L_A[i] \mid i \in 1 \dots |R_v|] + L_B$
%         \ENDIF
%     \ENDFOR
%     \STATE $\textbf{return } \text{max\_phc}, \text{best\_L}$
% \ENDFUNCTION

% \STATE 
% \STATE \textbf{return } \textsc{Recurse}($T$)
% \end{algorithmic}
% \label{alg:optimal}
% \end{algorithm}


\begin{algorithm}[t!]
\caption{Greedy Group Recursion (GGR)}
\begin{algorithmic}[1]
\small
\STATE \textbf{Input:} Table $T$, Functional Dependency $FD$
\STATE \textbf{Output:} Prefix Hit Count $S$, Reordered List of Tuples $L$


\item[]
\FUNCTION{$\textsc{HitCount} (v, c, T, FD)$}
    \STATE $R_v \gets \{i \mid T[i,c] = v\}$
    \STATE $\text{inferred\_cols} \gets \{c' \mid (c, c') \in FD\}$
    \STATE $\text{tot\_len} = \text{len}(v)^2 + \sum_{\substack{c' \in \text{inferred\_cols}}} \frac{\sum_{r \in R_v} \text{len}(T[r, c'])}{|R_v|}$
    \STATE \textbf{return } $\text{tot\_len} \times (|R_v| - 1)$, $[c] + \text{inferred\_cols}$
\ENDFUNCTION


\item[]
\FUNCTION{\textsc{GGR}($T$, $FD$)}
    % \IF{$|T|_{rows} = 1$ or $|T|_{cols} = 1$}
    %     \STATE \textbf{return } \text{Base case processing as in \optimal}
    % \ENDIF
    \IF{$|T|_{rows} = 1$}
        \STATE return 0, $[T[1]]$
    \ENDIF 
    \IF{$|T|_{cols} = 1$}
        \STATE $S \gets \sum_{v \in \text{distinct}(T[,1])} \textsc{HitCount}(v, 1, T)$ % groupby or sort, choose best group, append 
        \STATE {\bfseries Return} $S, sort([T[i] \mid i \in 1 \dots |T|_{rows}])$
    \ENDIF

    \STATE $max\_HC, b\_v, b\_c, b\_cols \gets -1, \text{None}, \text{None}, []$

    \FOR{$c \in \text{columns}(T)$, $v \in \text{distinct}(T[,c])$}
        \STATE $HC, cols \gets \textsc{HitCount}(v, c, T, FD)$
        \IF{$HC > max\_HC$}
            \STATE $max\_HC, b\_v, b\_c, b\_cols = HC, v, c, cols$
        \ENDIF
    \ENDFOR
    
    \STATE $R\_v \gets \{i \mid T[i, b\_c] = b\_v\}$
    \STATE $A\_HC, L\_A \gets \textsc{GGR}(T[\text{rows} \setminus R\_v, \text{cols}], FD)$
    \STATE $B\_HC, L\_B \gets \textsc{GGR}(T[R\_v, \text{cols} \setminus b\_cols], FD)$
    \STATE $C\_HC, \_ \gets \textsc{HitCount}(b\_v, b\_c, T, FD)$
    \STATE $S \gets A\_HC + B\_HC + C\_HC$
    \STATE $L \gets [[b\_v] + L_A[i] \mid i \in 1 \dots |R\_v|] + L\_B$
    \STATE \textbf{return } $S, L$
\ENDFUNCTION

\item[]
\STATE \textbf{return } \textsc{GGR}($T$, $FD$)
\end{algorithmic}
\label{alg:greedy}
\end{algorithm}
\vspace{-0.5em}
\subsection{Optimal Prefix Hit Recursion (OPHR) } \label{sec:optimal}

%%%%%%%%%%%%%%%%%% Matei's feedback 
% \shu{need to explain: consider reordering rows but also columns, and doing it differently for each row. Why is it important to do it differently, if you don't add it you'll get $c$ times worse, show some base cases; might be losing accuracy. 

% Motivate: necessary to reorder both rows and columns somewhere (differently across different records), that's the new thing; why is that --> we prove that if you use a fixed column order, can be off by a factor of $c$ by hit rate. Imagine there's a group in each column. 

% Then: now that we know we need to choose different orders even within each row (optimal --> GGR) 

% Don't need to put the algorithm of optimal. Don't need. Example of how it can be off by a factor of c, show the picture. Motivate this. What's the claim (scientific question), obvious cases it does poorly 

% Don't add too many appendix, sglang it is ok to mention it is similar. Queries can be in there. }
%%%%%%%%%%%%%%%%%% Matei's feedback 
% It finds the \textit{optimal} PHC for a given table $T$ by assigning column orders for each row (Algorithm~\ref{alg:\optimal}). 
% table layout algorithm, table ordering algorithm 
Our Optimal Prefix Hit Maximization (\optimal) algorithm is a recursive algorithm that finds the \textit{optimal} PHC for a given table $T$ by considering all possible ways to split the table into a group of cells with the same value and two sub-tables. 
The algorithm takes as input a table $T$ and computes the optimal PHC $S$ along with a reordered list of tuples $L$. 
If $T$ only has one row or field, \optimal computes PHC and trivially returns the sorted $T$. 

In the recursive case, for each field $c$ in $T$, the algorithm identifies all distinct values $v$ in the field and the rows $R_v$ for which the field value is $v$. For each distinct value $v$, the table is split into two sub-tables: one of $T$ excluding rows $R_v$ and one of $R_v$ excluding field $c$. PHC for the currently selected value $v$ is calculated as the sum of the PHC of the sub-tables and the PHC contribution of $v$. \optimal evaluates all possible groups of distinct values in each field and selects the value that yields the maximum PHC. 

% \amog{If possible, writing the recurrence relation formula might be easier to digest instead of or in addition to describing in sentences? $PHC(T) = max_v(PHC(R_v[v]) + PHC(R_v[!v]) + PHC(T \setminus R_v))$}

Notably, the \optimal algorithm has exponential complexity with respect to the number of rows and fields due to its recursive nature and the combinatorial explosion of possible distinct value groupings (we present a more efficient algorithm in Sec~\ref{sec:greedy}). 




\begin{figure}
    \centering
    % \includegraphics[width=0.49\textwidth]{figures/SIGMODfigures/prefix_hit_maximization.pdf}
    \includegraphics[width=0.49\textwidth]{figures/MLSys_Figures/prefix_hit_maximization.pdf}
    \vspace{-1.5em}
    \caption{GGR picks the group with the maximum hit count at each step and calculates PHC as the sum of PHC of the elected group values (yellow box), the sub-table $T$ excluding rows $R_v$ (green box), and the sub-table of rows $R_v$ excluding the field where the value is located in (blue box).}
    \label{fig:optimal}
    \vspace{-2em}
\end{figure}

\textbf{Optimality Proof}
In the base case, the \optimal algorithm trivially computes the best PHC: for the single row case, the PHC is 0; for the single field case, the PHC is the sum of the squared lengths of distinct values multiplied by their occurrences minus one, which accounts for the initial miss when a value is seen the first time. 
Next, we prove optimality by induction.
For the inductive case, assume that the \optimal algorithm is optimal for any table with $k \leq n$ rows and $l \leq m$ fields. 
For a table $T$ with $n+1$ rows and $m+1$ fields, the algorithm iterates through each field $c$. For each distinct value $v$ in field $c$, we split $T$ into two sub-tables: $T_A$ (rows not containing $v$), and $T_B$ (rows containing $v$ but excluding field $c$). Based on the inductive hypothesis, \optimal optimally computes PHC for both sub-tables because it is optimal for tables with fewer rows and fields. The PHC for $T$ is the sum of PHC for $T_A$ and $T_B$, plus the contribution of $v$. When the distinct value $v$ is used to partition the table, its full contribution to the PHC is captured. If the table were not split based on distinct values, this contribution could be fragmented or lost due to non-contiguous groupings, leading to suboptimal PHC.
Thus, the \optimal algorithm ensures optimal reordering by selecting the best from all possible configurations.
\vspace{-0.5em}
\subsection{Greedy Group Recursion (\greedy) Algorithm}
\label{sec:greedy}
Due to the computational complexity of the \optimal, we propose a Greedy Group Recursion (\greedy) algorithm (Algorithm~\ref{alg:greedy}) that approximates \optimal.  
The \greedy algorithm takes an input table $T$ and returns the PHC $S$ along with a reordered list of tuples $L$. It has the same base case as the \optimal algorithm if $T$ only has one row or one field. At a high level, the \greedy algorithm recursively selects the value $b_v$ with the maximum prefix hit count (lines 3-8) at each recursion step (lines 17-23) rather than iterating through all possible distinct values in the entire table. 
It then prioritizes the field $b_c$ where this $b_v$ is in, splits the table into groups of cells of the same values and recurses on the two sub-tables (lines 24-26) and calculates the total PHC as the sum of PHC of the subtables and contributions of $b_v$ (line 28) similar to the \optimal algorithm. 

Since \greedy does not iterate through all possible distinct values but instead selects the one that gives the highest hit count at each step, the number of recursive calls is significantly reduced (i.e. the maximum depth of recursion is $O(\min(n, m))$, where the algorithm reduces dimensions of the table at each recursive step). However, at each recursive step, the cost of scanning to determine distinct values can result in quadratic complexity in terms of table size.

\vspace{-0.5em}
\subsubsection{Functional Dependencies} We leverage functional dependencies to reduce the number of fields the \greedy algorithm needs to consider at each recursion step. This insight helps improve both the approximation and efficiency of the algorithm, bringing it closer to the optimal solution without the need for extensive backtracking as in the \optimal algorithm. 
A functional dependency (FD) is a constraint between two sets of attributes in a relation from the data. For example, let $R$ be a relation schema and let $X$ and $Y$ be nonempty sets of attributes in $R$. We define an instance $r$ of $R$ that satisfies the FD $X \leftrightarrow Y$ if for every pair of tuples $t_1$ and $t_2$ in $r$: if $t_1.X = t_2.X$ then $t_1.Y = t_2.Y$ and vice versa. In our \greedy algorithm, FDs help narrow down the fields that must be considered at each recursion step. Specifically, when a value $v$ in field $f$ is selected for a given row, all fields functionally dependent on $f$ are ordered directly besides $f$ in the final ordering for that row (lines 5-6). As an example, if $R(A,B,C)$ is a table with attributes (fields) $A,B,C$ where we have an FD $A \leftrightarrow C$, field $C$ is not in consideration in our recursive steps when $A$ has already been included in the prefix.%, as $C$ will not provide a different PHC. and an $FD$ of $A \rightarrow C$ means that the only need to be processed once (lines 5-6). As an example, if $R(A,B,C)$ is a table with attributes (fields) $A,B,C$, $A$ and $C$ are considered functional dependencies if two rows have the same value for $A$, then they must have the same value for $C$, and vice versa. In the example where $A \rightarrow C$,
\vspace{-0.5em}

\subsubsection{Table Statistics} 
% \shu{we need to talk about these statistics are common, etc.}
% \accheng{don't be DB person, describe table statistics as fallback, put statistics first before mechanism}
To further reduce the algorithm runtime, we introduce an early stopping mechanism that halts recursion by specific recursion depth (row-wise sub-table recursion, column-wise sub-table recursion) or when a threshold $\mathtt{HITCOUNT}$ score calculated using table statistics is not exceeded. These statistics are generally widely available, such as the number of unique entries (i.e., cardinality) and the distribution of length of values for each field.
With this information, our \greedy algorithm estimates a $\mathtt{HITCOUNT}$ score for each field $c$ with $\mathtt{HITCOUNT}(C) = \mathtt{avg}(\mathtt{len}(c))^2$. This score denotes the expected contribution of a field to the PHC, accounting for the average length of the values and their frequency. Using these statistics, the algorithm can prioritize fields more likely to contribute to the PHC. 
Additionally, we can further improve the quality of the solution by establishing a fixed field ordering for the subtables using table statistics once the recursion stops. 
% Databases typically maintain statistics on stored data, such as the number of unique entries (i.e., cardinality) and the distribution of length of values for each field. \asim{do we want to mention databases here still?}
Early termination and falling back to table statistics allows \greedy to avoid scanning the table and performing recursion on real-world workloads. 
% \souj{consider discussing the use of table statistics before the early stopping mechanism!} 

\vspace{-0.5em}
\subsubsection{Achieving Optimal PHC} While our \greedy approximates the \optimal algorithm, it can achieve optimal PHC in certain cases. 
When the table has only one row or one field, \greedy matches \optimal by construction. 
When functional dependencies are accurate and cover all the fields of a table, \greedy can also identify the optimal solution. For instance, if one field $A$ functionally determines all other fields, then \greedy prioritize groups of values in $A$ due to the accumulated \texttt{HITCOUNT} score (line 3 in Algorithm~\ref{alg:greedy}), capturing key correlations early and producing the optimal reordering.
However, when fields tie in \texttt{HITCOUNT}, \greedy may be suboptimal, as it lacks the exhaustive search used by \optimal to resolve such ties. 
%We show more empirical results in real-world datasets comparing PHC between \greedy and \optimal in Appendix~\ref{appendix:hit-rate}.

%\vspace{-0.5em}

\section{Implementation}
We implement our algorithms in approximately 1.3K lines of Python code and evaluate them with PySpark~\cite{spark-sql}, which is backed by Apache Spark~\cite{zaharia2012resilient} -- a widely adopted large-scale data processing engine in industry.
The \textit{LLM operator} implements the actual
LLM inference by calling a configurable LLM endpoint. We implement this function as a UDF in PySpark. It takes in a system prompt, a query prompt, and a single row of data as input (Appendix~\ref{appendix:prompts}). 
The row and field orders are input based on the ordering returned by the reordering function. 
The operator is also responsible for \textit{prompt construction}. Specifically, it converts the user-provided question and the table row values into a prompt that an LLM can parse. We use JSON formatting to encode row values to indicate the relationship between field names and values to the LLM. 
%, ensuring the model correctly interprets the inputs provided. % \shu{wordings are weird with LLM UDFs} The UDF implements The first part of every prompt is the \textit{system prompt} containing instructions for the LLM. In the \textit{user prompt}, the question is provided as well as a JSON encoding of the field values for a particular row.

% \asim{can maybe cut this example? or move to appendix if we are doing one}
% For example, given the following query \amog{move this into a Figure instead of having it in the text}:
% \begin{minted}[fontsize=\small]{sql}
% SELECT LLM("QUESTION") FROM my_table
% \end{minted}



% The final prompt is constructed like so:

% System prompt: 
% \begin{minted}[fontsize=\small]{text}
% You are a helpful data analyst. You will receive JSON data containing various fields and their corresponding values, representing different attributes. Use these fields to provide an answer to the user query. The user query will indicate which fields to use for your response. Your response should contain only the answer and no additional formatting.
% \end{minted}

% User prompt: 
% \begin{minted}[fontsize=\small]{text}
% Answer the below query:
% QUESTION
% given the following data:
% {col_name1: col_value1, col_name2: col_value2, ...}
% \end{minted}
% Furthermore, UDFs can be added to support LLM invocation inside queries. 
% which we leverage for our experiments. 
%\asim{do we need to update this text to be more batch analytics focused?}
% \shu{remove or just add more, say we have a python package (?) or this is generally applicable to all dataframes etc. }
% 1. ReorderFn: applies the reordering algorithms as described in the previous section.

% 2. LLMFn: performs LLM inference on each row of the table.
% \textbf{Reordering Algorithm} We implement the \greedy algorithm as described in Section 3.3. Our implementation takes the entire table as input and possibly any precomputed statistics about each field, such as cardinality and average length. It also optionally takes in any known functional dependencies. After applying \greedy, we return a list of indexes indicating the row and field orders.
% 1. RowOrder: A single integer containing the global ordering of a particular row after MGH is applied.
% 2. ColOrder: A sequence of integers which represents the column ordering for a particular row,
% After the algorithm is applied, the entire table is sorted by the RowOrder column.
% \shu{change the notion of UDF to operator, write a SQL example or dataframe example here?}
% \section{Evaluation}
\label{sec:evaluation}
% \shu{Matei: if we have time, I'd love an experiments that checks whether accuracy is significantly affected by column reordering in the queries we do that in.}

In this section, we evaluate the effectiveness of our optimizations within a constructed benchmark suite of queries. We aim to answer the following questions: 
% \shu{@Audrey: double check these questions?} %\shu{reframe this as question, not findings}
% \begin{enumerate}
%     \item \accheng{summarize main findings similar to NoScope S9s}
%     \item \accheng{do we have any non-negligible overheads for any of our optimizations? if we don't have any, we should state this somewhere}
% \end{enumerate}

% \begin{enumerate}
%     \item \accheng{summarize main findings similar to NoScope S9s}
%     \item \accheng{do we have any non-negligible overheads for any of our optimizations? if we don't have any, we should state this somewhere}
% \end{enumerate}
% \begin{enumerate}
%     \item Changing the order of inputs we provide to the LLM, both within a single prompt and within a batch of prompts, can noticeably reduce query latency.%, up to 3.0$\times$ over unreordered inputs in our experimental results. 
%     \item LLM output generation time dominates query latency, and optimizations that reduce the number of inputs the LLM receives significantly improve performance.% We see up to 1.7$\times$ over passing in the full input list to the LLM in our experimental results. 
% \end{enumerate}

\begin{enumerate}
    \item How does the request reordering optimization impact query latency across different LLM query types and datasets? 
    \item What effects do standard optimizations like deduplication and SQL optimizations have on query latency? 
    \item How does our request reordering algorithm influence LLM accuracy and solver time? 
\end{enumerate}
% \begin{enumerate}
%     \item What is the impact of reordering the inputs we provide to the LLM on the query latency, both within a single prompt and for a batch of prompts? 
%     %can noticeably reduce query latency.%, up to 3.0$\times$ over unreordered in∂∑puts in our experimental results. 
%     \item Does reducing the number of inputs the LLM significantly impact end-to-end latency? Does LLM output generation time dominate query latency?
%     % We see up to 1.7$\times$ over passing in the full input list to the LLM in our experimental results. 
% \end{enumerate}

% and compare it against baselines without these optimizations. 

\subsection{Evaluation Benchmark}
\label{sec:queries}

% \begin{table}[t]
%     \centering
%     \resizebox{\linewidth}{!}{%
%         \begin{tabular}{ m{7.5em}  m{16em}}
%           \toprule
%           {\bf Parameter} & {\bf Description}  \\
%         %   \hline
%             \midrule
%           Retriever & Structured and unstructured database \\%\hline
%       Analyzer & LLM, other ML models \\%\hline
%       Pipeline  & Single-hop, Multi-hop \\%\hline
%       SQL types & Selection, Projection, Join, Avg \\%\hline
%       \bottomrule
%       \end{tabular}
%   }
%   \vspace{0.5em}
%     \caption{Benchmark Query Parameters. \asim{this table isn't referenced anywhere}}
%     \label{tab:benchmark-table}
% \end{table}

Given the lack of standard benchmarks for LLM queries, we construct a benchmark suite to represent real-world data retrieval and processing tasks. We create a range of query types over datasets from various sources to assess the impact of integrating LLMs into relational analytics.

% The integration of LLMs into batch analytics and SQL query processing introduces a novel paradigm, yet there exists no benchmark specifically designed to evaluate the performance of such systems. Thus, in our evaluation, we have developed a benchmark suite tailored to assess the efficiency of a variety of LLM queries. 


\subsubsection{Datasets}

% Skip the tables 
% \begin{table}[!t]
%     \centering
%     \begin{minipage}{\linewidth}
%         \centering
%         \begin{tabular}{|c|c|c|c}
%             \hline
%             \textbf{Field} & \textbf{Type} & \textbf{Example} \\
%             \hline
%             asin & identifier & 0741304058 \\
%             \hline
%             reviewText & string & ``A favorite cd now...'' \\
%             \hline
%             verified & boolean & True \\
%             \hline
%             overall & double & 5.0 \\
%             \hline
%             summary & string & ``Five Stars'' \\
%             \hline
%             Format & string & "MP3 Music" \\
%             \hline
%             description & list of strings & [``Great CD for babies...'', ``'', ``''] \\
%             \hline
%         \end{tabular}
%     \end{minipage}
%     \caption{Amazon Products Schema}
%     \label{tab:products_schema}
% \end{table}

% \begin{table}[!t]
%     \centering
%     \begin{minipage}{\linewidth}
%         \centering
%         \begin{tabular}{|c|c|c|c}
%             \hline
%             \textbf{Field} & \textbf{Type} & \textbf{Example} \\
%             \hline
%             rotten-tomatoes-link & identifier & m/10002114-dark-water\\
%             \hline
%             review-type & string & ``Fresh'' \\
%             \hline
%             review-content & string & ``Fun, brisk and imaginative'' \\
%             \hline
%             top-critic & boolean & True \\
%             \hline
%             movie-info & string & ``In this moody...'' \\
%             \hline
%         \end{tabular}
%     \end{minipage}
%     \caption{Rotten Tomatoes Movies Schema}
%     \label{tab:movies_schema}
% \end{table}

We build our benchmark suite on a variety of commonly used datasets for recommendation and natural language processing (NLP) models. \textbf{Amazon Product Reviews}\cite{amazon-product-review-dataset} is a recommendation dataset that contains product reviews and metadata from Amazon. We utilize the 2023 version of the dataset and select the "Handmade Products" category, which consists of 586.6K reviews. We use 15,060 rows of this dataset in our queries. % The schema of this dataset includes  the fields of interest for this dataset is shown in Table \ref{tab:products_schema}.
\textbf{Rotten Tomatoes Movie Reviews}\cite{rotten-tomatoes-movies-dataset} is a recommendation dataset that stores critic review data along with movie metadata from the popular movie review website Rotten Tomatoes. This dataset consists of 1,130,018 reviews. We use 15,018 rows of this dataset in our queries. %We show part of the schema with fields of interest in Table \ref{tab:movies_schema}. 
\textbf{Stanford Question Answering Dataset (SQuAD)} \cite{squad-dataset} is a reading comprehension dataset with more than 100,000 rows and consists of questions posed by crowdworkers on a set of Wikipedia articles. The context to every question is a segment of text, or span, from the corresponding reading passage. \textbf{Fact Extraction and Verification (FEVER)}\cite{fever} is a dataset consisting of claims that have been generated by altering sentences from a set of Wikipedia passages. The claims have been classified by human annotaters as either Supports, Refutes, or NotEnoughInfo if the claims is factually correct based off the Wikipedia passages. We use a deduplicated labeled dev set of Fever consisting of 22,862 claims. The Wikipedia dataset contains over 5 million passages.
% \begin{enumerate}
%     \item \textbf{Amazon Product Reviews}\cite{amazon-product-review-dataset} is a recommendation dataset that contains product reviews and metadata from Amazon. We utilize the 2023 version of the dataset and select the "Handmade Products" category, which consists of 586.6K reviews. We use 15,060 rows of this dataset in our queries. 
%     % The schema of this dataset includes  the fields of interest for this dataset is shown in Table \ref{tab:products_schema}.
%     \item \textbf{Rotten Tomatoes Movie Reviews}\cite{rotten-tomatoes-movies-dataset} is a recommendation dataset that stores critic review data along with movie metadata from the popular movie review website Rotten Tomatoes. This dataset consists of 1,130,018 reviews. We use 15,018 rows of this dataset in our queries. %We show part of the schema with fields of interest in Table \ref{tab:movies_schema}. 
%     \item \textbf{Stanford Question Answering Dataset (SQuAD)} \cite{squad-dataset} is a reading comprehension dataset with more than 100,000 rows and consists of questions posed by crowdworkers on a set of Wikipedia articles. The context to every question is a segment of text, or span, from the corresponding reading passage. 
%     % accheng{unclear here}
%     \item \textbf{Fact Extraction and VERification (Fever)}\cite{fever} is a dataset consisting of claims that have been generated by altering sentences from a set of Wikipedia passages. The claims have been classified by human annotaters as either Supports, Refutes, or NotEnoughInfo if the claims is factually correct based off the Wikipedia passages. We use a deduplicated labelled dev set of Fever consisting of 22,862 claims. The Wikipedia dataset contains over 5 million passages.
% \end{enumerate}
 


% \begin{itemize}
%     \item asin - “id” field, each product has a unique asin
%     \item reviewText - a plain-text user submitted review for a product.
%     \item verified - a boolean field that denotes whether a given product is "verified" on Amazon. 
%     \item description - from the metadata table, description contains a short description for the product.
% \end{itemize} 

% \begin{itemize}
%     \item rotten-tomatoes-link - like an “id” field, each movie has a unique rotten-tomatoes-link
%     \item review-type - “Fresh” or “Rotten” (used in filtering queries)
%     \item review-content - a plaintext user or critic review for a movie
%     \item top-critic - a boolean field that denotes whether the author for a review is a "top critic" on the website or not
%     \item movie-info - from the metadata table, this field contains a short description for the movie
% \end{itemize}
% \newline

\subsubsection{LLM Queries}\label{llmqueries}
Our benchmark suite incorporates a wide range of query types and use cases. We show examples of each query type below. 

% Our query benchmark suite is designed to explore the full spectrum of \sys's capabilities, incorporating a broad range of query types and use cases:
\vspace{8pt}

\textbf{\textit{Q1/Q5: LLM projection.}} This query type makes calls to an LLM within a \texttt{SELECT} statement to process information from specified database column(s). It reflects common tasks in data analytics in which the LLM is used for summarization and interpretation based on certain data attributes. Q1 passes in a subset of columns from the table, while Q5 passes in the entire table. 
\vspace{8pt}
\begin{mdframed}[linecolor=black, linewidth=.5pt]
\begin{minted}[fontsize=\small]{sql}
SELECT LLM('Recommend movies for the user based on {movie information} and {user review}', m.info, r.review)
FROM reviews r JOIN movies m ON r.link = m.link
\end{minted}
\end{mdframed} 
\vspace{8pt}
\begin{mdframed}[linecolor=black, linewidth=.5pt]
\begin{minted}[fontsize=\small]{sql}
SELECT LLM('Given the following fields, answer in ONE word, Yes or No, whether the movie would be suitable for kids.', 
    mr.*)
FROM ( SELECT r.*, m.* FROM reviews r JOIN movies m ON r.link = m.link ) AS mr
\end{minted}
\end{mdframed} 
\vspace{8pt}
\textbf{\textit{Q2: LLM filter.}} This query type leverages LLM for filtering data within a \texttt{WHERE} clause. The LLM processes and analyzes information to meet specific criteria, such as identifying positive reviews. This query type illustrates typical use cases in sentiment analysis and content filtering, which are important for application tasks such as customer feedback analysis and content moderation. 
\vspace{4pt}
\begin{mdframed}[linecolor=black, linewidth=.5pt]
    \begin{minted}[fontsize=\small]{sql}
SELECT m.movie_title
FROM Movies m JOIN Reviews r ON r.link = m.link
WHERE LLM('Analyze whether this movie would be suitable for kids based on {movie information} and {user review}', m.info, r.review) = 'Yes'  
AND r.rtype == 'Fresh'
    \end{minted}
    \end{mdframed} 
\vspace{8pt}

\textbf{\textit{Q3: Multi-LLM invocation.}} This query type involves multiple LLM calls in different parts of the query and addresses scenarios in which several layers of data processing or analysis are required. It represents advanced analytical tasks, such as combining different data insights or performing sequential data transformations.
\vspace{8pt}
\begin{mdframed}[linecolor=black, linewidth=.5pt]
\begin{minted}[fontsize=\small]{sql}
SELECT LLM('Recommend movies for the user based on {movie information} and {user review}', m.info, r.review) AS recommendations
FROM Movies m JOIN Reviews r ON r.link = m.link
WHERE LLM('Analyze whether this movie would be suitable for kids based on {movie information} and {user review}', m.info, r.review) = 'Yes'  
AND r.rtype = 'Fresh'
\end{minted}
\end{mdframed} 
\vspace{8pt}

\textbf{\textit{Q4: LLM aggregation.}} This query type incorporates LLM outputs into further query processing. For example, one such query could use LLMs to assign sentiment scores to individual reviews and then aggregate these scores to calculate an average sentiment for overall customer feedback. This query type is essential for tasks that need to extract insights from complex textual data. %clear, actionable overview 
\vspace{8pt}
\begin{mdframed}[linecolor=black, linewidth=.5pt]
\begin{minted}[fontsize=\small]{sql}
SELECT AVG(LLM('Rate a satisfaction score between 0 (bad) and 5 (good) based on {review} and {info}: ', r.review, m.info)) as AverageScore
FROM reviews r JOIN movies m ON r.link = m.link
GROUP BY m.movie_title
\end{minted}
\end{mdframed} 
\vspace{8pt}
    
\textbf{\textit{Q6: Retrieval-augmented generation (RAG)}.} This query type leverages external knowledge bases for enhanced LLM processing, enriching LLM queries with broader context. It simulates use cases where queries need to pull in relevant information from external sources, such as document databases or knowledge graphs, to provide comprehensive answers. 
\vspace{8pt}
\begin{mdframed}[linecolor=black, linewidth=.5pt]
\begin{minted}[fontsize=\small]{sql}
SELECT LLM('Given the following {context}, answer this {question}', VectorDB.similarity_search(s.question), s.question)
FROM squad s WHERE s.is_impossible = False
\end{minted}
\end{mdframed}
\vspace{8pt}
% \begin{enumerate}
%     \item \textbf{\textit{Q1: LLM projection.}} This query type makes calls to an LLM within a \texttt{SELECT} statement to process information from specified database column(s). It reflects common tasks in data analytics in which the LLM is used for summarization and interpretation based on certain data attributes.
    
%     \item \textbf{\textit{Q2: LLM filter.}} This query type leverages LLM for filtering data within a \texttt{WHERE} clause. The LLM processes and analyzes information to meet some specified criteria, such as identifying positive reviews. This query type illustrates typical use cases in sentiment analysis and content filtering, which are important for application tasks, such as customer feedback analysis and content moderation. 
    
%     \item \textbf{\textit{Q3: Multi-LLM invocation.}} This query type involves multiple LLM calls in different parts of the query and addresses scenarios in which several layers of data processing or analysis are required. It represents advanced analytical tasks, such as combining different data insights or performing sequential data transformations.

%     \item \textbf{\textit{Q4: LLM aggregation.}} This query type incorporates LLM outputs into further query processing. For example, one such query could use LLMs to assign sentiment scores to individual reviews and then aggregate these scores to calculate an average sentiment for overall customer feedback. This query type is essential for tasks that need to extract insights from complex textual data. %clear, actionable overview 

%     \item \textbf{\textit{Q5: LLM Projection (Entire Table)}.} This query type uses multiple columns for each of the Movies and Products dataset. Specifically, 7 columns are used for Movies and 8 are used for Products.  
    
%     \item \textbf{\textit{Q6: Retrieval-augmented generation (RAG)}.} This query type leverages external knowledge bases for enhanced LLM processing, enriching LLM queries with a broader context. It simulates use cases where queries need to pull in relevant information from external sources, such as document databases or knowledge graphs, to provide comprehensive answers. 
% \end{enumerate}

% \begin{enumerate}
%     \item \textbf{\textit{Q1: Single-Invocation of LLM}} These queries involve a one-time call to an LLM to retrieve or process information.
%     \item \textbf{\textit{Q2: Multiple Invocations of LLM}} Complex queries that necessitate multiple calls to LLMs, often involving different pieces of information or processing steps.
%     \item \textbf{\textit{Q3: RAG}} Leveraging retrieval-augmented generation to enhance query processing by incorporating external knowledge sources like vector databases. 
%     \item \textbf{\textit{Q4: Different SQL Operators}} Including LLMs in filters, joins, or aggregations.
% \end{enumerate} \accheng{since this query takes significantly longer to run?} \asim{yes -- it involves 2 invocations so it takes much longer}

We run Q1-Q5 on the Amazon Product Reviews and Rotten Tomatoes Movie Reviews datasets and Q6 on SQuAD and FEVER. We evaluate Q1-Q5 on around 15,000 rows of the Movies and Products datasets. For Q6, we evaluate roughly 20,000 questions/claims for both SQuAD and FEVER, where each question retrieves \textit{K=3} contexts to augment its answer. 



\subsubsection{Evaluation Metrics}
Our key evaluation metric is the end-to-end query execution time, the most relevant metric for running analytical queries. %\accheng{@shu, because this is what analytical queries care about...?}. 
Additionally, we analyze the prefix hit rate, which represents the ratio of prefix tokens served from the KV cache and the input token length. This metric corresponds directly to query latency speed-up from the LLM side.
% While our previous algorithms use PHC as an objective, assuming that the entire value hits on a cell without allowing substring hits, THR is a more precise measure that reflects the actual performance during LLM execution. Thus, we report THR in our evaluation.  

% We also collect statistics on LLM inference performance, including tokens per second (TPS) and requests per second (RPS) processed. 
% This is a metric to measure the overall effectiveness of our systems in utilizing cached KV to speed-up query processing. 
% \accheng{what's the diff between token hit rate and hit rate?}

\subsubsection{Experimental Setup} %Hardware and model configurations
% We collect results for the queries formulated on single L4 machines with 24GB GPU memory, with larger experiments run on A100 machines with 80GB memory. For experiments, we use the LLaMA model with 7B parameters. This model was both lightweight and inexpensive which we considered critical for workloads in this context.  collect results for the queries
We run experiments on a g2-standard-48 GCP instance (48vCPUs, 192GB RAM) with an NVIDIA L4 GPU accelerator hosted in the us-central1 region. For the LLM endpoint, we use the instruction tuned variant of Meta's LLaMA-3 model with 8B parameters~\cite{llama3}. This model is lightweight and inexpensive to host locally, making it well-suited to analytical tasks. We use vLLM~\cite{vllm} as our model serving engine. For RAG queries, we use a GTE embedding model  (Alibaba-NLP/gte-base-en-v1.5)\cite{li2023towards} to embed the context and use Facebook Similarity Search Library (FAISS) ~\cite{johnson2019billion} to store these context embeddings into an index.

% \accheng{need some more context here? what does FAISS stand for?} \asim{@shu can you elaborate on this above}
% \begin{enumerate}
%     \item All instances are hosted in the us-central1 region.
%     \item For LLM endpoint, we use Meta's LLaMA-2 model with 7B parameters \cite{}.
%     \item This model was both lightweight and inexpensive to host locally, which we considered critical for workloads in this context \asim{do we need this??}.
%     \item We use vLLM as our model serving engine with LRU as the prefix cache eviction policy.
%     \item For RAG queries, we use BGE embedding models (BAAI/bge-large-en-v1.5) to embed the context, and use FAISS to store these context embeddings into an index. 
% \end{enumerate}



\begin{figure*}[tbp]
     \centering
     \begin{subfigure}[b]{0.48\textwidth}
        \centering
        % \includegraphics[width=\textwidth]{figures/movies_runtimes_e2e.pdf}
        \includegraphics[width=\textwidth]{figures/SIGMODfigures/movies.pdf}
        \caption{Rotten Tomatoes Movies Dataset}
        \label{fig:movies-runtimes}
    \end{subfigure}
    \hfill
    \begin{subfigure}[b]{0.48\textwidth}
        \centering
        % \includegraphics[width=\textwidth]{figures/products_runtimes_e2e.pdf}
        \includegraphics[width=\textwidth]{figures/SIGMODfigures/products.pdf}
        \caption{Amazon Products Dataset}
        \label{fig:products-runtimes}
    \end{subfigure}

    %\vspace{-2em}
    \caption{End-to-end Result: Our optimizations (Cache (\greedy + Dedup + SQL Opt)) achieve 2.1 - 3.0$\times$ on Movie Dataset and 2.2 - 2.8$\times$ speed-up on Product Dataset over Cache with FIFO ordering (Cache(FIFO)). }
    \label{fig:runtimes}
\end{figure*}

% \begin{figure*}[tbp]
%      \centering
%      \begin{subfigure}[b]{0.48\textwidth}
%         \centering
%         \includegraphics[width=\textwidth]{figures/SIGMODfigures/squad.pdf}
%         \caption{SQuAD Dataset}
%         \label{fig:squad}
%     \end{subfigure}
%     \hfill
%     \begin{subfigure}[b]{0.48\textwidth}
%         \centering
%         \includegraphics[width=\textwidth]{figures/SIGMODfigures/fever.pdf}
%         \caption{FEVER Dataset}
%         \label{fig:fever}
%     \end{subfigure}

%     %\vspace{-2em}
%     \caption{End-to-end Result: Our optimizations (Cache (\greedy + Dedup)) achieve 2.21 $\times$ on SQuAD Dataset and 2.13 $\times$ speed-up on FEVER Dataset over Cache with FIFO ordering (Cache(FIFO))}.
%     \label{fig:rag-runtimes}
% \end{figure*}

\begin{figure}[ht]
    \centering
    \includegraphics[width=0.45\textwidth]{figures/SIGMODfigures/rag.pdf}
    \caption{End-to-end Result: Our optimizations, Cache (\greedy + Dedup + SQL Opt), achieve 2.2$\times$ on SQuAD Dataset and   2.1$\times$ speed-up on FEVER Dataset over Cache with FIFO ordering (Cache (FIFO)).} %\shu{These bars are too wide, need to narrow it a bit; change Cache (GGR) caption to Cache (GGR + Dedup + SQL Opt)}}
    \label{fig:rag-runtimes}
\end{figure}

% \begin{figure}[ht]
%     \centering
%     \includegraphics[width=0.45\textwidth]{figures/SIGMODfigures/multicol.pdf}
%     \caption{End-to-end Result: On a query using multiple columns (7 columns for Movies and 8 columns for Products), our optimizations (Cache \greedy + Dedup) achieve 2.6 $\times$ on both the Movies and Products Datasets over Cache with FIFO ordering (Cache FIFO)}
%     \label{fig:rag-runtimes}
% \end{figure}


\subsection{End-to-End Benchmark Results}
%\shu{Needs to add a few words about: why do we not evaluate the \optimal algorithm}
\textbf{\textit{Overview}}. Figures~\ref{fig:runtimes} and~\ref{fig:rag-runtimes} show the end-to-end latency results on our optimization techniques for our full benchmark suite. 
As baselines, we show the results of not using the KV cache for prefixes (No Cache) and caching without any reordering (Cache (FIFO)). We also measure the impact of caching with our algorithm detailed in Algorithm \ref{alg:greedy}, denoted as Cache (\greedy), and measure it without and with additional optimizations, such as deduplication and SQL optimization (i.e., Cache (\greedy+Dedup+SQL Opt)).  Our evaluation shows that our approach can achieve up to 5.7$\times$ speedup compared to baselines. 
We do not evaluate the optimal prefix hit recursion algorithm in Section~\ref{sec:optimal} as its exponential complexity makes it infeasible to run over large tables. Even for a small table, the runtime of the algorithm far exceeds the LLM inference time. For example, solving for the optimal ordering with the \optimal algorithm takes several minutes for a 10-row table. 
% We thus ignore Algorithm~\ref{alg:optimal} and 
Next, we discuss the evaluation for each query type and the other baselines in detail as below.
% We constrain the output token length for each experiment run so that we limit the token length variation whether or not we apply 

% We compare our techniques against several baselines: no KV cache, KV cache with FIFO ordering, KV

% including no KV cache and KV cache with original request order. 

% \begin{enumerate}
%     % \item Cache (with ordering): involves both column reordering and row sorting.
%     \item \accheng{need to incorporate these details somewhere}
%     \item Column reordering might change the LLM output. 
%     \item We constrain the output token length for each experiment run so that we make sure with and without column reordering, the output token length variation is not too much.
% \end{enumerate}

% Overall, we find noticeable gains in performance using our algorithm for reordering and caching over the naive and non-reordered caching approaches for each of our queries.
% \begin{enumerate}
%     \item End-to-end experiments were run including all of our optimizations.
%     \item Each query was run without prefix caching, using prefix caching without request order optimization, and using prefix caching with request order optimization.
%     \item Results for Q1-Q4 are shown in Figure \ref{fig:runtimes}. 
%     \item Results for Q5 are visible in Figure.
%     \item Our results show that the LLM calls dominate the end-to-end runtime of the query
%     \item Up to over \textbf{3x} speedup from naive batched requests using our optimizations.
% \end{enumerate}
   
% The optimizations discussed in section 4 are implemented and ablation experiments were run investigating each of them. 

\noindent \textbf{\textit{Q1: LLM projection.}} This query type applies the LLM to selected data for a given task. For the Movie dataset, we use the LLM to recommend some movies to a user based on their review of a given movie. For the Product dataset, we use the LLM to analyze whether the product quality inferred from a user's review matches the quality advertised in the product description.

Compared to our No Cache baseline, we achieve up to 3.7$\times$ speed-up on projection queries in the Movie dataset and 3.6$\times$ speed-up in the Product dataset. 
% and KV cache with FIFO ordering. \asim{these speedups are over no KV cache, not KV cache with FIFO}%\accheng{update names} 
This significant speed-up results from the sharing of large prefixes. We observe significant savings by avoiding recomputation on these longer prefixes. The No Cache baseline constructs a prompt for each row in the table and thus sends as many prompts to the LLM as there are rows in the table. No computation is saved by the model itself, and as a result, this method incurs the highest query runtime for each of our queries. 
% We benchmark against this method with our optimizations. 

% This makes sense, as Q1 was run with the longest instruction prompt for the LLM, including few-shot examples for how to answer the question asked. The total length of this prompt was 172 tokens in the Movie dataset and 141 tokens in the Product dataset. As a result, due to prefix caching, a lot of LLM recomputation is saved, as visible in the speedup from Naive to Cache(Naive). From there, cache token hits are magnified by our reordering optimizations, which brings us the remaining speedup.

We analyze the impact of our optimizations in detail on the Movie dataset. Cache (FIFO) provides a 1.6$\times$ speedup over No Cache since we can now reuse computed tokens for the instruction prompt. Our Cache (\greedy) algorithm ensures that the \textit{movie\_info} column is ordered first and groups requests with similar prefixes together to achieve a further speed-up of 2.0$\times$. Standard techniques like deduplication and SQL optimizations have minimal impact on this query type over the datasets we use. This is because the \textit{review\_content} column contains few duplicates (only 99 rows can be deduplicated), and this query does not include a filter clause. Thus, our ultimate speedup using Cache \greedy and deduplication is 3.7$\times$ over Naive and 2.3$\times$ over Cache (FIFO).

% Analyzing the Movie dataset results in detail, we see speedup with each optimization introduction. 
% \begin{enumerate}
%     \item The naive method constructs a prompt for each row in our table using the columns inputted to the UDF, and thus inputs 15,000 prompts to the LLM. No computation is saved by the model itself, and as a result this method incurs the highest query runtime.
%     \item Adding FIFO prefix caching provides X$\times$ speedup, as computation is saved caching the instruction prompt.
%     \item To make the caching more effective, we reorder our columns to be able to cache \textit{movie\_info} as well, and sort our rows by this column to make use of caching locality. With this technique, we achieve a further Y$\times$ speedup.
%     \item Our final optimizations include deduplication of inputs to the LLM and SQL filter optimization. The latter does not apply to Q1 as no filter clause is present. Because \textit{review\_content} is included as part of the prompt in this query, there are only a few (~100 rows) that are  deduplicated, and speedup is marginal.   
% \end{enumerate}

Our optimizations achieve similar improvements on the Product dataset. Cache (FIFO) improves query latency by 1.6$\times$ over the No Cache baseline. Cache (\greedy) achieves much higher speed-ups by mostly ordering the \textit{description} column first. This column contains many longer prefixes shared across different user reviews. Consequently, we achieve a 3.4$\times$ speedup over the No Cache baseline and a 2.2$\times$ speedup over Cache (FIFO). Like the Movie dataset, deduplication and filter reordering have less impact on this query type, so Cache (\greedy+Dedup+SQL Opt) achieves a 2.3$\times$ speedup over Cache (FIFO) overall. 
% There is no filter clause, and the \textit{text} column of user reviews contains many unique values. 300 rows are deduplicated, leading to a 2.3$\times$ speedup over only our caching techniques.

% The Product dataset results show similar improvement with each optimization.
% \begin{enumerate}
%     \item Adding FIFO prefix caching improves query latency only 1.1x over naive. This is significantly less speedup than in the Movie dataset, and is because the suffix length dominates the prompt token length. Specifically, the Fig\ref{tab:example-alg-values} shows the average token length of the \textit{reviewText} and \textit{desccription} input columns to be 381.54 and 282.56 tokens respectively. The prompt token length itself is only 141 tokens, so even when caching it, the majority of the LLM input must be recomputed.
%     \item Reordering mitigates the previous issue by allows us to also cache the \textit{description} column with the system prompt. Rows are sorted by this column to make use of caching locality. With this technique, we achieve a 1.7$\times$ speedup over the naive method and 1.5$\times$ speedup over naive caching.
%     \item Similar to the Movie dataset, the deduplication and SQL filter optimizations have less profound impact in this scenario. No filter cluase is present and \textit{reviewText} is mostly unique. Nonetheless, roughly 2000 rows are deduplicated leading to a further 1.4$\times$ speedup over only caching with reordering. 
% \end{enumerate}

% \accheng{missing explanation of why we achieve these wins. need to mention cache hit rate results and if the other two optimizations make a difference}


\noindent \textbf{\textit{Q2: LLM filter.}} This query type applies LLM as a filtering tool. For the Movie dataset, we filter rows based on a standard SQL operator \textit{review\_type} (with condition ``Fresh'') and an LLM operator (with condition 'Yes'). For the Product dataset, we use rows where the \textit{rating} column is equal to ``5.0'' to filter and the same LLM operator to filter for ``Yes'' for whether a product is suitable for kids.

Our algorithm Cache (\greedy) can achieve 5.6$\times$ speed-up over No Cache in the Movie dataset and 5.2$\times$ in the Product dataset. In the Movie dataset, Cache (FIFO) provides up to 1.9$\times$ speed-up over No Cache since the former saves computation by caching the instruction prompt. Cache (\greedy) provides a further speed-up of 2.1$\times$ by increasing prefix sharing.
The \textit{review\_content} column has few duplicates, so deduplication has minimal impact. On the other hand, SQL optimizations demonstrate significant performance benefits (e.g., 1.4$\times$ improvement over Cache (\greedy)) because the execution order between the non-LLM and the LLM filter impacts query latency. Pushing the non-LLM filter down first results in only 10,461 rows being passed to the LLM after the first filter (\textit{review\_type} == ``Fresh'') out of the total 15,008 rows in the table.

% Each optimization introduction introduces overall speedup in query latency. 

% In the Movie dataset, ...
% \begin{enumerate}
%     \item Adding FIFO prefix caching provides 1.7$\times$ speedup over naive, as computation is saved caching the instruction prompt.
%     \item Column and row reordering of our inputs provides further 1.6$\times$ speedup. We cache \textit{movie\_info} alongside the instruction prompt and sort our rows by this column to make use of caching locality.
%     \item Our final optimizations include deduplication of inputs to the LLM and SQL filter optimization. Once again \textit{review\_content} is included as part of the prompt in this query so few prompts are  deduplicated to the LLM. However, Q2 contains both a non-LLM and an LLM filter, the order of execution of which impacts query latency. Pushing the non LLM filter down means that only 10000 rows are passed in to the LLM after the first filter (\textit{review\_type} == ``Fresh'') out of the entire 15000 row table. This results in 1.7$\times$ improvement over caching with reordering. 
% \end{enumerate}

For the Product dataset, Cache (FIFO) provides a 1.8$\times$ speedup over the No Cache baseline. For Cache (\greedy), we order \textit{description} as the first column and cache it alongside the instruction prompt to increase the cache hit rate. For deduplication, the \textit{text} column has few duplicates, so this optimization has limited impact. Our SQL optimization enables us to push down the non-LLM filter so that only 11,994 rows out of the 15,059 are passed to the LLM after the first filter (\textit{rating} == 5.0). As a result, we achieve 1.4$\times$ improvement over only caching.

% In the Product dataset ... 
% \begin{enumerate}
%     \item Adding FIFO prefix caching provides 1.3$\times$ speedup over naive, as computation is saved caching the instruction prompt.
%     \item Column and row reordering of our inputs provides further 1.6$\times$ speedup over FIFO caching. We cache \textit{movie\_info} alongside the instruction prompt and sort our rows by this column to make use of caching locality.
%     \item Our final optimizations include deduplication of inputs to the LLM and SQL filter optimization. Once again \textit{review\_content} is included as part of the prompt in this query so few prompts are  deduplicated to the LLM. This query contains both a non-LLM and an LLM filter so the order of execution of these filters greatly impacts query latency. Pushing the non LLM filter down means that only 8874 rows are passed in to the LLM after the first filter (\textit{verified} == True) out of the entire 15000 row table. This results in 2.2$\times$ improvement over caching with reordering. 
% \end{enumerate}

% With reordering, we bring down the latency v.s. no reordering by X \%. With our SQL optimization techniques, we make sure that we apply the other filter condition first to reduce amount of inputs passed into LLM, then apply the LLM filter on reduced number of rows. We show that by pushing down cheaper predicates, we achieve a further 1.8$\times$ speedup on the Movie dataset and 2.78$\times$ on the Product dataset over only caching and reordering inputs.  \accheng{explain any difference between the two datasets or if no differences, why}

% For evaluation, we perform a query with constrained output to filter on based on the review and description columns as context for both datasets. \accheng{why mention constrained output here if we already mention earlier?}

% \sys is able to achieve X.Y$\times$ speedup in the Movie dataset and X.Y$\times$ in the Product dataset on filter queries. 
% \begin{enumerate}
%     \item Q2 uses the LLM as a filter. 
%     \item Non-LLM condition for both dataset, and LLM condition for both \shu{@asim}
%     \item Specifically, for Movie dataset we filter {add}, and for Product dataset we filter {add}
% \end{enumerate}

\noindent \textbf{\textit{Q3: Multi-LLM invocation.}} In this query, we combine Q1 and Q2 for each dataset. We first apply an LLM filter before invoking the LLM again for recommending movies and analyzing products in the \texttt{SELECT} statement. We achieve 4.8$\times$ speed-up over the No Cache baseline on the Movie dataset and 4.2$\times$ on the Product dataset on multiple invocation queries over the No Cache baseline.

For the Movie dataset, Cache (FIFO) provides 2.0$\times$ improvement over the No Cache baseline. Cache (\greedy) provides 1.7$\times$ improvement over Cache (FIFO). Our SQL optimization significantly impacts latency (2.4$\times$ speed-up over FIFO caching and 4.8$\times$ over no caching) for this query type. Since the non-LLM filter selects roughly 5,000 rows, as detailed in the previous query analysis, we significantly reduce the number of LLM invocations. 

For the Product dataset, Cache (FIFO) provides a 1.9$\times$ speedup over No Cache. Our optimizations of Cache (\greedy + Dedup + SQL Opt) provide a further 2.2$\times$ speed-up, leading to a total speedup of 4.2$\times$ over the No Cache baseline. 

% The primary latency improvement in the Q3 experiments once again comes from the SQL filter optimization. 

% Movies
% \begin{enumerate}
%     \item FIFO caching provides 1.7$\times$ improvement over naive method.
%     \item This query is essentially a combination of Q1 and Q2, so speedup is similar along with the causes for speedup.
%     \item Reordered caching provides 1.4$\times$ improvement over FIFO caching.
%     \item Applying SQL filter optimization provides an additional 1.3$\times$ speedup over caching with reordering.
% \item Filter chosen was same as Q2: \textit{review\_type == ``Fresh''}
%     \item Non LLM filter has a selectivity ratio of 0.7.
% \end{enumerate}

% Products
% \begin{enumerate}
%     \item FIFO caching provides 1.5$\times$ improvement over naive method.
%     \item Reordered caching provides an additional 1.3 $\times$ improvement over FIFO caching.
%     \item Applying SQL filter optimization provides an additional 2.2$\times$ speedup over caching with reordering. The filter selected was the same as Q2: \textit{verified == True}.
%     \item Non LLM filter has a selectivity ratio of 0.7.
% \end{enumerate}

% \begin{enumerate}
%     \item We construct the multiple invocation queries by first 
%     \item Next, a projection invocation to LLM is performed similar to Q1.
% \end{enumerate} 

\noindent \textbf{\textit{Q4: LLM aggregation.}} In this query, we use the \texttt{AVG} operator to aggregate the average sentiment score on the reviews column with the description column provided as context. For the Movie dataset, we group by \textit{movie\_title} and average over the LLM sentiment score output. For the Product dataset, we group by \textit{parent\_asin} and average over the LLM sentiment score output. We achieve a 3.6$\times$ speed-up in both the Movie and Product datasets over the No Cache baseline on aggregation queries using our optimizations. The results of this query type are similar to that of Q1, as the same columns are passed into the LLM with an instruction prompt of similar length.
 
For the Movie dataset, Cache (FIFO) provides a 1.8$\times$ speed-up over the No Cache baseline. Cache (\greedy) generates an additional 2.2$\times$ speed-up over Cache (FIFO) since the \textit{movie\_info} columns contain many shared values. Like Q1, there is no LLM-filter clause and few duplicates in the \textit{review\_content} column, so not much extra benefit is achieved with these optimizations. As a result, the query latency improvement with all optimizations is 3.6$\times$ over no caching.

For the Product dataset, Cache (FIFO) leads to a 1.6$\times$ speed-up over the No Cache baseline, and Cache (\greedy) brings a 2.2$\times$ speed-up over Cache (FIFO). Like Q1, the \textit{description} column is cached with the instruction prompt. There are marginal deduplication benefits with roughly 300 rows being removed, and the ultimate speedup is 3.6$\times$ over no caching.

% \begin{enumerate}
%     \item We use the AVG operator to aggregate an average sentiment score on the reviews column with the description column provided as context.
%     \item For Movie dataset, we group by \textit{movie\_title} and average over the LLM sentiment score output.
%     \item For Product dataset, we group by \textit{asin} and average over the LLM sentiment score output. 
% \end{enumerate}
% The results of this query are similar to that of Q1, as the same columns are passed in to the LLM with an instruction prompt of similar length. Specifically, the length of the instruction prompt was 166 tokens in the Movie dataset and 112 tokens in the Product dataset.

% Movies
% \begin{enumerate}
%     \item FIFO caching provides us with 1.9$\times$ speedup over the naive method as we cache the instruction prompt.
%     \item Reordered prefix caching adds an additional 1.8$\times$ speedup over FIFO caching, since the \textit{movie\_info} columns can be sorted on and cached. 
%     \item Like Q1, there is no SQL optimization to be made here and few duplicates because of the \textit{review\_content} column. As a result, the query latency is nearly identical to reordered caching. 
% \end{enumerate}

% Products
% \begin{enumerate}
%     \item FIFO caching leads to 1.4$\times$ speedup over the naive method as we cache the instruction prompt.
%     \item Reordered prefix caching brings 1.5$\times$ speedup over FIFO caching. Similar to Q1, the \textit{description} column is cached with the instruction prompt. 
%     \item Marginal deduplication benefits can be seen with a 1.1$\times$ improvement over reordered prefix caching with deduplication, with roughly 2000 rows being deduplicated from the original input.
% \end{enumerate}
\noindent \textbf{\textit{Q5: LLM Projection (Entire Table).}} We run a projection query for all seven columns for the Movies dataset and all eight columns from the Products dataset as detailed in Section~\ref{llmqueries}. We achieve a 3.7$\times$ speedup over the No Cache baseline on the Movie dataset and a 3.9$\times$ speedup on the Product dataset.

For the Movie dataset, Cache (FIFO) gets 1.4$\times$ speedup over No Cache baseline. For this query, we provide hints of functional dependencies to our algorithm, such as the \textit{`rotten\_tomatoes\_link'}, \textit{`movie\_info'}, and \textit{`movie\_title'} columns. These columns are grouped as they have one-to-one dependencies, so this is an FD for our algorithm. Cache (\greedy) provides a further 2.5$\times$ speedup over Cache (FIFO). It is unlikely to have exact duplicate values across seven columns, so speedup from adding deduplication is minimal. Thus, our final speedup is 3.7$\times$ over No Cache and 2.6$\times$ over Cache (FIFO). For the Product dataset, the columns of \textit{`product\_title'} and \textit{`asin'} have one-to-one dependencies, which serve as the input of the FD to our algorithm. Results on this dataset show that Cache (FIFO) is 1.5$\times$ faster than the No Cache baseline. From here, Cache (\greedy) achieves an extra 2.5$\times$ speedup over Cache (FIFO). The final speedup of our algorithm is 3.9$\times$ over No Cache and 2.6$\times$ over Cache (FIFO). 
% \begin{itemize}
%     \item We run a projection query using 7 columns from the Movies dataset and 8 columns from the Products dataset as detailed in \ref{llmqueries}.
%     \item Movies
%     \begin{itemize}
%         \item Cache (FIFO) gets 1.4$\times$ speedup over No Cache baseline.
%         \item Cache (\greedy) gets further 2.5$\times$ speedup over Cache (FIFO).
%         \item The functional dependencies here are the 'rotten\_tomatoes\_link', 'movie\_info', and 'movie\_title' columns. These are grouped together.
%         \item Not likely to deduplicate exact values across 7 columns, so speedup from adding deduplication is minimal.
%         \item Final speedup is 3.7$\times$ over No Cache and 2.6$\times$ over Cache (FIFO). 
%     \end{itemize}
%     \item Products
%     \begin{itemize}
%         \item Cache (FIFO) gets 1.5$\times$ speedup over No Cache baseline.
%         \item Cache (\greedy) gets further 2.5$\times$ speedup over Cache (FIFO).
%         \item Few duplicate values across columns.
%         \item Final speedup is 3.9$\times$ over No Cache and 2.6$\times$ over Cache (FIFO). 
%     \end{itemize}
% \end{itemize}


\noindent \textbf{\textit{Q6: RAG}}. This query is performed on a table of questions and the top three supporting evidence extracted from the SQuAD and FEVER datasets. We achieve a 3.5$\times$ speed-up on the SQuAD and 2.7$\times$ on FEVER over the No Cache baseline. In this experiment, we embed all supporting contexts for a question/claim into a FAISS index. We perform a K-nearest neighbor search on the vector index for each question to fetch the top K relevant contexts, where we choose $K = 3$. The embeddings and retrieval are computed before query time. At runtime, we apply our \greedy algorithm to the table of questions and contexts to maximize cache hits.

For the SQuAD dataset, Cache (FIFO) results in a 1.6$\times$ improvement over No Cache. Cache (\greedy) improves this further with 1.3$\times$ over Cache (FIFO). In this dataset, deduplication yields significant benefits because of the duplicated evidence lists, with only 9,561 prompts passed into the LLM after deduplication. Thus, the final speedup is 2.2$\times$ over Cache (FIFO) and 3.5$\times$ over No Cache.

For the FEVER dataset, Cache (FIFO) provides a 1.3$\times$ speedup over No Cache. Cache (\greedy) presents a further 1.9$\times$ speedup over Cache (FIFO). Roughly 3,000 out of 20,000 prompts are deduplicated. Thus, the final speedup is 2.1$\times$ over Cache (FIFO) and 2.7$\times$ over No Cache.

\noindent \textbf{Prefix Hit Rate.} We also measure the prefix hit rate (\%) for Cache (FIFO) and Cache (\greedy) for the query types in Figure~\ref{fig:runtimes}. This metric represents the ratio of tokens that can be served from the KV cache over all tokens in the input prompt. It indicates the effectiveness of the KV cache and is directly correlated with latency performance. On the Movie dataset, Cache (\greedy) achieves an average hit rate of 83.8\% while Cache (FIFO)'s average hit rate is 47.4\%. Across queries, Cache (\greedy) provides between a 22.9--45.7\% token hit rate improvement over Cache (FIFO). On the Product dataset, Cache (\greedy) achieves an average hit rate of 83.9\% while Cache (FIFO) has an average hit rate of 48.3\%. Cache (\greedy) overall can achieve 25.1--46.2\% hit rate improvement over Cache (FIFO) across different query types. %\accheng{maybe also add the raw values for context? what's a typical good cache hit rate for KV caches?}

% \subsection{Row-Reordering Ablations}
% We fix the column ordering for dataset and perform ablation experiments varying the row ordering across different queries. In this experiment, we run the same query as described in the previous section but without constraining the output token length. The end-to-end result shows that our approach achieves up to 2$\times$ on Movie, and 1.5$\times$ speedup on Product.

% \begin{table}[!t]
%     \begin{tabular}{l|r|r|l}
%     \toprule
%     Column Name   & \multicolumn{1}{l|}{ASL} & \multicolumn{1}{l|}{Cardinality} & Score \\ \midrule
%     ``description'' & 282.56                  & 144                     & 29460.80  \\ \hline
%     ``reviewText''  & 381.54                  & 12932                  & 442.97  \\ \hline
%     ``Format''       &  8.93                 & 16                     & 8379.69 \\ \bottomrule
%     \end{tabular}
%     \caption{Column statistics for Product table.}
%     \label{tab:products-alg-values}
%     \vspace{-2em}
% \end{table}
% % 

% \begin{table}[!t]
%     \begin{tabular}{l|r|r|l}
%     \toprule
%     Column Name   & \multicolumn{1}{l|}{ASL} & \multicolumn{1}{l|}{Cardinality} & Score \\ \midrule
%     ``movie\_info'' & 407.27                  & 68                     &  89946.78 \\ \hline
%     ``review\_content''  & 131.50                  & 14977                   & 131.86  \\ \hline
%     ``review\_type''       & 5.3                   & 2                     & 39797.7 \\ \bottomrule
%     \end{tabular}
%     \caption{Column statistics for Movie table.}
%     \label{tab:movies-alg-values}
%     \vspace{-2em}
% \end{table}


\begin{figure}[t!]
     \centering
     \begin{subfigure}[b]{0.48\columnwidth}
        \centering
        \includegraphics[width=\textwidth]{figures/SIGMODfigures/movies_hr.pdf}
        \caption{Movie Dataset}
        \label{fig:cdf_size}
    \end{subfigure}
    \hfill
    \begin{subfigure}[b]{0.48\columnwidth}
        \centering
        \includegraphics[width=\textwidth]{figures/SIGMODfigures/products_hr.pdf}
        \caption{Product Dataset} 
        \label{fig:cdf_freqs}
    \end{subfigure}
    \label{fig:cachehitrate}
    \caption{Cache Hit Rate Ablation. We illustrate the cache hit rate improvements achieved by Cache (\greedy) compared to Cache (FIFO), showing up to a 46\% increase on both the Product dataset and Movie datasets.}
\end{figure}

\begin{figure}[t!]
     \centering
    \includegraphics[width=.65\linewidth]{figures/SIGMODfigures/order_ablation.pdf}
    \caption{Request Ordering Ablation. The default request order is the order of the original table. Execution with our algorithm (\greedy Request Order) achieves 2.5$\times$ speedup and 2.9$\times$ speedup on end-to-end query latency on Movie and Product Dataset, respectively.}
    \label{fig:col_ordering}
\end{figure}

\subsection{Impact of Request Reordering} 
% \accheng{Movie and Product naming?}
To measure the effect of request reordering, we evaluate how overall query latency changes under varying request orders on an LLM projection query similar to Q1. For this set of experiments, three columns provide input data into the LLM invocations from both the Movie and Product datasets. Figure \ref{fig:col_ordering} shows the query latency results for the default request order as well as the best request order outputted by our algorithm. 

For the Movie dataset, we use the \circled{1} \textit{movie\_info}, \circled{2} \textit{review\_type}, and \circled{3} \textit{review\_content} columns in the query. The default column order in execution is \circled{3}:\circled{2}:\circled{1}, whereas the \greedy puts \circled{1} first most often, followed by \circled{2} and then \circled{3}. Unsurprisingly, caching the \textit{movie\_info} column, which is repeated across different reviews of the same movie, produces a large speed-up of 2.5$\times$ over the default ordering (\textit{review\_content} first). While the \textit{review\_type} column has many shared values (there are only two unique values across the entire dataset), the length of each field is one token in length since the value is either ``Fresh'' or ``Rotten''. Due to the quadratic cost of LLM inference in regards to input length, it makes more sense to cache the \textit{movie\_info} column first as it has the longest average token length, resulting in large prefixes that improve performance significantly when they are shared across requests.
% \accheng{how long is it? why is it short?} 

    % Movies
    % \begin{enumerate}
    %     \item For the movies dataset, we choose the \textit{movie\_info}, \textit{review\_type}, and \textit{review\_content} columns. 
    %     \item Column metadata is shown in Fig \asim{insert column metadata table}.  
    %     \item Unsurprisingly, caching the \textit{movie\_info} column, which is repeated for multiple reviews on the same movie, produces the fastest query runtime at 2.0x improvement over the worst ordering of \textit{review\_content} first. 
    %     \item While the \textit{review\_type} column has many shared values across the dataset with only 2 unique values, its length is too short to see prefix caching benefits. 
    %     \item The \textit{movie\_info} column also has the longest average length, and as a result the prompt prefix to suffix ratio is high. This further improves the speedup in caching this column. 
    % \end{enumerate}

     % \accheng{why is the formatting of the col names different?}\asim{I use the default column names from the dataset?} 

For the Product dataset, we choose the \circled{1} \textit{description}, \circled{2} \textit{rating}, and \circled{3} \textit{text} columns. The default column order in execution is \circled{3}:\circled{2}:\circled{1}, whereas \greedy puts \circled{1} first most often, followed by \circled{2} and then \circled{3}. Caching the \textit{description} column, which is repeated across reviews of the same product, produces a speed-up of 2.9$\times$ improvement over the default ordering (\textit{text} first). While the \textit{rating} column has many shared values across the dataset (with only five unique values total), its length is too short, containing integer values between one and five, that are one token in length.


% \accheng{how long is it? why is it short?}  \accheng{in general or under our query?}
% thus the ratio of prefix to suffix length for the prompts in the products dataset is lower. \accheng{do you mean reviewText avg len is shorter than movie\_info avg len?} \asim{no, it's longer. that's why there's not as much improvement as in Movie dataset, because here even though the description is cached, the suffixes are longer}

    % Products
    % \begin{enumerate}
    %     \item For the products dataset, we choose the \textit{description}, \textit{format}, and \textit{reviewText} columns. 
    %     \item Column metadata is shown in Fig \asim{insert column metadata table}.  
    %     \item As expected, caching the \textit{description} column, which is repeated for multiple reviews on the same product, produces the fastest query runtime at 1.5x improvement over the worst ordering of \textit{reviewText} first. 
    %     \item While the \textit{format} column has many shared values across the dataset with only 17 unique values, its length is too short to see prefix caching benefits. 
    %     \item The improvement here is less than the movies dataset for two primary reasons: (1) The description column in the products dataset isn't replicated as often for our query. (2) The \textit{reviewText} column has the longest average length, and thus the ratio of prefix to suffix length for the prompts in the products dataset is lower.
    % \end{enumerate}

\subsection{Impact of Standard Optimizations}

\subsubsection{Deduplication}
We investigate the effects of basic deduplication in detail for our setting. We construct queries based on Q1 while changing the specific LLM column inputs. Specifically, we vary the selection of columns passed into the LLM for analysis based on their cardinality. 
% \accheng{vary the column order?} \asim{better?} \accheng{still confused, do you mean: vary which columns are passed in based on their cardinality?} 
Before LLM invocation, we deduplicate exact input prompt matches and pass only the first occurrence of each distinct prompt into the LLM. Figure~\ref{fig:dedup} shows the results on the Movie and Product datasets.

% \accheng{is this correct (pass in only 2 cols at a time)?} \asim{yes this is how this experiment was run}
For the Movie dataset, we pass in the \textit{movie\_info} column alongside either the \textit{review\_type}, \textit{review\_score}, or \textit{review\_content} columns. For these columns, the deduplication speedup is 18.4$\times$, 4$\times$, and 1.1$\times$, respectively. This is expected, as the \textit{review\_type} column has only two distinct values while the \textit{review\_score} column has only five distinct values, so the query with these columns is often deduplicated. In contrast, the \textit{review\_content} column contains mostly unique values, with only 99 rows deduplicated out of 15,018.

\begin{figure}[t!]
     \centering
     \begin{subfigure}[b]{0.48\columnwidth}
        \centering
        \includegraphics[width=\textwidth]{figures/SIGMODfigures/movies_dedup.pdf}
        \caption{Movie Dataset}
        \label{fig:dedup_movie}
    \end{subfigure}
    \hfill
    \begin{subfigure}[b]{0.48\columnwidth}
        \centering
        \includegraphics[width=\textwidth]{figures/SIGMODfigures/products_dedup.pdf}
        \caption{Product Dataset} 
        \label{fig:dedup_products}
    \end{subfigure}
    \caption{Deduplication Ablation. Exact prompt matches are deduplicated before going into the LLM. For the lowest cardinality column on each of the Movie (review\_type: 2) and Product tables (rating: 5), deduplicating inputs leads to 18.4$\times$ and 9.3$\times$ faster query execution, respectively.}
    \label{fig:dedup}
\end{figure}

\begin{figure}[t!]
% \centering
% \includegraphics[width=0.48\textwidth]{figures/selectivity.pdf}
% \caption{Selectivity Ablation. Different columns are filtered on with resulting rows passed into LLM invocation. The lowest selectivity of the non LLM-filter for each the Movie table (0.13) yields a 6.7$\times$ faster query runtime than filtering with the LLM first.}
    \centering
     \begin{subfigure}[b]{0.48\columnwidth}
        \centering
        \includegraphics[width=\textwidth]{figures/SIGMODfigures/movies_selectivity.pdf}
        \caption{Movie Dataset}
        \label{fig:movie_selectivity}
    \end{subfigure}
    \hfill
    \begin{subfigure}[b]{0.48\columnwidth}
        \centering
        \includegraphics[width=\textwidth]{figures/SIGMODfigures/products_selectivity.pdf}
        \caption{Product Dataset} 
        \label{fig:product_selectivity}
    \end{subfigure}
    \caption{Selectivity Ablation. Different columns are filtered on with resulting rows passed into LLM invocation. The lowest selectivity of the non-LLM filter for each of the Movie and Product tables (0.3 and 0.2, respectively) yields 4.2$\times$ and 3.4$\times$ faster query runtimes than filtering with the LLM first.}
    \label{fig:selectivity}
\end{figure}

For the Product dataset, we pass in the \textit{description} column alongside either the \textit{rating}, \textit{review\_title}, or \textit{text} column.
% pass in the \textit{description} column with either the \textit{summary} or \textit{reviewText} column as LLM inputs. We also pair the \textit{title} column with the \textit{overall} column as input to the LLM. 
% \accheng{unclear what this means}
The \textit{rating} column, which captures the review score, has only five distinct values, so the query with this column has 9.3$\times$ faster runtime than the no deduplication baseline. On the other hand, the \textit{review\_title} and \textit{text} columns do not contain many duplicate values as they are mostly unique to each user. As a result, deduplication achieves only 1.1$\times$ speedup for each. 

% We investigate the effect of deduplicating requests to the LLM in query execution and observe the query runtime. 
% \begin{enumerate}
%     \item We construct queries similar to Q1, varying the columns passed in to the LLM for analysis based on the cardinality of the column.
%     \item Prior to LLM invocation, we deduplicate exact prompt matches and pass only the first occurence of the prompt into the LLM.
%     \item Movies
%     \begin{enumerate}
%         \item We pass in the \textit{movie\_info} column along with the \textit{review\_type}, \textit{review\_score}, and \textit{review\_content} columns.
%         \item \textit{review\_type} has only 2 distinct values, so the query with this column had a runtime 8.2x faster than no deduplication.
%         \item \textit{review\_score} similarly has only 5 distinct values, so the query with this column had a runtime 3.3x faster than no deduplication.
%         \item \textit{review\_content} is mostly unique, with only 99 rows being deduplicated out of 15000. As a result, the differences in runtime for this query with deduplication is negligble.
%     \end{enumerate}
%     \item Products
%     \begin{enumerate}
%         \item We pass in the \textit{description} column with the \textit{summary} and \textit{reviewText} columns and the \textit{title} column with the \textit{overall} column. 
%         \item The \textit{overall} column has just 5 values representing a review score, so the query with this column had a runtime 1.9x faster than no deduplication. 
%         \item Both the \textit{summary} and \textit{reviewText} do not share many duplicate values as they are properties of a user review, and as a result deduplication achieves only 1.1x speedup for each. 
%     \end{enumerate}
% \end{enumerate}
% Figure \ref{fig:multiple-invocations-runtimes} displays end to end time runtime 
% results for the queries listed in Figure \ref{fig:multi-invoke}. A 2.3x speedup is observed from naive execution to our optimized execution of the query for the movies dataset. A 1.9x speedup is observed with the Amazon products dataset. 

\renewcommand{\arraystretch}{1} % Adjust row height for better spacing



% \begin{table}[h]
%     \centering
%     \begin{tabularx}{\columnwidth}{lllcc}
%     \hline
%     \textbf{Dataset}                      & \textbf{Model}                                                                 & \textbf{Ordering} & \textbf{Mean}              & \textbf{90\% CI}             \\ \hline
%     \multirow{4}{*}{Movies}      & \multirow{2}{*}{\begin{tabular}[c]{@{}l@{}}Llama-3\\ 8B-Instruct\end{tabular}} & Original          & 88.0\%                     & 82-93\%                     \\
%                                           &                                                                                & GGR               & 88.0\%                     & 83-93\%                     \\
%                                           & \multirow{2}{*}{GPT-4o}                                                        & Original          & \multicolumn{1}{l}{89.0\%} & \multicolumn{1}{l}{84-94\%} \\
%                                           &                                                                                & GGR               & \multicolumn{1}{l}{92.0\%} & \multicolumn{1}{l}{87-96\%} \\ \hline
%     \multirow{4}{*}{Movies-Full} & \multirow{2}{*}{\begin{tabular}[c]{@{}l@{}}Llama-3\\ 8B-Instruct\end{tabular}} & Original          & 88.0\%                     & 83-93\%                     \\
%                                           &                                                                                & GGR               & 87.0\%                     & 81-92\%                     \\
%                                           & \multirow{2}{*}{GPT-4o}                                                        & Original          & \multicolumn{1}{l}{91.0\%} & \multicolumn{1}{l}{86-95\%} \\
%                                           &                                                                                & GGR               & \multicolumn{1}{l}{93.9\%} & \multicolumn{1}{l}{90-98\%} \\ \hline
%     \multirow{4}{*}{Fever}       & \multirow{2}{*}{\begin{tabular}[c]{@{}l@{}}Llama-3\\ 8B-Instruct\end{tabular}} & Original          & 44.2\%                     & 44-45\%                     \\
%                                           &                                                                                & GGR               & 56.5\%                     & 56-57\%                     \\
%                                           & \multirow{2}{*}{GPT-4o}                                                        & Original          & \multicolumn{1}{l}{73.5\%} & \multicolumn{1}{l}{73-74\%} \\
%                                           &                                                                                & GGR               & \multicolumn{1}{l}{70.7\%} & \multicolumn{1}{l}{70-71\%} \\ \hline
%     \end{tabularx}
%     \vspace{1mm}
%     \caption{The accuracy of the original column ordering v.s. the GGR ordering. For each order, the LLM output is compared against ground truth labels to determine the accuracy of the responses. For each dataset, we perform statistical bootstrapping \cite{bootstrapping} to get a distribution of accuracy measurements across 10,000 runs. Mean is the average across all the bootstrap runs, and 90\% CI represents a 90\% confidence interval for the accuracy.}
%     \label{tab:accuracyresults}

%     % \hspace{1cm}

% \end{table}
% \begin{table*}[ht]
% \centering
% \renewcommand{\arraystretch}{1.2}
% \setlength{\tabcolsep}{4pt} % Adjust column padding for compactness
% \begin{tabular*}{\textwidth}{@{\extracolsep{\fill}} l c cccccc cccccc cccccc}
% \hline
% \textbf{Model} & \textbf{Ordering} & \multicolumn{2}{c}{\textbf{Movies}} & \multicolumn{2}{c}{\textbf{Products}} & \multicolumn{2}{c}{\textbf{BIRD}} & \multicolumn{2}{c}{\textbf{PDMX}} & \multicolumn{2}{c}{\textbf{FEVER}} \\
%  & & \small \textbf{Mean} & \small \textbf{90\% CI} & \small \textbf{Mean} & \small \textbf{90\% CI} & \small \textbf{Mean} & \small \textbf{90\% CI} & \small \textbf{Mean} & \small \textbf{90\% CI} & \small \textbf{Mean} & \small \textbf{90\% CI} \\
% \hline

% Llama-3-8B-Instruct & Original & \small 94.0\% & \small 93-99\% & \small 95.9\% & \small 92-99\% & \small 67.0\% & \small 59-75\% & \small 81.2\% & \small 75-88\% & \small 44.2\% & \small -- \\
%  & GGR      & \small 97.0\% & \small 94-99\% & \small 95.0\% & \small 91-98\% & \small 67.0\% & \small 59-75\% & \small 74.0\% & \small 67-81\% & \small 56.5\% & \small -- \\
% \hline

% Llama-3-70B-Instruct & Original & \small 86.0\% & \small 80-92\% & \small 96.0\% & \small 93-99\% & \small 82.0\% & \small 76-88\% & \small 92.0\% & \small 87-96\% & \small -- & \small -- \\
%  & GGR      & \small 90.0\% & \small 85-95\% & \small 97.0\% & \small 94-99\% & \small 82.9\% & \small 77-89\% & \small 82.0\% & \small 75-88\% & \small -- & \small -- \\
% \hline

% GPT-4o & Original & \small 96.0\% & \small 93-99\% & \small 99.0\% & \small 97-100\% & \small 84.0\% & \small 78-90\% & \small 87.9\% & \small 82-93\% & \small -- & \small -- \\
%  & GGR      & \small 93.0\% & \small 89-97\% & \small 97.0\% & \small 94-99\% & \small 83.0\% & \small 77-89\% & \small 83.0\% & \small 77-89\% & \small -- & \small -- \\

% \hline
% \end{tabular*}
% \caption{The accuracy of the original column ordering v.s. the GGR ordering. For each order, the LLM output is compared against ground truth labels to determine the accuracy of the responses. For each dataset, we perform statistical bootstrapping \cite{bootstrapping} to get a distribution of accuracy measurements across 10,000 runs. Mean is the average across all the bootstrap runs, and 90\% CI represents a 90\% confidence interval for the accuracy.}
% \label{tab:accuracyresults}
% \end{table*}

% \begin{table*}[ht]
% \centering
% \renewcommand{\arraystretch}{1.2}
% \setlength{\tabcolsep}{2pt} % Reduce column padding for compactness
% \small
% \begin{tabular*}{\textwidth}{@{\extracolsep{\fill}} p{1.7cm}<{\centering} p{1.5cm}<{\centering} cccccc cccccc cccccc cccccc}
% \hline
% \multirow{2}*{\textbf{\small Model}} & \multirow{2}*{\textbf{\small Ordering}}  & \multicolumn{2}{c}{\textbf{\small Movies}} & \multicolumn{2}{c}{\textbf{\small Products}} & \multicolumn{2}{c}{\textbf{\small BIRD}} & \multicolumn{2}{c}{\textbf{\small PDMX}} & \multicolumn{2}{c}{\textbf{\small Beer}} & \multicolumn{2}{c}{\textbf{\small FEVER}} \\
%  & & \footnotesize Mean & \footnotesize 90\% CI & \footnotesize Mean & \footnotesize 90\% CI & \footnotesize Mean & \footnotesize 90\% CI & \footnotesize Mean & \footnotesize 90\% CI & \footnotesize Mean & \footnotesize 90\% CI & \footnotesize Mean & \footnotesize 90\% CI \\
% \hline

% \footnotesize{8B-Instruct} & \footnotesize{Original} & \small 94.0\% & \small 93-99\% & \small 95.9\% & \small 92-99\% & \small 67.0\% & \small 59-75\% & \small 81.2\% & \small 75-88\% & \small 86.9\% & \small 81-92\% & \small 44.2\% & \small -- \\
%  & \footnotesize{GGR}      & \small 97.0\% & \small 94-99\% & \small 95.0\% & \small 91-98\% & \small 67.0\% & \small 59-75\% & \small 74.0\% & \small 67-81\% & \small 81.1\% & \small 74-87\% & \small 56.5\% & \small -- \\
% \hline

% \footnotesize{70B-Instruct} & \footnotesize{Original} & \small 86.0\% & \small 80-92\% & \small 96.0\% & \small 93-99\% & \small 82.0\% & \small 76-88\% & \small 92.0\% & \small 87-96\% & \small 93.0\% & \small 89-97\% & \small -- & \small -- \\
%  & \footnotesize{GGR}      & \small 90.0\% & \small 85-95\% & \small 97.0\% & \small 94-99\% & \small 82.9\% & \small 77-89\% & \small 82.0\% & \small 75-88\% & \small 90.0\% & \small 85-95\% & \small -- & \small -- \\
% \hline

% \footnotesize{GPT-4o} & \footnotesize{Original} & \small 96.0\% & \small 93-99\% & \small 99.0\% & \small 97-100\% & \small 84.0\% & \small 78-90\% & \small 87.9\% & \small 82-96\% & \small 90.0\% & \small 85-95\% & \small -- & \small -- \\
%  & \footnotesize{GGR}      & \small 93.0\% & \small 89-97\% & \small 97.0\% & \small 94-99\% & \small 83.0\% & \small 77-89\% & \small 83.0\% & \small 77-89\% & \small 87.0\% & \small 81-92\% & \small -- & \small -- \\

% \hline
% \end{tabular*}

% \caption{The accuracy of the original column ordering v.s. the GGR ordering. For each order, the LLM output is compared against ground truth labels to determine the accuracy of the responses. We achieve \shu{@amog}}
% \label{tab:accuracy_results}
% \end{table*}

\begin{figure*}[tbp]
     \centering
     \begin{subfigure}[b]{0.33\textwidth}
        \centering
        % \includegraphics[width=\textwidth]{figures/movies_runtimes_e2e.pdf}
        \includegraphics[width=\textwidth]{figures/MLSys_Figures/8b.pdf}
        \caption{Meta-Llama-3-8B-Instruct}
        \label{fig:movies-runtimes}
    \end{subfigure}
    \hfill
    \begin{subfigure}[b]{0.33\textwidth}
        \centering
        % \includegraphics[width=\textwidth]{figures/products_runtimes_e2e.pdf}
        \includegraphics[width=\textwidth]{figures/MLSys_Figures/70b.pdf}
        \caption{Meta-Llama-3-70B-Instruct}
        \label{fig:products-runtimes}
    \end{subfigure}
    \begin{subfigure}[b]{0.33\textwidth}
        \centering
        % \includegraphics[width=\textwidth]{figures/products_runtimes_e2e.pdf}
        \includegraphics[width=\textwidth]{figures/MLSys_Figures/gpt.pdf}
        \caption{OpenAI GPT-4o}
        \label{fig:products-runtimes}
    \end{subfigure}

    \vspace{-1em}
    \caption{Accuracy of original v.s. \greedy ordering: we perform statistical bootstrapping to get a distribution of exact match accuracy measurements across 10,000 runs. The numbers indicate the difference in the median accuracy of \greedy compared to the original ordering.}
    % \SHU: put this into foot note: Note that since FEVER has 22,665 labeled rows, the bootstrapping results have much less variance compared to the other datasets.
    % For each dataset, we perform statistical bootstrapping \cite{bootstrapping} to get a distribution of exact match accuracy measurements across 10,000 runs. 
    % \caption{Box plot of exact match accuracy of the original column ordering v.s. \greedy ordering. For each dataset, we perform statistical bootstrapping \cite{bootstrapping} to get a distribution of exact match accuracy measurements across 10,000 runs. The numbers indicate the difference in median accuracy of \greedy ordering compared to the original ordering. Note that since FEVER has 22,665 labeled rows, the bootstrapping results have much less variance compared to the other datasets. \shu{make this font bigger}}
    \label{fig:accuracy}
    \vspace{-0.5em}
\end{figure*}















% \begin{table}
%     \centering
%     \begin{tabularx}{\columnwidth}{X X X X r}
%         \hline
%         \textbf{Dataset} & \textbf{Rows} & \textbf{Columns} & \textbf{Mean Solver Runtime (s)} \\ \hline
%         {Movies}   & 15019   & 7                & 20.7 \\ \hline
%         {Products} & 15058         & 8                & 26.2  \\ \hline
%         {SQuAD}    & 19928         & 4                & 33.1   \\ \hline
%         {FEVER}    & 22681         & 4                & 90.4          \\ \hline
%     \end{tabularx}
%     \vspace{1mm}
%     \caption{\greedy algorithm solver times across datasets. \greedy takes the longest on FEVER, followed by SQuAD, Products, and Movies, which have similar solver times.}   %\shu{re-explain the caption: rows, columns, etc.}} % We collect results with early\_stop as group size 1000 for FEVER and size 2 for the others.
%     \label{tab:solvertimes}
% \end{table}

\subsubsection{SQL Optimizations}
Finally, we investigate the effects of our SQL optimizations. Specifically, we evaluate the latency impact of varying the order in which filter clauses for LLM queries are applied. We construct queries identical in structure to Q2 and vary the column(s) to filter on, alongside the LLM predicate of whether the movie/product is suitable for kids. %\accheng{what is the LLM predicate?}). 
We measure the overall query runtime for two scenarios: (1) the LLM filter and (2) the non-LLM filters are executed first. 
We choose columns to filter on alongside the LLM based on their \textit{selectivity ratio}, measured as the ratio of LLM input size to the table size. Figure~\ref{fig:selectivity} shows query latency as a factor of selectivity ratio. 
% \accheng{what does choose columns mean? we just vary the selectivity ratio right?} \asim{different columns are being used to do filtering to get different number of inputs into LLM}

For the Movie dataset, we choose the columns \textit{review\_type} and \textit{top\_critic} to filter on. We construct queries filtering with each possible value in \textit{review\_type} (``Fresh'', ``Rotten'') and as well as a combination of values (\textit{review\_type} == ``Fresh'' \& top\_critic = False). We find that, as expected, query latency decreases as selectivity decreases since fewer inputs are being passed into the LLM. At the lowest selectivity level in this experiment (0.3), applying this filter order optimization yields a 4.2$\times$ faster overall runtime than executing the LLM filter first. For the Product dataset, we choose the columns \textit{rating} and \textit{price}. We construct query filtering with each possible value in \textit{verified} (True, False) along with a filter on \textit{price} > 15. Like the Movie dataset, query latency decreases with fewer inputs being passed in at a lower selectivity. At the lowest selectivity level in this experiment of 0.3, applying this filter order optimization yields a 3.4$\times$ faster runtime overall over executing the LLM filter first. 
% We investigate the effect of pulling up non-LLM filter clauses prior to the LLM filter clause in an ablation experiment and observe the query runtime.
% \begin{enumerate}
    % \item We construct queries identical in structure to Q2, and vary the column(s) to filter on alongside the LLM filter.
    % \item Written naively, these queries will execute the LLM filter first on the entire table before applying the non-LLM filters. 
    % \item We collect the runtime in two scenarios: (1) the LLM filter is executed first; (2) the non LLM filters are executed first. 
    % \item We choose columns to filter on alongside the LLM based on their \textit{selectivity ratio}, measured as the ratio of inputs to the LLM to the table size. \asim{does selectivity ratio need to be explained in a DB paper??}
    % \item Movies
    % \begin{enumerate}
    %     \item We choose the columns \textit{review\_type} and \textit{top\_critic}. We ran queries filtering with each possible value of \textit{review\_type} ("Fresh", "Rotten") and "top\_critic" (True, False), and additionally ran a query filtering with a combination of those values (\textit{review\_type} = Fresh \& top\_critic = True).
    %     \item We intuitively see that query runtime decreases as selectivity decreases. We were surprised to find that the improvement was linear with the number of requests, indicating similar request token length for different queries constructed in this dataset. 
    % \end{enumerate}
%     \item Products \accheng{products results not in yet}
%     \begin{enumerate}
%         \item We choose the columns "summary", "verified", and "overall" columns. 
%         \item We ran queries filtering the 
%     \end{enumerate}
% \end{enumerate}

% \todo{Use the stage breakdown plot, plot it in matplotlib, no gray grid, use large font (i.e. 20)}

% \subsubsection{SQL Optimization}
% We evaluate the speedup in optimizing execution within the SQL engine of complex queries with multiple filter steps at various selectivities for the non LLM filter. Results are shown in Figure \ref{fig:selectivity}. We can see an obvious and intuitive speedup in end-to-end runtime for execution when evaluating the non-LLM filter first, with the speedup increasing as selectivity decreases. 



\subsection{Algorithm Analysis} %\asim{this title is a bit confusing/vague}
\subsubsection{Impact of Reordering on Accuracy} %\shu{write in paragraph, no bullet points. for any paragraphs like this with one or two sentences more than a line just rephrase and make it shorter}
In this section, we evaluate the effect of our reordering algorithm \greedy on the accuracy of LLM queries. Since our algorithm changes the order in which data fields are placed in the prompt inputted into the model, we want to ensure that doing so does not affect the quality of the LLM outputs. We select the Movie and FEVER datasets for our experiment. For the Movie dataset, we run the filter expression from Q2 as described in Section~\ref{llmqueries}. The LLM outputs either "Yes" or "No" on whether a movie is suitable for kids given movie\_info and review\_description fields. We randomly sample 100 rows and manually label them as the ground truth for the query. We run the experiment with two columns (Movies) and seven columns (Movies-Full). For the Product dataset, we run Q5 from Section~\ref{llmqueries} on the Fever Dataset. Given a claim and three pieces of evidence, the LLM is asked to determine if the claim is factually correct, outputting SUPPORTS, REFUTES, or NOT ENOUGH INFO. The FEVER dataset already contains ground truth labels for all 22,682 rows.

We run all three experiments against the Llama-3-8B-Instruct model and GPT-4o. We measure accuracy as the percentage of exact matches between the LLM output and the ground truth labels. As Movies and Movies-Full only have 100 labeled rows, we employ statistical bootstrapping \cite{bootstrapping}. We perform 10,000 bootstrap runs, where on each bootstrap, we sample a new dataset from the original dataset with replacement and calculate the accuracy of this sampled dataset. This gives us a distribution of accuracy measurements across all 10,000 runs.

In Table~\ref{tab:accuracyresults}, we show the mean over all of the bootstrap runs and a 90\% confidence interval for accuracy. Across the board, we see that the accuracy distribution of \greedy ordering is within 5\% accuracy of the original ordering. The only exception is Fever with Llama3-8B, in which the ordering with \greedy performs significantly better than the original. This is due to the \greedy algorithm preferring to place the "claim" column at the end of the prompt instead of at the beginning. However, the same behavior does not hold for GPT-4o, showing that higher-quality models are more robust to column reordering.

\subsubsection{Algorithm Overheads}

Table~\ref{tab:solvertimes} shows the \greedy algorithm overheads on each of our experiment datasets. The solver is run on the Movies, Products, and SQuAD datasets with an early stopping threshold of group size 2. In the recursive process, if the max group found is size two or lower, we fall back to the column orderings for the sub-tables using table statistics.  We use an early stopping threshold of 1,000 on the FEVER dataset. The FEVER dataset requires the longest solving time using \greedy, with a mean solver time of 90 seconds. This dataset has the highest number of unique groups, so the recursive depth is also the highest. The Movies, Products, and SQuAD datasets require a solver time between 20 and 30 seconds, which is small compared to the LLM inference time, which is generally in the range of tens of minutes. Adding deduplication following our algorithm adds only linear time overhead, so its impact on end-to-end latency is minimal. The deduplication step can even be skipped if it is inferred from table statistics that there are many unique values.
% \shu{Add a few sentences on the algorithm overhead of dedup (over many distinct tables, say if the table statistics indicate that many distinct value, it does not worth doing dedup )} 

%\asim{shu: review this analysis}



% \begin{figure*}[tbp]
%      \centering
%      \begin{subfigure}[b]{0.48\textwidth}
%         \centering
%         \includegraphics[width=\textwidth]{figures/cache_hit_rate_movies_pdf.png}
%         \caption{Rotten Tomatoes Movies Dataset}
%         \label{fig:movies-runtimes}
%     \end{subfigure}
%     \hfill
%     \begin{subfigure}[b]{0.48\textwidth}
%         \centering
%         \includegraphics[width=\textwidth]{figures/cache_hit_rate_products_pdf.png}
%         \caption{Amazon Products Dataset}
%         \label{fig:products-runtimes}
%     \end{subfigure}

%     %\vspace{-2em}
%     \caption{Cache Hit Rate}
%     \label{fig:runtimes}
% \end{figure*}


% \begin{figure*}[tbp]
%      \centering
%      \begin{subfigure}[b]{0.33\textwidth}
%         \centering
%         \includegraphics[width=\textwidth]{figures/movies_runtimes.pdf}
%         \caption{Rotten Tomatoes Movies Dataset}
%         \label{fig:movies-runtimes}
%     \end{subfigure}
%     \hfill
%     \begin{subfigure}[b]{0.33\textwidth}
%         \centering
%         \includegraphics[width=\textwidth]{figures/products_runtimes.pdf}
%         \caption{Amazon Products Dataset}
%         \label{fig:products-runtimes}
%     \end{subfigure}
%     \hfill
%     \begin{subfigure}[b]{0.33\textwidth}
%         \centering
%         \includegraphics[width=\textwidth]{figures/squad_5000_50_cache2.pdf}
%         \caption{SQuAD Dataset}
%         \label{fig:squad}
%     \end{subfigure}
%     %\vspace{-2em}
%     \caption{Row Orderings Ablation: (a) (b) Movie and Product Datasets - our approach achieves 3 - 4.3$\times$ speedup on Movie, and 1.7 - 2$\times$ speedup on Product (c) our approach achieves 1.7 - 1.9 $\times$ speedup on SQuAD}
%     \label{fig:runtimes}
% \end{figure*}



% \begin{figure*}[tbp]
%      \begin{subfigure}[b]{0.48\linewidth}
%         \centering
%         \includegraphics[width=\textwidth]{figures/movies_col_runtimes_vllm.pdf}
%         \caption{Movie}
%         \label{fig:col_order_product}
%     \end{subfigure}
%     \hfill
%     \begin{subfigure}[b]{0.48\linewidth}
%         \centering
%         \includegraphics[width=\textwidth]{figures/products_col_runtimes_vllm.pdf}
%         \caption{Product}
%         \label{fig:col_order_product}
%     \end{subfigure}
%     \label{fig:variability}
%     \caption{Column Orderings Ablation: our approach achieves up to 2$\times$ on Movie, and 1.5$\times$ speedup on Product \shu{replot this, two bar for each dataset}}
% \end{figure*}

% \begin{figure*}[tbp]
%      \centering
%      \begin{subfigure}[b]{0.24\linewidth}
%         \centering
%         \includegraphics[width=\textwidth]{figures/movies_col_runtimes.pdf}
%         \caption{Movie: Description, Title, Review}
%         \label{fig:col_order_movie}
%     \end{subfigure}
%     \hfill
%     \begin{subfigure}[b]{0.24\linewidth}
%         \centering
%         \includegraphics[width=\textwidth]{figures/movies_col_runtimes_new.pdf}
%         \caption{Movie: Description, Type, Review}
%         \label{fig:col_order_product}
%     \end{subfigure}
%     \hfill
%     \begin{subfigure}[b]{0.24\linewidth}
%         \centering
%         \includegraphics[width=\textwidth]{figures/products_col_runtimes.pdf}
%         \caption{Product: Description, Title, Review}
%         \label{fig:col_order_product}
%     \end{subfigure}
%     \hfill
%     \begin{subfigure}[b]{0.24\linewidth}
%         \centering
%         \includegraphics[width=\textwidth]{figures/products_col_runtimes_new.pdf}
%         \caption{Product: Description, Format, Review}
%         \label{fig:col_order_product}
%     \end{subfigure}
%     \label{fig:variability}
%     \caption{Column Orderings (SGLang, Cache More Columns): }
% \end{figure*}

% \begin{figure*}[t]
% \centering
% \includegraphics[width=0.5\textwidth]{figures/movies_runtimes.pdf}
% \label{fig:movies-runtimes}
% \caption{End-to-end runtimes for movies queries.} 
% \end{figure*}
% \begin{figure}[tbp]
%     \begin{subfigure}[b]{.5\linewidth}
%         \centering
%         \includegraphics[width=\linewidth]{figures/selectivity.pdf}
%         \label{fig:select-movie}
%     \end{subfigure}
%     \begin{subfigure}[b]{.5\linewidth}
%         \centering
%         \includegraphics[width=\linewidth]{figures/selectivity.pdf}
%         \label{fig:select-movie}
%     \end{subfigure}
%     \caption{Selectivity: Movie and Product Datasets.}
%     \label{fig:selectivity}
% \end{figure}





%\vspace{-0.5em}

\section{Evaluation}
\label{sec:evaluation}
In this section, we evaluate the effectiveness of our optimizations within a constructed benchmark suite of queries. We aim to answer the following questions: 
\vspace{-1em}

\begin{enumerate}\setlength{\itemsep}{-1.5pt}
    \item How does our request reordering optimization impact query latency and costs across different LLM query types and datasets? 
    \item How does the request reordering algorithm influence LLM accuracy for different models?
    \item What is our algorithm solver time, and how does that compare to end-to-end query latency? 
\end{enumerate}

\vspace{-1em}

\subsection{Evaluation Benchmark}
\label{sec:queries}
Given the lack of standard benchmarks for LLM queries, we construct a benchmark suite to represent real-world data retrieval and processing tasks (Sec~\ref{subsec:dataset}). 
We define a range of query types (Sec~\ref{subsec:llmqueries}) over datasets from various sources to assess the impact of LLMs in relational analytics.


\subsubsection{Datasets}
\label{subsec:dataset}
\begin{table}[ht]
\centering
% \footnotesize
\small
\begin{tabular*}{\columnwidth}{@{\extracolsep{\fill}}cccccl}
\toprule
Dataset & \( n_{\text{rows}} \) & \( n_{\text{fields}} \) & \( \text{input}_{\text{avg}} \)  & \( \text{output}_{\text{avg}} \) & \text{Query Type} \\ 
\midrule
Movies & 15000 & 8 & 276 & $\{2, 29, 16, 2\}$ & T1-T4 \\ 
Products & 14890 & 8 & 377 & $\{3, 107, 62, 2\}$ & T1-T4\\
BIRD & 14920 & 4 & 765 & $\{2, 43\}$ & T1, T2 \\ 
PDMX & 10000 & 57 & 738 & $\{2, 72\}$ & T1, T2\\
Beer & 28479 & 8 & 156 & $\{2, 38\}$ & T1, T2\\ 
SQuAD & 22665 & 5 & 1047 & {11} & T5\\
FEVER & 19929 & 5 & 1302 & {3} & T5\\ 
% gsm & X & 1536 & 400 \\
\bottomrule
\end{tabular*}
\caption{Datasets: $n_{\text{rows}}$ and $n_{\text{fields}}$ denote the number of rows and fields, respectively. $\text{input}_{\text{avg}}$ and $\text{output}_{\text{avg}}$ represent average input and output token lengths. Query Type is detailed in Sec~\ref{subsec:llmqueries}. Since $\text{input}_{\text{avg}}$ remains consistent across query types, we report a single overall average, while $\text{output}_{\text{avg}}$ varies, with each bracketed value corresponding to a specific query type.} %\shu{@asim: add output token length for movies and products here}}
\label{tab:dataset}
\vspace{-1em}
\end{table}
% \shu{OpenAI --> create a dataset with prefix or repeating fields (run it, and simulated thing --> analyze)}

% \begin{table}[ht]
% \centering
% % \footnotesize
% \begin{tabular}{ccccl}
% \toprule
% Dataset & \( n_{\text{rows}} \) & \( n_{\text{fields}} \) & \( \text{input}_{\text{avg}} \)  & \( \text{output}_{\text{avg}} \) \\
% \midrule
% Movies & 14890 & 8 & 276 & \{29, 2\} \\ 
% Products & 15000 & 8 & 377 & \{107, 3\} \\
% BIRD & 14920 & 4 & 765 & \{43, 2\} \\ 
% PDMX & 10000 & 57 & 738 & \{72, 2\} \\
% Beer & 28479 & 8 & 156 & \{38, 2\} \\ 
% SQuAD & 22665 & 5 & 1214 & {11} \\
% FEVER & 19929 & 5 & 1047 & {13} \\ 
% % gsm & X & 1536 & 400 \\
% \bottomrule
% \end{tabular}
% \vspace{-0.5em}
% \caption{Dataset Configurations.}
% \label{tab:dataset}
% % \vspace{-1em}
% \end{table}

% We subselect the top xx of 15000 rows and 8 different fields from this dataset. % The schema of this dataset includes  the fields of interest for this dataset is shown in Table \ref{tab:products_schema}. %We show part of the schema with fields of interest in Table \ref{tab:movies_schema}. % This dataset consists of 28749 rows which we use for our experiments. % We use a deduplicated labeled dev set of Fever consisting of 22,862 claims. The Wikipedia dataset contains over 5 million passages.We select the "Handmade Products" category in our experiments.

We build our benchmark suite on 7 commonly used recommendation and natural language processing datasets, shown in Table~\ref{tab:dataset}. These datasets vary in the number of rows, fields, average input/output token lengths, and appropriate query types (Sec~\ref{subsec:llmqueries}). 
The datasets include Rotten Tomatoes Movie Reviews (Movies)~\cite{rotten-tomatoes-movies-dataset}, Amazon Product Reviews (Products)~\cite{amazon-product-review-dataset}, BIRD~\cite{li2024can}\footnote{We use Posts and Comments table joined by PostID from the BIRD dataset.}, Public Domain MusicXML (PDMX)~\cite{pdmx}, RateBeer Reviews (Beer)~\cite{ratebeer}, Stanford Question Answering Dataset (SQuAD)~\cite{squad-dataset}, and Fact Extraction and Verification (FEVER)~\cite{fever}. Details on the fields are in the Appendix~\ref{appendix:fields}.

% \textit{Rotten Tomatoes Movie Reviews (Movies)}~\cite{rotten-tomatoes-movies-dataset} contains critic reviews and movie metadata from Rotten Tomatoes. 
% \textit{Amazon Product Reviews (Products)}~\cite{amazon-product-review-dataset} includes "Handmade Products" user reviews and item metadata collected from Amazon in 2023.  
% \textit{BIRD}~\cite{} is a Text2SQL benchmark across multiple domains; we use Posts and Comments table from Codebase Community domain.
% \textit{Public Domain MusicXML (PDMX)}~\cite{} consists of MusicXML files from MuseScore, containing diverse multitrack symbolic music with song metadata.
% \textit{RateBeer Reviews (Beer)}~\cite{} is a beer reviews dataset including reviews about product and user information, followed by ratings of different aspects of beer. 
% \textit{Stanford Question Answering Dataset (SQuAD)}~\cite{squad-dataset} is a reading comprehension dataset consisting of questions posed by crowdworkers on Wikipedia articles. Each question includes a context—a text span from the corresponding passage.
% \textit{Fact Extraction and Verification (FEVER)}~\cite{fever} contains claims generated by altering sentences from Wikipedia passages, classified by human annotators as either Supports, Refutes, or NotEnoughInfo if the claims are factually correct based on the Wikipedia passages. Details on the field configuration of each dataset are provided in Appendix~\ref{}.
\vspace{-0.5em}

\subsubsection{LLM Queries}\label{subsec:llmqueries}
% Our query benchmark suite is designed to explore the full spectrum of \sys's capabilities, incorporating a broad range of query types and use cases:
% \vspace{8pt}

Our evaluation consists of 16 queries across 5 query types corresponding to different real-world use cases, as shown in Table~\ref{tab:dataset}. 
We discuss each query type below and provide details on queries for each dataset in Appendix~\ref{appendix:queries} and ~\ref{appendix:fields}. 

% We include 5 projection queries across the Movies, Products, BIRD, PDMX, and Beer datasets.

\textbf{\textit{(T1) LLM filter.}} Filter queries mimic SQL \texttt{WHERE} clauses and use LLMs to categorize data. This query type illustrates typical use cases in sentiment analysis, categorization, and content filtering.
Given their binary or categorical focus, these queries often yield short outputs (e.g., "Yes" or "No"). We construct five filter queries spanning all datasets except for SQuAD and FEVER. \newline 
\textbf{\textit{(T2) LLM projection.}} Projection queries use LLMs to summarize or interpret specific table field(s), similar to a SQL \texttt{SELECT} statement. 
% It reflects common tasks such as using LLMs for table summarization and interpretation based on certain data fields. 
These tasks typically produce longer outputs due to the descriptive nature of the results. We construct five projection queries spanning all datasets except SQuAD and FEVER. \newline 
% The LLM processes and analyzes information to meet specific criteria, such as 
% We include 5 filtering queries across the Movies, Products, BIRD, PDMX, and Beer datasets.
\textbf{\textit{(T3) Multi-LLM invocation.}} Multi-LLM queries involve sequential LLM calls (e.g., a filter followed by a projection), supporting tasks like multi-step data processing and combining insights. 
% , such as a projection following a filter. It represents advanced analytical tasks, such as combining different data insights or performing sequential data filterings. 
Output lengths vary by task but generally mix short and long responses.
We construct two example multi-LLM invocation queries on Movies and Products datasets. \newline 
%  and addresses scenarios in which several layers of data processing or analysis are required
% We include 2 multi-invocation queries on the Movies and Products datasets.
\textbf{\textit{(T4) LLM aggregation.}} 
Aggregation queries incorporate LLM outputs into aggregate functions, like averaging sentiment scores given by LLMs for individual reviews. These tasks usually generate concise numeric outputs for analysis (e.g., ratings of 1 to 5), resulting in shorter output lengths similar to filter queries. We construct two example aggregation queries on Movies and Products datasets. \newline 
% Aggregation queries use LLM-generated outputs in aggregate function, such as averaging sentiment scores from individual reviews, crucial for deriving insights from texual data. 
% Since these tasks often yield numerical outputs for aggregation, they also typically have short output token length, e.g. ratings from 1 to 5. 
\textbf{\textit{(T5) Retrieval-augmented generation (RAG)}.} RAG queries involve fetching external knowledge as context, such as retrieving relevant document segments before generating answers. We evaluate FEVER and SQuAD datasets, fetching 4 contexts for FEVER and 5 contexts for SQuAD for question answering.

% is a selection query with an additional step to fetch external knowledge as context for a given question from a VectorDB. It simulates use cases where queries need to pull in relevant information from external sources, such as document databases or knowledge graphs, to provide comprehensive answers. We include 2 RAG queries on the FEVER and SQuAD datasets, fetching K=4 contexts for FEVER and K=5 contexts for SQuAD during augmentation.

% For example, one such query could use LLMs to assign sentiment scores to individual reviews and then aggregate these scores to calculate an average sentiment for overall customer feedback. This query type is essential for tasks that need to extract insights from complex textual data. We include 2 aggregation queries on the Movies and Products datasets. %clear, actionable overview 

\begin{figure*}[tbp]
     \centering
     \begin{subfigure}[b]{0.48\textwidth}
        \centering
        % \includegraphics[width=\textwidth]{figures/products_runtimes_e2e.pdf}
        \includegraphics[width=\textwidth]{figures/MLSys_Figures/dataset_runtimes_comparison_Q2.pdf}
        \caption{Filter Queries}
        \label{fig:filter-q}
    \end{subfigure}
    \hfill 
    \begin{subfigure}[b]{0.48\textwidth}
        \centering
        % \includegraphics[width=\textwidth]{figures/movies_runtimes_e2e.pdf}
        \includegraphics[width=\textwidth]{figures/MLSys_Figures/dataset_runtimes_comparison.pdf}
        \caption{Projection and RAG Queries}
        \label{fig:selection-rag}
    \end{subfigure}
    % \shu{prefix recompute overhead, save memory, batch size gets larger; prefix has many requests, output token shrinks}
    \vspace{-1em}
    \caption{End-to-end Result (Filter, Selection, RAG): Our optimizations Cache (\greedy) achieve 1.5 -- 3.4$\times$ speed-up in end-to-end runtime over caching without reordering (Cache (Original)), and 1.8 -- 3.8$\times$ over No Cache baseline. }
    \label{fig:q1q5}
    \vspace{-1.3em}
\end{figure*}




% recompute + batch size (only help for decode throughput) 
% decode is memory bound, 2 requests in a batch v.s. 10 requests in a batch, decode step 
% 10 requests in a batch total throughput will be much higher (decode)
% batch size --> decode throughput 
% decode 2, batch size benefits 

% 2) decode token too short, more memory to put mroe prefix 
% decode throughput benefits reduce, or prefix 

% movies, products prompt is short, decode short, memory to put more prefix 

% Compute overhead (cache more in memory), recompute reduces 

% A -> B reduce: decode throughput benefits shrink, output token shrinks 

% recompute benefits (decode two times), B - compute brings benefitgs 
% A: generae make it slow, A - batch size similar (not too much overhead)
% Prefill benefits: generate more, amortized the benfit
% A: batch size similar (generate more, benefits amortized), output 少了, benefit 没了,batch size larger, decode多一步赚一点 
% batch size一样的,decode越长,throughput saturate 到decode,relative performance变小
% prefill快一点
% hit 多,prompt也场,batch size倍数差的不多(hit rate差很多,batch size差两倍)
    


% This query leverages external knowledge bases for enhanced LLM processing, enriching LLM queries with broader context. 

% \asim{we don't need this anymore due to dataset info table right?} We run Q1-Q5 on the Amazon Product Reviews and Rotten Tomatoes Movie Reviews datasets and Q6 on SQuAD and FEVER. We evaluate Q1-Q5 on around 15,000 rows of the Movies and Products datasets. For Q6, we evaluate roughly 20,000 questions/claims for both SQuAD and FEVER, where each question retrieves \textit{K=3} contexts to augment its answer. 


\vspace{-0.5em}
\subsubsection{Evaluation Setup}


% We also report the prefix hit rate as the ratio of prefix tokens served from the KV cache and the input token length. 
\textbf{Metrics} We evaluate \textit{end-to-end query latency} for each LLM query. We also measure the \textit{monetary cost} of using OpenAI and Anthropic endpoints. Additionally, we hand-label a subset of the LLM filter queries to evaluate the reordering implications for query \textit{accuracy}. \newline 
\textbf{Models} We run setups shown in Table~\ref{tab:dataset} using Meta Llama-3-8B-Instruct ~\cite{llama3}. For RAG queries, we use Alibaba-NLP/gte-base-en-v1.5~\cite{li2023towards} to embed the context and use Facebook Similarity Search Library (FAISS)~\cite{johnson2019billion} for context retrieval. We also run Llama-3-70B-Instruct~\cite{llama3} for LLM Filter queries. For cost results, we evaluate with OpenAI GPT-4o-mini and Anthropic Claude 3.5 Sonnet.  \newline 
\textbf{Hardware} We evaluate Llama-3-8B-Instruct on a single NVIDIA L4 GPU (GCP g2-standard-4) with 24GB of GPU Memory. We also run a larger model Llama-3-70B-Instruct on 8xL4 GPUs (GCP g2-standard-48). For OpenAI and Anthropic cost experiments, we utilize their API endpoints. \newline 
% on a g2-standard-48 GCP instance (48vCPUs, 192GB RAM) with an NVIDIA L4 GPU accelerator. 
% \textbf{Setup} We run our evaluations on a g2-standard-48 GCP instance (48vCPUs, 192GB RAM) with an NVIDIA L4 GPU accelerator. 
% We use the instruction tuned variant of Meta's LLaMA-3 model with 8B parameters~\cite{llama3}, with an additional experiment on a g2-standard-96 GCP instance (96vCPUs, 384GB RAM) using the 70B parameter model discussed in \ref{sec:modelablation}. 
% We use vLLM~\cite{vllm} as our model serving engine. For RAG queries, we use a GTE embedding model, Alibaba-NLP/gte-base-en-v1.5~\cite{li2023towards}, to embed the context and use Facebook Similarity Search Library (FAISS) ~\cite{johnson2019billion} to store these context embeddings into an index. 
\textbf{Baselines} Our algorithm (\textit{Cache (GGR)}) is compared against two baselines: one without prompt caching (\textit{No Cache}) and another with caching enabled but without reordering (\textit{Cache (Original)}). We do not evaluate the optimal prefix hit recursion algorithm (Sec~\ref{sec:optimal}) as it is infeasible over large tables (e.g., solving a 10-row table takes several minutes). 
The algorithm runtime far exceeds the LLM inference time for larger tables for the optimal algorithm. 
% This metrics reflects latency speed-up from LLMs.

% \begin{figure*}[tbp]
%      \centering
%      \begin{subfigure}[b]{0.48\textwidth}
%         \centering
%         % \includegraphics[width=\textwidth]{figures/movies_runtimes_e2e.pdf}
%         \includegraphics[width=\textwidth]{figures/MLSys_Figures/dataset_runtimes_comparison.pdf}
%         \caption{Selection (Q1) and RAG (Q5) Query}
%         \label{fig:movies-runtimes}
%     \end{subfigure}
%     % \hfill
%     % \begin{subfigure}[b]{0.48\textwidth}
%     %     \centering
%     %     % \includegraphics[width=\textwidth]{figures/products_runtimes_e2e.pdf}
%     %     \includegraphics[width=\textwidth]{figures/MLSys_Figures/dataset_runtimes_comparison_Q2.pdf}
%     %     \caption{Filter (Q2) Query}
%     %     \label{fig:products-runtimes}
%     % \end{subfigure}

%     %\vspace{-2em}
%     \caption{End-to-end Result: Our optimizations (Cache (\greedy + Dedup + SQL Opt)) achieve 2.1 - 3.0$\times$ on Movie Dataset and 2.2 - 2.8$\times$ speed-up on Product Dataset over Cache with FIFO ordering (Cache(FIFO)). \shu{merge all results into a single large graph}}
%     \label{fig:runtimes}
% \end{figure*}

% \begin{figure}[ht]
%     \centering
%     \includegraphics[width=0.45\textwidth]{figures/SIGMODfigures/rag.pdf}
%     \caption{End-to-end Result: Our optimizations, Cache (\greedy + Dedup + SQL Opt), achieve 2.2$\times$ on SQuAD Dataset and   2.1$\times$ speed-up on FEVER Dataset over Cache with FIFO ordering (Cache (FIFO)).} %\shu{These bars are too wide, need to narrow it a bit; change Cache (GGR) caption to Cache (GGR + Dedup + SQL Opt)}}
%     \label{fig:rag-runtimes}
% \end{figure}


\vspace{-0.5em}

\subsection{End-to-End Benchmark Results}
\label{sec:end-to-end}


%\shu{Needs to add a few words about: why do we not evaluate the \optimal algorithm}
\textbf{\textit{Overview}}. Fig~\ref{fig:q1q5} and Fig~\ref{fig:q3q4} show the end-to-end latency results of our techniques on LLM filter, projection, multi-LLM invocation, aggregation, and RAG queries with the Llama-3-8B-Instruct model on a single L4. Our evaluation shows that our approach can achieve 1.5 to 3.4$\times$ speedup over Cache (Original) and 1.8 to 3.8$\times$ speedup over No Cache across 16 queries. 
We discuss the evaluation for each query type in detail as below.
% As baselines, we show the results of not using the KV cache for prefixes (No Cache) and caching without any reordering (Cache (Original)). We also measure the impact of caching with our algorithm detailed in Algorithm \ref{alg:greedy}, denoted as Cache (\greedy).  

% For example, solving for the optimal ordering with the \optimal algorithm takes several minutes for a 10-row table. \asim{last two sentences can potentially be cut for space}

% and measure it without and with additional optimizations, such as deduplication and SQL optimization (i.e., Cache (\greedy+Dedup+SQL Opt))
\vspace{-0.5em}
\noindent \textbf{\textit{LLM filter.}} 
This query type uses an LLM operator to filter rows, often producing concise outputs of only a few tokens (see Table~\ref{tab:dataset}). Examples include question-answering tasks limited to 'Yes' or 'No' responses, or sentiment labels like 'Positive,' 'Negative,' or 'Neutral.' 
We construct five such queries on the datasets shown in Fig~\ref{fig:filter-q}. 
Our Cache (\greedy) approach achieves a 2.1 -- 3.8$\times$ speed-up over No Cache by caching repeated prefixes from system prompts and input data. 
Cache (Original) with prompt caching enabled can achieve a modest speedup of 1.03 -- 1.9$\times$ over No Cache by reusing instruction prompts and repeated values from the default input table. 
For queries with short decode stages, the primary benefit of prompt caching is the saved prefill computations. 
Our Cache (\greedy) algorithm further reduces end-to-end latency by 1.8 -- 3.0$\times$ over Cache (Original) through reordering rows and fields in the input table to maximize prefix reuse. 
% The main performance gain of prompt caching for this type of query where decode stage is short comes from saving the prefill computations. 

Most review datasets, such as Movies, Products, and BIRD, contain highly distinct values in the first few default fields due to the joining of reviews with metadata tables.
For instance, these tables often begin with a \texttt{review\_content} field. 
Our algorithm prioritizes fields with repeated values, like \texttt{description} and \texttt{product\_title}, leading to a 57 -- 74\% increase in prefix hit rates and a 2.5 -- 3$\times$ speed-up over the original ordering. 
PDMX is a dataset containing 57 fields with many unique, lengthy text entries. In this dataset, our algorithm raises the hit rate from an initial 12\% to 57\%, resulting in a 1.8$\times$ reduction in end-to-end latency. This lower speed-up is due to the nature of long input and 43\% of cache miss from this dataset even for Cache (GGR).
The Beer dataset contains some duplicated values in early fields like \texttt{review/profileName} and Cache (Original) can achieve an initial hit rate of 50\%. Cache (\greedy) can further increase the hit rate by an additional 30\% to reach 80\% and achieve a 2$\times$ speedup.

\vspace{-0.5em}
\noindent \textbf{\textit{LLM projection.}} 
This query type applies the LLM to the selected
data for a specific task, producing longer outputs ranging from 29 to 107 tokens (see Table~\ref{tab:dataset}). 
For example, LLMs can be used to summarize the positive aspects of movies leading to favorable ratings in the Movies dataset.
As shown in Fig~\ref{fig:selection-rag}, for datasets except for SQuAD and FEVER (i.e. RAG queries), Cache (GGR) achieves 2.4$\times$ to 3.7$\times$ speed-up over No Cache, and 1.5$\times$ to 3.4$\times$ speed-up over Cache (Original). 
Notice that as the output token length increases, query execution time across all baselines also grows. 
In cases where the decode stage dominates, benefits from prefill caching are less pronounced, leading to smaller relative performance gains than with LLM Filter queries with shorter output length. 
However, for datasets like BIRD and PDMX, which contain long strings, prompt caching saves memory during the decode stage, making the speedup more noticeable with longer decode times.

% Table~\ref{tab:dataset} reveals that our selection queries have the longest average output length from 38 to 107 tokens across datasets. With more output tokens, the decoding stage of LLM inference becomes a larger proportion of the end-to-end latency, during which prefix caching is irrelevant. As a result, although BIRD (765) and PDMX (738) have similar average input tokens, our methods achieve higher speed-up on BIRD, where the average output length is 43 tokens, as opposed to 72 tokens for PDMX. Additionally, datasets that result from a joined table of reviews and movies, products, or BIRD produce significantly more redundancy and more caching opportunities (e.g., the same movie description applies to multiple reviews) with higher speed-up.

\begin{figure}[t!]
    \centering
    \includegraphics[width=0.95\columnwidth]{figures/MLSys_Figures/dataset_runtimes_comparison_Q3_Q4.pdf}
    \vspace{-1em}
    \caption{End-to-end Result (Multi-LLM Invocation, Aggregation): Our optimizations Cache (\greedy) achieve 1.7 - 2.8$\times$ speed-up over Cache (Original), and 2.7 - 3.7$\times$ speed-up over No Cache. }
    \label{fig:q3q4}

\end{figure}

% \begin{table}[t!]
% \small
% \begin{tabularx}{\columnwidth}{l@{\hskip 4pt}c@{\hskip 4pt}c@{\hskip 4pt}c@{\hskip 4pt}c@{\hskip 4pt}c@{\hskip 4pt}c@{\hskip 4pt}c}
% \toprule
% \textbf{Method} & \textbf{Movies} & \textbf{Products} & \textbf{BIRD} & \textbf{PDMX} & \textbf{Beer} & \textbf{FEVER} & \textbf{SQuAD} \\
% \midrule
% \textbf{Original}  & 34.6            & 26.7              & 10.4          & 11.8          & 49.9          & 11.2           & 11.0 \\
% \textbf{GGR}    & 85.7            & 83.3              & 84.8          & 56.6          & 80.1          & 67.4           & 69.7 \\
% \bottomrule
% \end{tabularx}

% \vspace{-0.5em}
% \caption{Comparison of PHR (\%) averaging across query types for Naive and GGR across datasets.}
% \label{tab:algoresults_simplified}
% \end{table}

\begin{table}[t!]
\footnotesize
\setlength{\tabcolsep}{6pt} % Adjust column separation
\begin{tabularx}{\columnwidth}{l@{\hskip 2pt}c@{\hskip 2pt}c@{\hskip 2pt}c@{\hskip 2pt}c@{\hskip 2pt}c@{\hskip 2pt}c@{\hskip 2pt}c}
\toprule
\textbf{Method} & \textbf{Movies} & \textbf{Prods.} & \textbf{BIRD} & \textbf{PDMX} & \textbf{Beer} & \textbf{FEVER} & \textbf{SQuAD} \\
\midrule
\textbf{Original}  & 35\%            & 27\%               & 10\%           & 12\%          & 50\%           & 11\%           & 11\%  \\
\textbf{GGR}    & 86\%             & 83\%               & 85\%           & 57\%           & 80\%           & 67\%           & 70\%  \\
\bottomrule
\end{tabularx}

\caption{PHR (\%) of LLM Filter and RAG queries for Original and GGR, which achieves 30 -- 75\% higher hit rates.}
\label{tab:hit-rate}

\end{table}



% From this improvement, the simulated cost savings are 39\% and 79\% under OpenAI and Anthropic's pricing models respectively.
% \vspace{-0.5em}
% \caption{With solver overhead of less than 15 seconds across all datasets, GGR achieves up to 74\% PHR improvements. From this improvement, the simulated cost savings are 39\% and 79\% under OpenAI and Anthropic's pricing models respectively.}
% \label{tab:algoresults}
% \end{table}




% and KV cache with FIFO ordering. \asim{these speedups are over no KV cache, not KV cache with FIFO}%\accheng{update names} 
% These significant speed-ups result from the sharing of large prefixes. We observe significant savings by avoiding recomputation on these longer prefixes. The No Cache baseline constructs a prompt for each row in the table and thus sends as many prompts to the LLM as there are rows in the table. No computation is saved by the model itself, and as a result, this method incurs the highest query runtime for each of our queries. 
% We benchmark against this method with our optimizations. 

% This makes sense, as Q1 was run with the longest instruction prompt for the LLM, including few-shot examples for how to answer the question asked. The total length of this prompt was 172 tokens in the Movie dataset and 141 tokens in the Product dataset. As a result, due to prefix caching, a lot of LLM recomputation is saved, as visible in the speedup from Naive to Cache(Naive). From there, cache token hits are magnified by our reordering optimizations, which brings us the remaining speedup.

% We analyze the impact of our optimizations in detail on the Movie dataset. Cache (FIFO) provides a 1.6$\times$ speedup over No Cache since we can now reuse computed tokens for the instruction prompt. Our Cache (\greedy) algorithm ensures that the \textit{movie\_info} column is ordered first and groups requests with similar prefixes together to achieve a further speed-up of 2.0$\times$. Standard techniques like deduplication and SQL optimizations have minimal impact on this query type over the datasets we use. This is because the \textit{review\_content} column contains few duplicates (only 99 rows can be deduplicated), and this query does not include a filter clause. Thus, our ultimate speedup using Cache \greedy and deduplication is 3.7$\times$ over Naive and 2.3$\times$ over Cache (FIFO).


% Analyzing the Movie dataset results in detail, we see speedup with each optimization introduction. 
% \begin{enumerate}
%     \item The naive method constructs a prompt for each row in our table using the columns inputted to the UDF, and thus inputs 15,000 prompts to the LLM. No computation is saved by the model itself, and as a result this method incurs the highest query runtime.
%     \item Adding FIFO prefix caching provides X$\times$ speedup, as computation is saved caching the instruction prompt.
%     \item To make the caching more effective, we reorder our columns to be able to cache \textit{movie\_info} as well, and sort our rows by this column to make use of caching locality. With this technique, we achieve a further Y$\times$ speedup.
%     \item Our final optimizations include deduplication of inputs to the LLM and SQL filter optimization. The latter does not apply to Q1 as no filter clause is present. Because \textit{review\_content} is included as part of the prompt in this query, there are only a few (~100 rows) that are  deduplicated, and speedup is marginal.   
% \end{enumerate}

% Our optimizations achieve similar improvements on the Product dataset. Cache (FIFO) improves query latency by 1.6$\times$ over the No Cache baseline. Cache (\greedy) achieves much higher speed-ups by mostly ordering the \textit{description} column first. This column contains many longer prefixes shared across different user reviews. Consequently, we achieve a 3.4$\times$ speedup over the No Cache baseline and a 2.2$\times$ speedup over Cache (FIFO). Like the Movie dataset, deduplication and filter reordering have less impact on this query type, so Cache (\greedy+Dedup+SQL Opt) achieves a 2.3$\times$ speedup over Cache (FIFO) overall. 
% There is no filter clause, and the \textit{text} column of user reviews contains many unique values. 300 rows are deduplicated, leading to a 2.3$\times$ speedup over only our caching techniques.

% The Product dataset results show similar improvement with each optimization.
% \begin{enumerate}
%     \item Adding FIFO prefix caching improves query latency only 1.1x over naive. This is significantly less speedup than in the Movie dataset, and is because the suffix length dominates the prompt token length. Specifically, the Fig\ref{tab:example-alg-values} shows the average token length of the \textit{reviewText} and \textit{desccription} input columns to be 381.54 and 282.56 tokens respectively. The prompt token length itself is only 141 tokens, so even when caching it, the majority of the LLM input must be recomputed.
%     \item Reordering mitigates the previous issue by allows us to also cache the \textit{description} column with the system prompt. Rows are sorted by this column to make use of caching locality. With this technique, we achieve a 1.7$\times$ speedup over the naive method and 1.5$\times$ speedup over naive caching.
%     \item Similar to the Movie dataset, the deduplication and SQL filter optimizations have less profound impact in this scenario. No filter cluase is present and \textit{reviewText} is mostly unique. Nonetheless, roughly 2000 rows are deduplicated leading to a further 1.4$\times$ speedup over only caching with reordering. 
% \end{enumerate}

% \accheng{missing explanation of why we achieve these wins. need to mention cache hit rate results and if the other two optimizations make a difference}



% \shu{not accurate, but do one experiments; evidence and claim, take the text repeat itself K times, or select subset of FEVER that is long (duplicating it, double it, or select subset that's long), make the documents bigger, or only large documents; results with API models, subset of the things, API things have some limitations (1024), upfront all the things we run; one server with Llama, and one section on commercial }

% In the Movie dataset, Cache (FIFO) provides up to 1.9$\times$ speed-up over No Cache since the former saves computation by caching the instruction prompt. Cache (\greedy) provides a further speed-up of 2.1$\times$ by increasing prefix sharing.
% The \textit{review\_content} column has few duplicates, so deduplication has minimal impact. On the other hand, SQL optimizations demonstrate significant performance benefits (e.g., 1.4$\times$ improvement over Cache (\greedy)) because the execution order between the non-LLM and the LLM filter impacts query latency. Pushing the non-LLM filter down first results in only 10,461 rows being passed to the LLM after the first filter (\textit{review\_type} == ``Fresh'') out of the total 15,008 rows in the table.

% Each optimization introduction introduces overall speedup in query latency. 

% In the Movie dataset, ...
% \begin{enumerate}
%     \item Adding FIFO prefix caching provides 1.7$\times$ speedup over naive, as computation is saved caching the instruction prompt.
%     \item Column and row reordering of our inputs provides further 1.6$\times$ speedup. We cache \textit{movie\_info} alongside the instruction prompt and sort our rows by this column to make use of caching locality.
%     \item Our final optimizations include deduplication of inputs to the LLM and SQL filter optimization. Once again \textit{review\_content} is included as part of the prompt in this query so few prompts are  deduplicated to the LLM. However, Q2 contains both a non-LLM and an LLM filter, the order of execution of which impacts query latency. Pushing the non LLM filter down means that only 10000 rows are passed in to the LLM after the first filter (\textit{review\_type} == ``Fresh'') out of the entire 15000 row table. This results in 1.7$\times$ improvement over caching with reordering. 
% \end{enumerate}

% For the Product dataset, Cache (FIFO) provides a 1.8$\times$ speedup over the No Cache baseline. For Cache (\greedy), we order \textit{description} as the first column and cache it alongside the instruction prompt to increase the cache hit rate. For deduplication, the \textit{text} column has few duplicates, so this optimization has limited impact. Our SQL optimization enables us to push down the non-LLM filter so that only 11,994 rows out of the 15,059 are passed to the LLM after the first filter (\textit{rating} == 5.0). As a result, we achieve 1.4$\times$ improvement over only caching.

% In the Product dataset ... 
% \begin{enumerate}
%     \item Adding FIFO prefix caching provides 1.3$\times$ speedup over naive, as computation is saved caching the instruction prompt.
%     \item Column and row reordering of our inputs provides further 1.6$\times$ speedup over FIFO caching. We cache \textit{movie\_info} alongside the instruction prompt and sort our rows by this column to make use of caching locality.
%     \item Our final optimizations include deduplication of inputs to the LLM and SQL filter optimization. Once again \textit{review\_content} is included as part of the prompt in this query so few prompts are  deduplicated to the LLM. This query contains both a non-LLM and an LLM filter so the order of execution of these filters greatly impacts query latency. Pushing the non LLM filter down means that only 8874 rows are passed in to the LLM after the first filter (\textit{verified} == True) out of the entire 15000 row table. This results in 2.2$\times$ improvement over caching with reordering. 
% \end{enumerate}

% With reordering, we bring down the latency v.s. no reordering by X \%. With our SQL optimization techniques, we make sure that we apply the other filter condition first to reduce amount of inputs passed into LLM, then apply the LLM filter on reduced number of rows. We show that by pushing down cheaper predicates, we achieve a further 1.8$\times$ speedup on the Movie dataset and 2.78$\times$ on the Product dataset over only caching and reordering inputs.  \accheng{explain any difference between the two datasets or if no differences, why}

% For evaluation, we perform a query with constrained output to filter on based on the review and description columns as context for both datasets. \accheng{why mention constrained output here if we already mention earlier?}

% \sys is able to achieve X.Y$\times$ speedup in the Movie dataset and X.Y$\times$ in the Product dataset on filter queries. 
% \begin{enumerate}
%     \item Q2 uses the LLM as a filter. 
%     \item Non-LLM condition for both dataset, and LLM condition for both \shu{@asim}
%     \item Specifically, for Movie dataset we filter {add}, and for Product dataset we filter {add}
% \end{enumerate}

\vspace{-0.5em}
\noindent \textbf{\textit{Multi-LLM invocation.}} This query type combines Filter and Selection operations, beginning with an initial LLM filter (e.g., selecting positive reviews), followed by an LLM summarization of the filtered table. 
Applied to the Movies and Products datasets, as shown in Fig~\ref{fig:q3q4}, Cache (\greedy) achieves a 2.7$\times$ and 2.8$\times$ speedup over the No Cache baseline for Movies and Products, respectively. Compared to Cache (Original), Cache (\greedy) attains a speedup of 1.7$\times$ and 2.2$\times$. The relative speedup compared to Cache (Original) reduces for both datasets compared to Filter and Projection queries. This is because the first LLM invocation for filtering is over distinct reviews for sentiment analysis, so Cache (Original) and Cache (\greedy) performance will be similar, reducing the overall benefits. For Movies, this number reduces from 2.5$\times$ to 1.7$\times$ as the first invocation accounts for nearly half the query time; while for Products, the second invocation on Projection dominates runtime due to long decode output length (i.e., around 107), so we can still see 2.2$\times$ speed-up over Cache (Original).

% We integrate two queries of this type on the Movies and Products datasets as illustrated in Fig~\ref{fig:q3q4}. Our Cache (GGR) demonstrates a 2.7$\times$ and 2.8$\times$ speedup over the No Cache baseline for Movies and Products, respectively. 
% Additionally, compared with Cache (Original), Cache (GGR) achieves a speedup of 1.7$\times$ and 2.2$\times$ for each dataset. 

% For sentiment analysis in the first LLM invocation, sharing opportunities are limited due to mostly distinct reviews, reducing the speedup gain of Cache (GGR) over Cache (Original) for the Movies dataset, where this step accounts for nearly half the query time. In the Products dataset, however, the second invocation dominates runtime due to long decode output length (i.e. around 107). Overall, our approach demonstrates significant end-to-end latency improvements across complex query patterns involving both LLM filtering and summarization.

% For this query, an initial LLM Filter operation of sentiment analysis over user reviews has few sharing opportunities besides instruction prompts, because reviews are mostly distinct. As a result, the overall relative speed-up of Cache (GGR) over Cache (Original) reduces on the Movies dataset, where the runtime for this first invocation is nearly half the total query time. In the Products dataset, however, the second invocation dominates the runtime, so adding the first invocation for sentiment analysis does not affect speed-up substantially. 
% Our evaluation of these queries confirms that even while orchestrating complex patterns of analysis, including both LLM filtering and projection steps, our methods show notable end-to-end latency improvements.
% For the Movie dataset, Cache (FIFO) provides 2.0$\times$ improvement over the No Cache baseline. Cache (\greedy) provides 1.7$\times$ improvement over Cache (FIFO). Our SQL optimization significantly impacts latency (2.4$\times$ speed-up over FIFO caching and 4.8$\times$ over no caching) for this query type. Since the non-LLM filter selects roughly 5,000 rows, as detailed in the previous query analysis, we significantly reduce the number of LLM invocations. 

% For the Product dataset, Cache (FIFO) provides a 1.9$\times$ speedup over No Cache. Our optimizations of Cache (\greedy + Dedup + SQL Opt) provide a further 2.2$\times$ speed-up, leading to a total speedup of 4.2$\times$ over the No Cache baseline. 

% The primary latency improvement in the Q3 experiments once again comes from the SQL filter optimization. 

% Movies
% \begin{enumerate}
%     \item FIFO caching provides 1.7$\times$ improvement over naive method.
%     \item This query is essentially a combination of Q1 and Q2, so speedup is similar along with the causes for speedup.
%     \item Reordered caching provides 1.4$\times$ improvement over FIFO caching.
%     \item Applying SQL filter optimization provides an additional 1.3$\times$ speedup over caching with reordering.
% \item Filter chosen was same as Q2: \textit{review\_type == ``Fresh''}
%     \item Non LLM filter has a selectivity ratio of 0.7.
% \end{enumerate}

% Products
% \begin{enumerate}
%     \item FIFO caching provides 1.5$\times$ improvement over naive method.
%     \item Reordered caching provides an additional 1.3 $\times$ improvement over FIFO caching.
%     \item Applying SQL filter optimization provides an additional 2.2$\times$ speedup over caching with reordering. The filter selected was the same as Q2: \textit{verified == True}.
%     \item Non LLM filter has a selectivity ratio of 0.7.
% \end{enumerate}

% \begin{enumerate}
%     \item We construct the multiple invocation queries by first 
%     \item Next, a projection invocation to LLM is performed similar to Q1.
% \end{enumerate} 

\vspace{-0.5em}
\noindent \textbf{\textit{LLM aggregation.}} This query type uses \texttt{AVG} operator to aggregate the sentiment score on the reviews column with additional columns provided as context. We achieve a 3.5$\times$ speed-up in the Movies dataset and a 3.7$\times$ speed-up in the Products dataset over the No Cache baseline. We also achieve 2.5$\times$ speed-up on Movies and 2.8$\times$ speed-up on Products over Cache (Original). The results of this query type are similar to filtering query results, as the average output length is similar.

% For the Movie dataset, we group by \textit{movie\_title} and average over the LLM sentiment score output. For the Product dataset, we group by \textit{parent\_asin} and average over the LLM sentiment score output. 
 
% For the Movie dataset, Cache (FIFO) provides a 1.8$\times$ speed-up over the No Cache baseline. Cache (\greedy) generates an additional 2.2$\times$ speed-up over Cache (FIFO) since the \textit{movie\_info} columns contain many shared values. Like Q1, there is no LLM-filter clause and few duplicates in the \textit{review\_content} column, so not much extra benefit is achieved with these optimizations. As a result, the query latency improvement with all optimizations is 3.6$\times$ over no caching.

% For the Product dataset, Cache (FIFO) leads to a 1.6$\times$ speed-up over the No Cache baseline, and Cache (\greedy) brings a 2.2$\times$ speed-up over Cache (FIFO). Like Q1, the \textit{description} column is cached with the instruction prompt. There are marginal deduplication benefits with roughly 300 rows being removed, and the ultimate speedup is 3.6$\times$ over no caching.

% \begin{enumerate}
%     \item We use the AVG operator to aggregate an average sentiment score on the reviews column with the description column provided as context.
%     \item For Movie dataset, we group by \textit{movie\_title} and average over the LLM sentiment score output.
%     \item For Product dataset, we group by \textit{asin} and average over the LLM sentiment score output. 
% \end{enumerate}
% The results of this query are similar to that of Q1, as the same columns are passed in to the LLM with an instruction prompt of similar length. Specifically, the length of the instruction prompt was 166 tokens in the Movie dataset and 112 tokens in the Product dataset.

% Movies
% \begin{enumerate}
%     \item FIFO caching provides us with 1.9$\times$ speedup over the naive method as we cache the instruction prompt.
%     \item Reordered prefix caching adds an additional 1.8$\times$ speedup over FIFO caching, since the \textit{movie\_info} columns can be sorted on and cached. 
%     \item Like Q1, there is no SQL optimization to be made here and few duplicates because of the \textit{review\_content} column. As a result, the query latency is nearly identical to reordered caching. 
% \end{enumerate}

% Products
% \begin{enumerate}
%     \item FIFO caching leads to 1.4$\times$ speedup over the naive method as we cache the instruction prompt.
%     \item Reordered prefix caching brings 1.5$\times$ speedup over FIFO caching. Similar to Q1, the \textit{description} column is cached with the instruction prompt. 
%     \item Marginal deduplication benefits can be seen with a 1.1$\times$ improvement over reordered prefix caching with deduplication, with roughly 2000 rows being deduplicated from the original input.
% \end{enumerate}
% \noindent \textbf{\textit{Q5: LLM Projection (Entire Table).}} We run a projection query for all seven columns for the Movies dataset and all eight columns from the Products dataset as detailed in Section~\ref{llmqueries}. We achieve a 3.7$\times$ speedup over the No Cache baseline on the Movie dataset and a 3.9$\times$ speedup on the Product dataset.

% For the Movie dataset, Cache (FIFO) gets 1.4$\times$ speedup over No Cache baseline. For this query, we provide hints of functional dependencies to our algorithm, such as the \textit{`rotten\_tomatoes\_link'}, \textit{`movie\_info'}, and \textit{`movie\_title'} columns. These columns are grouped as they have one-to-one dependencies, so this is an FD for our algorithm. Cache (\greedy) provides a further 2.5$\times$ speedup over Cache (FIFO). It is unlikely to have exact duplicate values across seven columns, so speedup from adding deduplication is minimal. Thus, our final speedup is 3.7$\times$ over No Cache and 2.6$\times$ over Cache (FIFO). For the Product dataset, the columns of \textit{`product\_title'} and \textit{`asin'} have one-to-one dependencies, which serve as the input of the FD to our algorithm. Results on this dataset show that Cache (FIFO) is 1.5$\times$ faster than the No Cache baseline. From here, Cache (\greedy) achieves an extra 2.5$\times$ speedup over Cache (FIFO). The final speedup of our algorithm is 3.9$\times$ over No Cache and 2.6$\times$ over Cache (FIFO). 
% \begin{itemize}
%     \item We run a projection query using 7 columns from the Movies dataset and 8 columns from the Products dataset as detailed in \ref{llmqueries}.
%     \item Movies
%     \begin{itemize}
%         \item Cache (FIFO) gets 1.4$\times$ speedup over No Cache baseline.
%         \item Cache (\greedy) gets further 2.5$\times$ speedup over Cache (FIFO).
%         \item The functional dependencies here are the 'rotten\_tomatoes\_link', 'movie\_info', and 'movie\_title' columns. These are grouped together.
%         \item Not likely to deduplicate exact values across 7 columns, so speedup from adding deduplication is minimal.
%         \item Final speedup is 3.7$\times$ over No Cache and 2.6$\times$ over Cache (FIFO). 
%     \end{itemize}
%     \item Products
%     \begin{itemize}
%         \item Cache (FIFO) gets 1.5$\times$ speedup over No Cache baseline.
%         \item Cache (\greedy) gets further 2.5$\times$ speedup over Cache (FIFO).
%         \item Few duplicate values across columns.
%         \item Final speedup is 3.9$\times$ over No Cache and 2.6$\times$ over Cache (FIFO). 
%     \end{itemize}
% \end{itemize}

\vspace{-0.5em}
\noindent \textbf{\textit{RAG.}} This query is performed on a table of questions and the top four to five supporting evidence items extracted from the FEVER and SQuAD datasets. Cache (GGR) achieves a 1.9$\times$ speed-up on both FEVER and SQuAD over the No Cache baseline. We also achieve a 1.8$\times$ speed-up on FEVER and 1.7$\times$ on SQuAD over Cache (Original). 
In this experiment, we embed all supporting contexts for a question/claim into a vector index. We perform a K-nearest neighbor search on the vector index for each question to fetch relevant contexts.
At runtime, we apply our \greedy algorithm to the table of questions and contexts to maximize cache hits. Cache (\greedy) can achieve 56 -- 59\% prefix hit rate improvements over Cache (Original), as multiple questions might share similar contexts, and Cache (\greedy) can rearrange contexts to maximize prefix reuse. 
% Table~\ref{tab:dataset} highlights that the average input token length for these queries is much longer than our other queries: 1047 tokens in SQuAD and 1214 tokens in FEVER.
% Furthermore, as unique fields appear first in the dataset ('question' in SQuAD and 'claim' in FEVER), the original ordering achieves cache reuse only for the shared system prompt instruction prompt, making speed-ups similar over No Cache and Cache (Original). 

% As a result, the overall PHR is lower in these datasets than the recommendation and reviews datasets.
% For the SQuAD dataset, Cache (FIFO) results in a 1.6$\times$ improvement over No Cache. Cache (\greedy) improves this further with 1.3$\times$ over Cache (FIFO). In this dataset, deduplication yields significant benefits because of the duplicated evidence lists, with only 9,561 prompts passed into the LLM after deduplication. Thus, the final speedup is 2.2$\times$ over Cache (FIFO) and 3.5$\times$ over No Cache.

% For the FEVER dataset, Cache (FIFO) provides a 1.3$\times$ speedup over No Cache. Cache (\greedy) presents a further 1.9$\times$ speedup over Cache (FIFO). Roughly 3,000 out of 20,000 prompts are deduplicated. Thus, the final speedup is 2.1$\times$ over Cache (FIFO) and 2.7$\times$ over No Cache.


% \noindent \textbf{Prefix Hit Rate.} We also measure the prefix hit rate (\%) for Cache (Original) and Cache (\greedy) for the query types in Table~\ref{tab:algoresults}. This metric represents the ratio of tokens that can be served from the KV cache over all tokens in the input prompt. It indicates the effectiveness of the KV cache and is directly correlated with latency performance. Across queries, Cache (\greedy) provides between a 1.7-12.3$\times$ prefix hit rate improvement over Cache (Original). 



% \begin{figure}[t!]
%      \centering
%      \begin{subfigure}[b]{0.48\columnwidth}
%         \centering
%         \includegraphics[width=\textwidth]{figures/SIGMODfigures/movies_hr.pdf}
%         \caption{Movie Dataset}
%         \label{fig:cdf_size}
%     \end{subfigure}
%     \hfill
%     \begin{subfigure}[b]{0.48\columnwidth}
%         \centering
%         \includegraphics[width=\textwidth]{figures/SIGMODfigures/products_hr.pdf}
%         \caption{Product Dataset} 
%         \label{fig:cdf_freqs}
%     \end{subfigure}
%     \label{fig:cachehitrate}
%     \caption{Cache Hit Rate Ablation. We illustrate the cache hit rate improvements achieved by Cache (\greedy) compared to Cache (FIFO), showing up to a 46\% increase on both the Product dataset and Movie datasets.}
% \end{figure}
% \begin{figure}[t!]
%     \centering
%     \includegraphics[width=0.8\columnwidth]{figures/MLSys_Figures/phr.pdf}
%     \label{fig:cachehitrate}
%     \caption{Cache Hit Rate Ablation. We illustrate the cache hit rate improvements achieved by Cache (\greedy) compared to Cache (FIFO), showing up to a 75\% increase across datasets.}
% \end{figure}
% Fig~\ref{fig:modelablation} shows the latency results of our techniques on LLM filter queries with the Llama-3-70B-Instruct model on 8xL4(24GB), demonstrating a 1.9 - 3.3$\times$ speedup compared to Cache (Original).
\vspace{-0.5em}
\textbf{Results on Different Model Sizes} Fig~\ref{fig:modelablation} shows the evaluation of our Cache (\greedy) method compared with Cache (original) on filtering queries, using Llama-3-70B-Instruct with 70B parameters. We run this model on an 8$\times$L4 instance with tensor parallelism and measure the end-to-end query latency. Cache (\greedy) achieves 1.9$\times$ to 3.3$\times$ speed-up under this setup, showing a trend similar compared to the Llama-3-8B model. We evaluate the larger model accuracy on LLM Filter queries in Sec~\ref{sec:accuracy}. We also show results for the smaller 1B model in Appendix~\ref{appendix:models}.
% Although end to-end runtime is slower for this model than the 8B parameter model, Cache (\greedy) still achieves between 1.9$\times$ and 3.3$\times$ speed-up over Cache (Original).


 
% \asim{is there anything else we want to say about this ablation?}
\begin{figure}[t!]
    \centering
    \includegraphics[width=0.9\columnwidth]{figures/MLSys_Figures/dataset_runtimes_tensor_parallel.pdf}
    \vspace{-1em}
    \caption{Cache (GGR) is able to achieve 1.9 -- 3.3$\times$ speed-up over Cache (Original) for filter queries on Llama3-70B.}
    \label{fig:modelablation}
    \vspace{-2em}
\end{figure}

\renewcommand{\arraystretch}{1} % Adjust row height for better spacing



% \begin{table}[h]
%     \centering
%     \begin{tabularx}{\columnwidth}{lllcc}
%     \hline
%     \textbf{Dataset}                      & \textbf{Model}                                                                 & \textbf{Ordering} & \textbf{Mean}              & \textbf{90\% CI}             \\ \hline
%     \multirow{4}{*}{Movies}      & \multirow{2}{*}{\begin{tabular}[c]{@{}l@{}}Llama-3\\ 8B-Instruct\end{tabular}} & Original          & 88.0\%                     & 82-93\%                     \\
%                                           &                                                                                & GGR               & 88.0\%                     & 83-93\%                     \\
%                                           & \multirow{2}{*}{GPT-4o}                                                        & Original          & \multicolumn{1}{l}{89.0\%} & \multicolumn{1}{l}{84-94\%} \\
%                                           &                                                                                & GGR               & \multicolumn{1}{l}{92.0\%} & \multicolumn{1}{l}{87-96\%} \\ \hline
%     \multirow{4}{*}{Movies-Full} & \multirow{2}{*}{\begin{tabular}[c]{@{}l@{}}Llama-3\\ 8B-Instruct\end{tabular}} & Original          & 88.0\%                     & 83-93\%                     \\
%                                           &                                                                                & GGR               & 87.0\%                     & 81-92\%                     \\
%                                           & \multirow{2}{*}{GPT-4o}                                                        & Original          & \multicolumn{1}{l}{91.0\%} & \multicolumn{1}{l}{86-95\%} \\
%                                           &                                                                                & GGR               & \multicolumn{1}{l}{93.9\%} & \multicolumn{1}{l}{90-98\%} \\ \hline
%     \multirow{4}{*}{Fever}       & \multirow{2}{*}{\begin{tabular}[c]{@{}l@{}}Llama-3\\ 8B-Instruct\end{tabular}} & Original          & 44.2\%                     & 44-45\%                     \\
%                                           &                                                                                & GGR               & 56.5\%                     & 56-57\%                     \\
%                                           & \multirow{2}{*}{GPT-4o}                                                        & Original          & \multicolumn{1}{l}{73.5\%} & \multicolumn{1}{l}{73-74\%} \\
%                                           &                                                                                & GGR               & \multicolumn{1}{l}{70.7\%} & \multicolumn{1}{l}{70-71\%} \\ \hline
%     \end{tabularx}
%     \vspace{1mm}
%     \caption{The accuracy of the original column ordering v.s. the GGR ordering. For each order, the LLM output is compared against ground truth labels to determine the accuracy of the responses. For each dataset, we perform statistical bootstrapping \cite{bootstrapping} to get a distribution of accuracy measurements across 10,000 runs. Mean is the average across all the bootstrap runs, and 90\% CI represents a 90\% confidence interval for the accuracy.}
%     \label{tab:accuracyresults}

%     % \hspace{1cm}

% \end{table}
% \begin{table*}[ht]
% \centering
% \renewcommand{\arraystretch}{1.2}
% \setlength{\tabcolsep}{4pt} % Adjust column padding for compactness
% \begin{tabular*}{\textwidth}{@{\extracolsep{\fill}} l c cccccc cccccc cccccc}
% \hline
% \textbf{Model} & \textbf{Ordering} & \multicolumn{2}{c}{\textbf{Movies}} & \multicolumn{2}{c}{\textbf{Products}} & \multicolumn{2}{c}{\textbf{BIRD}} & \multicolumn{2}{c}{\textbf{PDMX}} & \multicolumn{2}{c}{\textbf{FEVER}} \\
%  & & \small \textbf{Mean} & \small \textbf{90\% CI} & \small \textbf{Mean} & \small \textbf{90\% CI} & \small \textbf{Mean} & \small \textbf{90\% CI} & \small \textbf{Mean} & \small \textbf{90\% CI} & \small \textbf{Mean} & \small \textbf{90\% CI} \\
% \hline

% Llama-3-8B-Instruct & Original & \small 94.0\% & \small 93-99\% & \small 95.9\% & \small 92-99\% & \small 67.0\% & \small 59-75\% & \small 81.2\% & \small 75-88\% & \small 44.2\% & \small -- \\
%  & GGR      & \small 97.0\% & \small 94-99\% & \small 95.0\% & \small 91-98\% & \small 67.0\% & \small 59-75\% & \small 74.0\% & \small 67-81\% & \small 56.5\% & \small -- \\
% \hline

% Llama-3-70B-Instruct & Original & \small 86.0\% & \small 80-92\% & \small 96.0\% & \small 93-99\% & \small 82.0\% & \small 76-88\% & \small 92.0\% & \small 87-96\% & \small -- & \small -- \\
%  & GGR      & \small 90.0\% & \small 85-95\% & \small 97.0\% & \small 94-99\% & \small 82.9\% & \small 77-89\% & \small 82.0\% & \small 75-88\% & \small -- & \small -- \\
% \hline

% GPT-4o & Original & \small 96.0\% & \small 93-99\% & \small 99.0\% & \small 97-100\% & \small 84.0\% & \small 78-90\% & \small 87.9\% & \small 82-93\% & \small -- & \small -- \\
%  & GGR      & \small 93.0\% & \small 89-97\% & \small 97.0\% & \small 94-99\% & \small 83.0\% & \small 77-89\% & \small 83.0\% & \small 77-89\% & \small -- & \small -- \\

% \hline
% \end{tabular*}
% \caption{The accuracy of the original column ordering v.s. the GGR ordering. For each order, the LLM output is compared against ground truth labels to determine the accuracy of the responses. For each dataset, we perform statistical bootstrapping \cite{bootstrapping} to get a distribution of accuracy measurements across 10,000 runs. Mean is the average across all the bootstrap runs, and 90\% CI represents a 90\% confidence interval for the accuracy.}
% \label{tab:accuracyresults}
% \end{table*}

% \begin{table*}[ht]
% \centering
% \renewcommand{\arraystretch}{1.2}
% \setlength{\tabcolsep}{2pt} % Reduce column padding for compactness
% \small
% \begin{tabular*}{\textwidth}{@{\extracolsep{\fill}} p{1.7cm}<{\centering} p{1.5cm}<{\centering} cccccc cccccc cccccc cccccc}
% \hline
% \multirow{2}*{\textbf{\small Model}} & \multirow{2}*{\textbf{\small Ordering}}  & \multicolumn{2}{c}{\textbf{\small Movies}} & \multicolumn{2}{c}{\textbf{\small Products}} & \multicolumn{2}{c}{\textbf{\small BIRD}} & \multicolumn{2}{c}{\textbf{\small PDMX}} & \multicolumn{2}{c}{\textbf{\small Beer}} & \multicolumn{2}{c}{\textbf{\small FEVER}} \\
%  & & \footnotesize Mean & \footnotesize 90\% CI & \footnotesize Mean & \footnotesize 90\% CI & \footnotesize Mean & \footnotesize 90\% CI & \footnotesize Mean & \footnotesize 90\% CI & \footnotesize Mean & \footnotesize 90\% CI & \footnotesize Mean & \footnotesize 90\% CI \\
% \hline

% \footnotesize{8B-Instruct} & \footnotesize{Original} & \small 94.0\% & \small 93-99\% & \small 95.9\% & \small 92-99\% & \small 67.0\% & \small 59-75\% & \small 81.2\% & \small 75-88\% & \small 86.9\% & \small 81-92\% & \small 44.2\% & \small -- \\
%  & \footnotesize{GGR}      & \small 97.0\% & \small 94-99\% & \small 95.0\% & \small 91-98\% & \small 67.0\% & \small 59-75\% & \small 74.0\% & \small 67-81\% & \small 81.1\% & \small 74-87\% & \small 56.5\% & \small -- \\
% \hline

% \footnotesize{70B-Instruct} & \footnotesize{Original} & \small 86.0\% & \small 80-92\% & \small 96.0\% & \small 93-99\% & \small 82.0\% & \small 76-88\% & \small 92.0\% & \small 87-96\% & \small 93.0\% & \small 89-97\% & \small -- & \small -- \\
%  & \footnotesize{GGR}      & \small 90.0\% & \small 85-95\% & \small 97.0\% & \small 94-99\% & \small 82.9\% & \small 77-89\% & \small 82.0\% & \small 75-88\% & \small 90.0\% & \small 85-95\% & \small -- & \small -- \\
% \hline

% \footnotesize{GPT-4o} & \footnotesize{Original} & \small 96.0\% & \small 93-99\% & \small 99.0\% & \small 97-100\% & \small 84.0\% & \small 78-90\% & \small 87.9\% & \small 82-96\% & \small 90.0\% & \small 85-95\% & \small -- & \small -- \\
%  & \footnotesize{GGR}      & \small 93.0\% & \small 89-97\% & \small 97.0\% & \small 94-99\% & \small 83.0\% & \small 77-89\% & \small 83.0\% & \small 77-89\% & \small 87.0\% & \small 81-92\% & \small -- & \small -- \\

% \hline
% \end{tabular*}

% \caption{The accuracy of the original column ordering v.s. the GGR ordering. For each order, the LLM output is compared against ground truth labels to determine the accuracy of the responses. We achieve \shu{@amog}}
% \label{tab:accuracy_results}
% \end{table*}

\begin{figure*}[tbp]
     \centering
     \begin{subfigure}[b]{0.33\textwidth}
        \centering
        % \includegraphics[width=\textwidth]{figures/movies_runtimes_e2e.pdf}
        \includegraphics[width=\textwidth]{figures/MLSys_Figures/8b.pdf}
        \caption{Meta-Llama-3-8B-Instruct}
        \label{fig:movies-runtimes}
    \end{subfigure}
    \hfill
    \begin{subfigure}[b]{0.33\textwidth}
        \centering
        % \includegraphics[width=\textwidth]{figures/products_runtimes_e2e.pdf}
        \includegraphics[width=\textwidth]{figures/MLSys_Figures/70b.pdf}
        \caption{Meta-Llama-3-70B-Instruct}
        \label{fig:products-runtimes}
    \end{subfigure}
    \begin{subfigure}[b]{0.33\textwidth}
        \centering
        % \includegraphics[width=\textwidth]{figures/products_runtimes_e2e.pdf}
        \includegraphics[width=\textwidth]{figures/MLSys_Figures/gpt.pdf}
        \caption{OpenAI GPT-4o}
        \label{fig:products-runtimes}
    \end{subfigure}

    \vspace{-1em}
    \caption{Accuracy of original v.s. \greedy ordering: we perform statistical bootstrapping to get a distribution of exact match accuracy measurements across 10,000 runs. The numbers indicate the difference in the median accuracy of \greedy compared to the original ordering.}
    % \SHU: put this into foot note: Note that since FEVER has 22,665 labeled rows, the bootstrapping results have much less variance compared to the other datasets.
    % For each dataset, we perform statistical bootstrapping \cite{bootstrapping} to get a distribution of exact match accuracy measurements across 10,000 runs. 
    % \caption{Box plot of exact match accuracy of the original column ordering v.s. \greedy ordering. For each dataset, we perform statistical bootstrapping \cite{bootstrapping} to get a distribution of exact match accuracy measurements across 10,000 runs. The numbers indicate the difference in median accuracy of \greedy ordering compared to the original ordering. Note that since FEVER has 22,665 labeled rows, the bootstrapping results have much less variance compared to the other datasets. \shu{make this font bigger}}
    \label{fig:accuracy}
    \vspace{-0.5em}
\end{figure*}















% \begin{table}
%     \centering
%     \begin{tabularx}{\columnwidth}{X X X X r}
%         \hline
%         \textbf{Dataset} & \textbf{Rows} & \textbf{Columns} & \textbf{Mean Solver Runtime (s)} \\ \hline
%         {Movies}   & 15019   & 7                & 20.7 \\ \hline
%         {Products} & 15058         & 8                & 26.2  \\ \hline
%         {SQuAD}    & 19928         & 4                & 33.1   \\ \hline
%         {FEVER}    & 22681         & 4                & 90.4          \\ \hline
%     \end{tabularx}
%     \vspace{1mm}
%     \caption{\greedy algorithm solver times across datasets. \greedy takes the longest on FEVER, followed by SQuAD, Products, and Movies, which have similar solver times.}   %\shu{re-explain the caption: rows, columns, etc.}} % We collect results with early\_stop as group size 1000 for FEVER and size 2 for the others.
%     \label{tab:solvertimes}
% \end{table}
\vspace{-0.5em}

\subsection{Cost Savings on Proprietary API Endpoints}
% 
This section evaluates the cost efficiency of our \greedy algorithm with closed models that support prompt caching. 
For OpenAI, cached prompts are offered at a 50\% discount compared to uncached prompts.  
Anthropic beta prompt caching~\cite{anthropicpromptcaching} requires users to manually specify prompts to cache. Writing to the cache costs 25\% more than the base input token price for any given model while using cached content costs only 10\% of the base rate. We evaluate OpenAI GPT-4o-mini and Anthropic Claude 3.5 Sonnet, using their pricing models in our cost calculations.\footnote{GPT-4o-mini charges \$0.075/1M tokens for cached tokens versus \$0.15/1M tokens for uncached tokens.}\footnote{Claude 3.5 Sonnet standard input tokens are priced at \$3 per million tokens, cache writes at \$3.75 per million, and cache reads at \$0.30 per million tokens.}

% GPT-4o-mini charging \$0.075/1M cached tokens versus \$0.15/1M uncached tokens.
% Claude 3.5 Sonnet standard input tokens priced at \$3/1M tokens, cache writes at \$3.75/1M tokens, and reads from the cache at \$0.30/1M tokens. 

% We assess the impact of our Cache (GGR) algorithm when employing closed models that support caching. We examine caching policies implemented by OpenAI and Anthropic. OpenAI’s caching policy stores the longest prefix of a previously computed prompt, beginning at 1,024 tokens and increasing in increments of 128 tokens \cite{openaipromptcaching}. GPT-4o-mini applies a reduced rate for cached tokens at \$0.075 per million tokens, while uncached tokens are priced at \$0.15 per million tokens. Anthropic’s beta caching policy requires users to specify prompts to write to cache manually. Standard non-cache related input tokens are \$3/1MTok, while writing to the cache with Anthropic is \$3.75/1MTok and reading from cache is \$0.30/1MTok. \cite{anthropicpromptcaching}. % \shu{remove TPS results, just keep costs. } \asim{how did we get these numbers? need to cite}. . %This caching approach can reduce latency by up to 80\% and costs by 50\%, accounting for the additional charge on cached tokens \cite{}
% This naive method reflects the behavior of current systems, which neither reuse nor reorder rows or columns. 

% We compare reordering with \greedy against the original table order. 

\begin{table}[h!]
\centering
\small
\renewcommand{\arraystretch}{1.2} % Increase row height (default is 1.0)

\begin{tabular}{c c c c c c}
\hline
\textbf{Dataset} & \textbf{Model} & \textbf{Method} & \textbf{PHR (\%)} & \textbf{Cost (\$)} & \textbf{Savings (\%)} \\
\hline
\multirow{4}{*}{FEVER} 
& \multirow{2}{*}{4o-mini} & Original & 0.0   & 0.81  & -     \\
&                              & GGR      & 62.2  & 0.55 & 32\%  \\
\cline{2-6}
& \multirow{2}{*}{Sonnet} & Original & 0.0   & 5.49  & -     \\
&                                & GGR      & 30.6  & 4.33  & 21\%  \\
% \hline
% \multirow{4}{*}{SQuAD} 
% & \multirow{2}{*}{4o-mini} & Original & 0 & 0 & 0 \\
% &                              & GGR      & 0 & 0 & 0 \\
% \cline{2-6}
% & \multirow{2}{*}{Sonnet} & Original & 0 & 0 & 0 \\
% &                                & GGR      & 0 & 0 & 0 \\
\hline
\end{tabular}
\caption{OpenAI and Anthropic Costs: cache hit rate (HR\%), cost, and savings comparison of GGR over Original for GPT-4o-mini and Claude 3.5 Sonnet in FEVER.}
\label{tab:cost_comparison}

\end{table}

\begin{table}[t!]
\centering
\small
\begin{tabularx}{\columnwidth}{l@{\hskip 20pt}c@{\hskip 20pt}c@{\hskip 8pt}c@{\hskip 20pt}c@{\hskip 8pt}c}
\toprule
\multirow{2}{*}{\textbf{Dataset}} & \multicolumn{2}{c}{\small \textbf{PHR (\%)}} & \multicolumn{2}{c}{\small \textbf{Est. Cost Savings (\%)}} \\
\cmidrule(lr){2-3} \cmidrule(lr){4-5}
                  & \small Original & \small GGR & \small OpenAI & \small Anthropic \\ 
\midrule
\textbf{Movies}  & 34.6          & 85.7         & 31            & 73              \\ 
\textbf{Products}& 26.7          & 83.3         & 33            & 73              \\ 
\textbf{BIRD}    & 10.4          & 84.8         & 39            & 79              \\ 
\textbf{PDMX}    & 11.8          & 56.6         & 24            & 48              \\ 
\textbf{Beer}    & 49.9          & 80.1         & 20            & 55              \\ 
\textbf{FEVER}   & 11.2          & 67.4         & 30            & 60              \\ 
\textbf{SQuAD}   & 11.0          & 69.7         & 31            & 63              \\ 
\bottomrule
\end{tabularx}

\vspace{-0.5em}
\caption{Estimated cost savings: across datasets using PHR from Sec~\ref{sec:end-to-end} and OpenAI and Anthropic's pricing model. }
\label{tab:algoresults}
\vspace{-1.75em}
\end{table}
% Our estimates are based on the assumption that automatic prefix caching is enabled for prefixes of all sizes.
Since both OpenAI and Anthropic require a minimum prefix length of 1,024 tokens for caching, we duplicate each field value five times, approximating a more realistic dataset with detailed conversations and descriptions.
We select the FEVER dataset for its long input length and use 1000 rows from this dataset. 
For Anthropic experiments, we specify cache write for only the first 1,024 tokens per request as a conservative assumption, as Anthropic does not support automatic prefix detection. 

We evaluate \greedy reordering on two tables submitted to the OpenAI and Anthropic APIs (each row is a request): one reordered with \greedy and one in the original row and field order. 
Table~\ref{tab:cost_comparison} shows that \greedy achieves 32\% cost savings with GPT-4o-mini and 21\% savings with Claude 3.5 Sonnet. 
The hit rate in OpenAI for \greedy-reordered table is 62.2\%, closely matching the hit rate (i.e., 67\%) measured from our previous experiment in Table~\ref{tab:hit-rate}. 
Original ordering receives no cached tokens with 0\% cache hits, as the shared prefix does not meet the 1,024-token minimum.
The Anthropic cache hit rate is around 30.6\%, two times lower than the OpenAI hit rate due to our conservative caching threshold. 

Assume that in the future, automatic prefix caching is enabled and prompts can be cached at arbitrary token lengths. We use the hit rate numbers collected from our previous experiments in Table~\ref{tab:hit-rate} to simulate cost-saving ratios achievable by GGR, compared to the original unordered algorithm. \greedy yields 20 to 39\% cost savings under the OpenAI pricing model and up to 79\% cost savings with Anthropic.

% Notice that original orderings get 0 cached tokens because the prefix matched length is less than 1024. 
% Anthropic cost savings are 21\%, which is because we conservatively cache just the first 1024 tokens of each row, as automatic prefix detection and caching are not enabled in the API. This is a pessimistic assumption. 

% We select the FEVER dataset due to its longest average input length. We select 1000 rows of this dataset containing the most frequently appearing evidence fields for our experiments. For Anthropic experiments, we conservatively cache just the first 1024 tokens of each row, as automatic prefix detection and caching are not enabled in the API. This is a pessimistic assumption. 



% With GPT-4o-mini, \greedy achieves x\%-15\% cost savings. Using Anthropic's pricing model, our Cache (\greedy) optimization achieves x\%-21\% cost savings. FEVER has significantly more cost benefits than SQuAD due to the %single token output constraints for FEVER. \joey{here is another way to say the next line.}
% to the query asking for a 1 word answer.
% As both vendors price output tokens at a premium and caching applies only to input tokens, a higher ratio of input to output tokens increases cost savings. Our cost and hit rate improvements on Anthropic are notable despite only caching the first 1024 tokens, and represent a lower bound on achievable cost savings under the mentioned model.

% \joey{Can we add back the discussion on projected savings if the 1024 token limit were dropped?  Would we argue for vendors to offer this?}


% \amog{update the column headers in the table to be explicit about simulated pricing.}
% \amog{Clarify that this is using the same pricing *model* as OpenAI and Anthropic, but these are not the actual pricing results for these specific datasets as there is a minimum cache token length that's required. For datasets with longer text fields this would generalize.} 
%Table~\ref{tab:performance_comparison_openai} and \ref{tab:performance_comparison_anthropic} show the calculated cost savings for each dataset including estimated input token cost savings for Cache (Original) and Cache (\greedy) using OpenAI's and Anthropic's cost model for prompt caching. We use our calculated $PHR$ as the percentage of cached input tokens and $1 - PHR$ as the percentage of uncached input tokens, and compute cost using these values according to each vendor's pricing model for cached vs uncached input tokens. OpenAI provides a 50\% discount for cached input tokens with GPT-4o cached input tokens priced at \$1.25/1MTok as opposed to \$2.50/1MTok for uncached tokens. The same pricing difference applies to GPT-4o-mini, with \$0.075/1MTok and \$0.15/1MTok for cached and uncached input tokens respectively. Using OpenAI's pricing model, our Cache (\greedy) optimization achieves 20\%--39\% cost savings across datasets.  

%Anthropic's pricing model is more nuanced, with \$3/1MTok for standard input tokens, \$3.75/1MTok for cache writes, and \$0.30/1MTok for cache reads. Thus, after a single cache write, future cached tokens can be 90\% cheaper than uncached input tokens. Using Anthropic's pricing model, our Cache (\greedy) optimization achieves 48\%--79\% cost savings across datasets. 
%\shu{write in paragraph, no bullet points. for any paragraphs like this with one or two sentences more than a line just rephrase and make it shorter} 
% As \greedy alters the prompt field order, we evaluate its impact on LLM query accuracy using LLM Filter queries (see Sec~\ref{subsec:llmqueries}) across datasets and an RAG query of FEVER. 

\vspace{-0.5em}
\subsection{Impact of Reordering on Accuracy} \label{sec:accuracy}
As \greedy order alters the input prompt to the LLM, we assess the impact this has on query accuracy using LLM Filter queries (Sec~\ref{subsec:llmqueries}) with constrained output. We also evaluate a RAG query of FEVER, excluding SQuAD due to its open-ended questions. 
FEVER includes ground-truth labels for all records, while 100 rows from other datasets are manually labeled. Using statistical boostrapping~\cite{bootstrapping}, we perform 10K runs, sampling with replacement on each run to obtain a  distribution of accuracy results. Accuracy experiments are conducted with Llama-3-8B-Instruct, Llama-3-70B-Instruct, and GPT-4o models, measured as the percentage of exact matches between the LLM output and the ground truth labels. 

% while for other datasets, we randomly sample 100 rows and manually label them with ground truth answers.  For all datasets, we employ statistical bootstrapping \cite{bootstrapping}, performing 10,000 bootstrap runs where, on each bootstrap, we sample a new dataset from the original with replacement and calculate the accuracy on this sample. This gives us a distribution of accuracy across all 10,000 runs. We run our accuracy experiments using the Llama-3-8B-Instruct, Llama-3-70B-Instruct, and GPT-4o models, measuring accuracy as the percentage of exact matches between the LLM output and the ground truth labels. 

% Our evaluation includes LLM filter queries from Sec~\ref{subsec:llmqueries} for all datasets and a RAG query with the FEVER dataset. We omit SQuAD for this experiment since it contains open-ended questions. FEVER already contains ground truth labels for all 22,665 records, but for the other datasets, we randomly sample 100 rows and manually label them with the ground truth answers. 

% In the FEVER dataset, given a claim and four pieces of evidence, the LLM is asked to determine if the claim is factually correct, outputting SUPPORTS, REFUTES, or NOT ENOUGH INFO, for which the dataset already contains ground truth labels for all 22,682 rows.
% We select the Movie and FEVER datasets for our experiment.The LLM outputs either "Yes" or "No" on whether a movie is suitable for kids given movie\_info and review\_description fields. We run the experiment with two fields (Movies) and seven fields (Movies-Full) For the Product dataset, we run Q5 from Section~\ref{llmqueries} on the Fever Dataset.



In Fig~\ref{fig:accuracy}, we plot the accuracy distributions across the bootstrap runs and the relative difference in median accuracy of \greedy versus original ordering. The accuracy distribution of \greedy ordering is within 5\% accuracy of the original ordering, with the only exception being FEVER with Llama-3-8B, where the ordering with \greedy performs 14.2\% \emph{better} than the original. This is due to the \greedy algorithm places the ``claim'' field at the end of the prompt instead of at the beginning, which Llama3-8B prefers. However, the same behavior does not hold for the larger models. Overall, we can see that larger models like Llama-3-70B and GPT-4o are within 5\% of accuracy difference compared with original ordering and are more robust to field reordering.

\vspace{-0.2em}
\subsection{Algorithm Overhead} %\asim{this title is a bit confusing/vague}
% \begin{table}[t!]
% \footnotesize
% \setlength{\tabcolsep}{6pt} % Adjust column separation
% \begin{tabularx}{\columnwidth}{cc}
% \toprule
% \textbf{Solver Time (s)}  \\
% \textbf{Solver time (s)}  3.3  & 4.5  & 1.2 & 12.6  & 8.0           & 5.6          & 4.5 \\
% \bottomrule
% \end{tabularx}

% \vspace{-0.5em}
% \caption{PHR (\%) of LLM Filter and RAG queries for Original and GGR. GGR achieves 30 - 75\% higher hit rates than the original ordering. }
% \label{tab:algosolvertimes}
% \end{table}



% \subsubsection{Algorithm Overheads}
\begin{table}[t!]
\vspace{1em}
\centering
\footnotesize
\renewcommand{\arraystretch}{1.1} % Increase row height (default is 1.0)

\begin{tabular}{c}
\hline
\multicolumn{1}{c}{\textbf{Solver Time (s)}} \\
\begin{tabular}{ccccccc}
Movies & Products & BIRD & PDMX & Beer & FEVER & SQuAD \\
\hline
3.3 & 4.5 & 1.2 & 12.6 & 8.0 & 5.6 & 4.5 \\  % Replace with actual times
\hline
\end{tabular} \\
\hline
\end{tabular}
\vspace{-0.1em}
\caption{\greedy Solver time (s): \greedy runs under 15 seconds for datasets with up to 30K rows and 57 fields.}
\label{tab:algosolvertimes}
\vspace{-2em}
\end{table}
\textbf{Latency} Table~\ref{tab:algosolvertimes} shows the average overheads of \greedy across datasets, using a row recursion depth of four and column recursion depth of two, or an early stopping threshold of 0.1M hit count. In all cases, \greedy runs in under 15 seconds -- less than 0.01\% of LLM query runtimes. \newline 
\textbf{Memory} \greedy only requires the input table $T$ ($n$ rows, $m$ columns) touched by the query to be loaded into memory. Recursive splitting reduces table size at each step, keeping total memory usage at $O(n \times m)$, aside from minimal stack and temporary variable overhead.
% Table~\ref{tab:algosolvertimes} shows the \greedy algorithm overheads averaging across queries for each dataset. \greedy is run on each dataset with termination thresholds of four columns and two rows (recursion stops once a specified depth is exceeded row-wise or column-wise). We also have an early stopping threshold of 100000 for the hit-count score and stop recursion if the max hit-count score across values is less than this score, falling back to an ordering approximated by table statistics. Our solver overhead is minimal, running in under 15 seconds for all our experiment datasets. This is less than 0.01\% of the actual runtime of the LLM queries.

% The longest solver time is for PDMX with 57 columns due to the linear scanning in each recursive step used to calculate value counts, which increases in complexity with the number of columns. The percentage of the end-to-end runtime attributed to the solver is under 1\% for every dataset.





% \begin{figure*}[tbp]
%      \centering
%      \begin{subfigure}[b]{0.48\textwidth}
%         \centering
%         \includegraphics[width=\textwidth]{figures/cache_hit_rate_movies_pdf.png}
%         \caption{Rotten Tomatoes Movies Dataset}
%         \label{fig:movies-runtimes}
%     \end{subfigure}
%     \hfill
%     \begin{subfigure}[b]{0.48\textwidth}
%         \centering
%         \includegraphics[width=\textwidth]{figures/cache_hit_rate_products_pdf.png}
%         \caption{Amazon Products Dataset}
%         \label{fig:products-runtimes}
%     \end{subfigure}

%     %\vspace{-2em}
%     \caption{Cache Hit Rate}
%     \label{fig:runtimes}
% \end{figure*}


% \begin{figure*}[tbp]
%      \centering
%      \begin{subfigure}[b]{0.33\textwidth}
%         \centering
%         \includegraphics[width=\textwidth]{figures/movies_runtimes.pdf}
%         \caption{Rotten Tomatoes Movies Dataset}
%         \label{fig:movies-runtimes}
%     \end{subfigure}
%     \hfill
%     \begin{subfigure}[b]{0.33\textwidth}
%         \centering
%         \includegraphics[width=\textwidth]{figures/products_runtimes.pdf}
%         \caption{Amazon Products Dataset}
%         \label{fig:products-runtimes}
%     \end{subfigure}
%     \hfill
%     \begin{subfigure}[b]{0.33\textwidth}
%         \centering
%         \includegraphics[width=\textwidth]{figures/squad_5000_50_cache2.pdf}
%         \caption{SQuAD Dataset}
%         \label{fig:squad}
%     \end{subfigure}
%     %\vspace{-2em}
%     \caption{Row Orderings Ablation: (a) (b) Movie and Product Datasets - our approach achieves 3 - 4.3$\times$ speedup on Movie, and 1.7 - 2$\times$ speedup on Product (c) our approach achieves 1.7 - 1.9 $\times$ speedup on SQuAD}
%     \label{fig:runtimes}
% \end{figure*}



% \begin{figure*}[tbp]
%      \begin{subfigure}[b]{0.48\linewidth}
%         \centering
%         \includegraphics[width=\textwidth]{figures/movies_col_runtimes_vllm.pdf}
%         \caption{Movie}
%         \label{fig:col_order_product}
%     \end{subfigure}
%     \hfill
%     \begin{subfigure}[b]{0.48\linewidth}
%         \centering
%         \includegraphics[width=\textwidth]{figures/products_col_runtimes_vllm.pdf}
%         \caption{Product}
%         \label{fig:col_order_product}
%     \end{subfigure}
%     \label{fig:variability}
%     \caption{Column Orderings Ablation: our approach achieves up to 2$\times$ on Movie, and 1.5$\times$ speedup on Product \shu{replot this, two bar for each dataset}}
% \end{figure*}

% \begin{figure*}[tbp]
%      \centering
%      \begin{subfigure}[b]{0.24\linewidth}
%         \centering
%         \includegraphics[width=\textwidth]{figures/movies_col_runtimes.pdf}
%         \caption{Movie: Description, Title, Review}
%         \label{fig:col_order_movie}
%     \end{subfigure}
%     \hfill
%     \begin{subfigure}[b]{0.24\linewidth}
%         \centering
%         \includegraphics[width=\textwidth]{figures/movies_col_runtimes_new.pdf}
%         \caption{Movie: Description, Type, Review}
%         \label{fig:col_order_product}
%     \end{subfigure}
%     \hfill
%     \begin{subfigure}[b]{0.24\linewidth}
%         \centering
%         \includegraphics[width=\textwidth]{figures/products_col_runtimes.pdf}
%         \caption{Product: Description, Title, Review}
%         \label{fig:col_order_product}
%     \end{subfigure}
%     \hfill
%     \begin{subfigure}[b]{0.24\linewidth}
%         \centering
%         \includegraphics[width=\textwidth]{figures/products_col_runtimes_new.pdf}
%         \caption{Product: Description, Format, Review}
%         \label{fig:col_order_product}
%     \end{subfigure}
%     \label{fig:variability}
%     \caption{Column Orderings (SGLang, Cache More Columns): }
% \end{figure*}

% \begin{figure*}[t]
% \centering
% \includegraphics[width=0.5\textwidth]{figures/movies_runtimes.pdf}
% \label{fig:movies-runtimes}
% \caption{End-to-end runtimes for movies queries.} 
% \end{figure*}
% \begin{figure}[tbp]
%     \begin{subfigure}[b]{.5\linewidth}
%         \centering
%         \includegraphics[width=\linewidth]{figures/selectivity.pdf}
%         \label{fig:select-movie}
%     \end{subfigure}
%     \begin{subfigure}[b]{.5\linewidth}
%         \centering
%         \includegraphics[width=\linewidth]{figures/selectivity.pdf}
%         \label{fig:select-movie}
%     \end{subfigure}
%     \caption{Selectivity: Movie and Product Datasets.}
%     \label{fig:selectivity}
% \end{figure}







% \subsubsection{Llama3-70B results} \label{sec:modelablation}
% Fig~\ref{fig:modelablation} shows evaluation of our methods using a larger model Llama-3-70B-Instruct. We run this model using a tensor parallel configuration for vLLM across 8 L4 instances. Although end-to-end runtime is slower for this model than the 8B parameter model, Cache (\greedy) still achieves between 1.9$\times$ and 3.3$\times$ speed-up over Cache (Original). \asim{is there anything else we want to say about this ablation?}

% \subsubsection{Impact of Algorithm Parameter}

% \subsubsection{Impact of KV Cache Size}
% \begin{table}[H]
% \centering
% \small
% \begin{tabularx}{\columnwidth}{l@{\hskip 4pt}c@{\hskip 4pt}c@{\hskip 4pt}c@{\hskip 4pt}c}
% \toprule
% \textbf{Algorithm} & \textbf{Latency (s)} & \textbf{TPS} & \textbf{Cache Hit Rate (\%)} & \textbf{Costs (\$)} \\
% \midrule
% GGR   & 0.77 & 11268 & 70.6 & 0.065 \\
% Naive & 0.94 & 9198 & 0.0 & 0.10 \\
% \bottomrule
% \end{tabularx}
% \caption{OpenAI GPT-4o mini performance comparison of GGR and naive method in terms of Average Latency, TPS (Tokens per Second), Cache Hit Rate, and Cost. The experiment utilized the FEVER dataset and selected the 100 longest prompts. GPT-4o-mini applies a reduced rate for cached tokens at \$0.075 per million tokens, while uncached tokens are priced at \$0.15 per million tokens.}
% \label{tab:performance_comparison_openai}
% \end{table}

% \begin{table}[H]
% \centering
% \small
% \begin{tabular}{cccccc}
% \hline
% \textbf{Model} & \textbf{Method} & \textbf{HR (\%)} & \textbf{Cost (\$)} & \textbf{Savings (\%)} \\
% \hline
% \multirow{2}{*}{GPT-4o-mini} 
% & Original     & 0.0          & 0.10      & -\\
% & GGR       & 70.6         & 0.065     & 35\%\\
% \hline
% \multirow{2}{*}{Claude Sonnet} 
% & Original     & 0.0 & 5.49 & -\\
% & GGR       & 30.6 & 4.33 & 21\%\\
% \hline
% \end{tabular}
% \caption{Performance comparison of GGR and original method regarding cache hit rate (HR\%) and Cost for GPT-4o-mini and Claude-3.5-Sonnet on the FEVER dataset. We achieve 21\% cost savings on Anthropic and x\% cost savings on OpenAI using \greedy.}
% \label{tab:performance_comparison}
% \vspace{-1em}
% \end{table}
% \section{Discussion} 
% \accheng{discussion or limitations?}
% \shu{@Audrey: removing subsections, and just have short sections on paragraph}
% \shu{Some additional discussions we want to have: different prompt length, different GPU types, different models (reference the VLDB reviewer comments)}
% Our techniques for optimizing LLM queries have been pragmatically motivated: we consider straightforward optimizations that apply broadly to most LLM inference systems and relational databases. 
We consider straightforward optimizations that apply broadly to most LLM inference systems and relational databases. In this section, we outline future research directions to enhance LLM inference in relational analytics further.
% that can extend the current capabilities of optimizing SQL queries involving LLM invocations. 
% Our discussion is organized into three subsections: distributed cache and request routing, cost estimation for LLM invocations, and semantic de-duplication. These directions are critical to improving the scalability of database systems with LLM invocations. 

% \accheng{not sure about this term} \simon{maybe the reverse: ... scalability of database systems with LLM invocations}, ensuring that the benefits of LLM integration can be realized more widely.
% achieved across multiple computational environments. 

% \joey{We don't leverage this decode length knowledge but it could be a big opportunity for a future version of this paper. I wonder if we want to mention that in the future work. }

% \subsection{Column Reordering and Row Sorting} 
% \label{sec:reordering-future}
% Our column reordering and row sorting algorithms leverage straightforward, practical heuristics to efficiently identify shared prefixes. We can further extend these algorithms in future work by adopting a finer-grained approach. For instance, instead of maintaining a fixed order of columns for all rows in a table, we can consider different column orderings for subsets of rows to potentially enable higher token hit rates.

% To do so, one potential strategy would be to iteratively determine column orders for each group of rows that share a value in a designated ``pivot'' column. Once an order is established for a group, we can exclude these rows from subsequent iterations, thereby reducing the complexity and focusing on determining the orders of the remaining rows. This recursive approach would allow us to progressively refine the column order to enhance the cache hit rate without giving up computational efficiency. Moreover, we plan to introduce a pruning mechanism for columns that consistently show negligible impact for prefix sharing. By excluding these columns from the reordering process, we can further reduce the computational overhead of this technique. %and potentially further optimize the overall performance of the system. 

% before show potential for optimizing cache efficiency through intelligent data arrangement. 
% However, the current approach, which maintains a fixed order of columns for all data rows, may not work well for all datasets. In our future work, we aim to refine the algorithm by adopting a more fine-grained approach that considers unique value-based column ordering for each data segment.  

% One potential strategy involves an iterative process where, for each group of rows sharing a unique value in a designated "pivot" column, we determine an optimal column ordering. Once an order is established for a group, we exclude these rows from subsequent iterations, thereby reducing the complexity and focusing on remaining data subsets. This recursive approach allows for progressively refining the data structure to maximize cache hit rates and computational efficiency. Moreover, we plan to introduce a pruning mechanism for columns that consistently show negligible impact on cache efficiency. By excluding these columns from the reordering process, we can reduce computational overhead and potentially further optimize the overall performance of the system. 

% \subsection{Prefix Caching: Opportunities}
% \subsubsection{Decode Length Knowledge} 
% While our work focuses on memory management for input prompts, output decoding also has a significant impact on performance. In particular, there are potential performance benefits with guided decoding in the analytics setting. Guided decoding, also known as structured generation or controlled generation, is a popular method to deterministically enforce a schema on the output of LLM \cite{guidance, sglang, outlines, lmql, aici}. Since guided decoding involves computing a large intermediate state, it does not typically employ batching, hurting GPU utilization. In the relational analytics scenario, we can utilize our knowledge of the output schema and decoding length to construct the output constraint ahead of time. In many cases, we can skip tokens that are known a priori and perform batching to improve utilization.
% \simon{Shu I think this paragraph is broken}
% \shu{We can skip this}


% on the optimizing the input process of the data in LLM, decoding the output is equally important factor to consider. There are many opportunities in applying guided decoding in relational analytics settings. Guided decoding, also termed structured generation or controlled generation, is a popular method to deterministically enforce a schema on the output of LLM \cite{guidance, sglang, outlines, lmql, aici}. The typical method of guided decoding involves computing a large finite state machine over the token space to constrain the tokens to be allowed for sampling. This process typically does not batch, hurting GPU utilization.



% \shu{@simon add the constrained decoding and some opportunities with batching here}
% \asim{can maybe add a small section here to discuss "very large" tables leading to long prompts, and say it is unclear how this impacts our algorithm performance and overall runtime. however, we can also clarify that for such a table it would be standard practice to only pass in a few columns to the LLM}
\paragraph{Batch-Aware Reordering.} \label{sec:smarter_reordering}
While column reordering and row sorting can improve cache hit rates, they don't account for how requests in a batch are processed in parallel during LLM inference. Some systems \cite{tgi} do not support prefix sharing within the same batch. For example, processing requests in the sequence $[1,1,1][2,2,2][3,3,3]$ leads to six cache misses, as each prefix ($1,2,3$) is recomputed in the same batch. However, reordering to $[1,2,3][1,2,3][1,2,3]$ reduces cache misses to three. To address this, we propose a batch-aware reordering strategy. This strategy should include a warm-up phase to pre-load the cache with diverse prefixes expected to be reused in subsequent batches and should be aware of inference batch size.% The algorithm sorts prefixes by frequency and arranges them across batches to ensure each batch contains distinct prefixes, thereby reducing cache misses significantly.
% ensure maximum cache utilization due to how 
% \begin{itemize}
%     \item Since inference occurs simultaneously for all requests in a batch, prefixes cannot be shared within the same batch.
%     \item Consider a sequence of requests $[1,1,1][2,2,2][3,3,3]$. None of these prefixes benefit from being cached because each distinct prefix ($1,2,3$) is computed multiple times with the same batch. As a result, there are six cache misses on this workload.
%     \item However, if we instead reorder these requests according to the sequence $[1,2,3][1,2,3][1,2,3]$, we observe benefits from using the prefix cache---$1, 2, 3$ are computed and cached after the first batch, allowing them to be reused in subsequent batches. Thus, there are only three cache misses on this series of requests.
% \end{itemize}

% The constraints of parallel batch computation necessitate a \textit{warm-up phase} that loads the cache with prefixes that will reused in subsequent batches. To further increase cache hit rate, we ...\accheng{shu, can you summarize the high-level idea of the alg in 1-2 sentences?}
% % e develop an algorithm to further increase cache hit rate while also ensuring we can run the largest batch size possible to fully utilize the GPU compute and memory capacity.

% \begin{itemize}
%     \item First, we sort all prefixes based on descending frequency. 
%     \item Then, we fill batches with distinct prefixes based on the sorted order.
%     \item Once all requests involving a given prefix have completed, we replace this prefix as we serve further requests (with different prefixes) following our sorted order.

%     \item For the example above with request prefixes $\{1,1,1,2,2,2,3,3,3\}$, our algorithm would construct batches $[1,2,3][1,2,3][1,2,3]$, which maximizes cache hit rate.
%     \item More complex example: reference Table \ref{tab:cache-state-comparison-warmup}.
    
% \end{itemize}

% \begin{table}[!t]
%     \centering
%     \begin{tabular}{c|c|c|c|c}
%         \multicolumn{1}{c|}{T} & \multicolumn{2}{c|}{Naive Reordering} & \multicolumn{2}{c}{Better Reordering} \\ \hline
%         & Keys accessed & Cache state & Keys accessed & Cache state \\ \hline
%         1 & ${\color{red}1, 1, 1}$ & - & ${\color{red}1, 2, 3}$ & - \\ \hline
%         2 & - & $1$ & - & $1, 2, 3$ \\ \hline
%         3 & ${\color{red}2, 2, 3}$ & $1, 2, 3$ & $1, 2, 3$ & $1, 2, 3$ \\ \hline
%         4 & - & $1, 2, 3$ & - & $1, 2, 3$ \\ \hline
%         5 & $3, 3, 3$ & $1, 2, 3$ & $1, 3, 3$ & $1, 2, 3$ \\ \hline
%     \end{tabular}
%     \vspace{0.5em}
%     \caption{Comparison of cache state of different input orderings for Section 3.4: three requests of 1, two requests of 2, four requests of 3. Better ordering can reduce the cache misses by half.} 
%     \label{tab:cache-state-comparison-warmup}
% \end{table}

% \shu{Future work; put it before related work; bsection about each of the possible future extension: http://www.bailis.org/papers/bolton-sigmod2013.pdf}
% \accheng{have we discussed these with a prof yet?}
% \simon{I think it is a very natural to claim this}
% \accheng{as in, is there something specific about batch analytics + the distributed setting we want to mention? the text currently seems to apply to LLM inference systems in general}
% \accheng{go back to calculations for 100GB so want to distribute but this is hard
% reiterate that we're opening a new area}
\paragraph{Distributed Cache and Request Routing.}
On large-scale datasets, we need to distribute the inference workload over many machines to enable queries to be completed in a reasonable amount of time. However, distributed KV caching is challenging; we need to ensure that requests with similar or identical prefixes are directed to the same machine. As such, balancing load over different machines with varying cache contents will be a critical objective. 

%RAG queries bring additional complexities since the KV cache must store not only LLM prefixes but also embeddings and associated retrieved contents, which may be distributed across nodes.

% Our current system has demonstrated potential for improving performance through request formulation and prefix caching for LLM on a single machine, tailored specifically for relational analytics tasks. However, deploying such systems in real-world scenarios often involves distributed computing environments where queries are processed across multiple nodes \cite{spotserve}. 

% In a distributed environment, maintaining cache locality is important. We want to ensure that requests with similar or identical prefixes are directed to the node where those prefix KVs are cached. The challenge compounds when dealing with RAG queries. Here, the cache not only contains the LLM prefixes but also can be the embeddings and associated retrieved contents, where may be distributed across nodes. Optimizing cache locality in this context both means LLM internal KV cache locality and also the data locality associated with query processing. 

% In addition, we want to ensure that no single node becomes a bottleneck. Our request routing should distribute both the computational load and data storage in a manner that prevents any single node from being overwhelmed. This requires real-time monitoring and adaptive request distribution strategies that can react to the cache access patterns and specific query demands. 


% Future work should focus on the development of distributed LLM cache management strategies that optimize these factors. This could involve techniques for intelligent request routing based on cache content locality, thereby leveraging local cache hits to minimize LLM computation. For RAG queries, where embeddings can potentially be stored across different nodes, it is also essential to consider the locality of retrieved contents to reduce data transfer times. 

% \subsection{Cost Estimation}
% \accheng{what makes LLM workloads interesting is that we can have decode constraints so we can predict what the responses will be like} \accheng{could drop this if nothing unique}
% Our system utilizes predicate push-down techniques to reduce the number of LLM invocations within SQL queries, focusing primarily on local execution scenarios. However, in many real-world scenarios, LLMs are not run locally but are accessed via remote endpoints \cite{openai-pricing}, and there can be different types of LLM models invoked in the query.

% Future research should aim at developing cost estimation models that account for the types of models that are used, and the complexities of remote LLM executions. This involves assessing factors such as network latency and data transfer costs, which are particularly relevant when LLMs are deployed across different geographical locations. Moreover, understanding the computational load of different LLMs and how they interact with SQL queries is essential for precise cost modeling.

% Building on this, there is a compelling need for development on a more intelligent SQL optimizer.  

\paragraph{Semantic Deduplication.}\label{sec:semantic_dedup}
We can extend our exact deduplication technique to filter out more data by implementing rule-based methods. Specifically, we can leverage NLP pre-processing techniques that remove case sensitivity (e.g. ‘Apple’ and ‘apple’) and noise (e.g. ‘<Apple>’ and ‘Apple’) as well as apply stemming and lemmatization (e.g. ‘running’, ‘ran’, ‘runs’ ), spelling correction (e.g. ‘mitake’ vs. ‘mistake’), and stop word removal (e.g. remove ‘a’, ‘the’, etc.). These could be implemented as hard constraints (optionally specified by the user) to reduce the number of LLM invocations without impacting query accuracy since the equivalence of text is specified via the given rules. We can deduplicate even more aggressively by identifying semantically similar rows. For example, "I like this restaurant" and "I love this place" have the same semantic meaning and will likely produce identical output under an LLM sentiment analysis task, so we can send only one of these inputs.
%As such, sending only one of these inputs to LLM for processing should be sufficient for most purposes. 
% While there has been previous work on statistically bounding the accuracy of inference for neural networks~\cite{} \accheng{and other settings?}, this is an open problem for LLMs.  \simon{I don't need we need to open to this direction? Added SemDeDup}
% One possible direction is to evaluate token similarity within the LLM to identify semantically equivalent requests. We are also looking forward to bringing in methods in training dataset de-duplication such as SemDeDup \cite{semdedup} for production workload processing.
% We could provide different thresholds for the user to specify desired query accuracy, similar to techniques used in neural network queries over video~\cite{noscope}. \accheng{others have explored bounding statistical settings but it's an open problem for LLM}
% \accheng{token level similarity }


% \simon{
% A natural extension to this work is to de-duplicate semantically similar rows to reduce computation. For example, de-duplicating "I like this restaurant" and "I love this place" for a sentiment analysis task because they have the same semantic meaning and the output will be identical.

% Importantly, we believe the de-duplication should not affect accuracy of LLM generation. This means the final output should be exactly the same as if it is processed with original batch of rows. Therefore, we see the future work to leverage rule based methods instead of embedding based similarity search. 

% For example, the rules can include using \textit{DISTINCT} for exact de-duplication, removing case sensitivity (e.g. ‘Apple’ and ‘apple’), removing noise (e.g. ‘<Apple>’ and ‘Apple’), stemming and lemmatization (e.g. ‘running’, ‘ran’, ‘runs’ ), noun normalization (e.g. ‘USA’ and ‘United States’ ), spelling correction (e.g. ‘mitake’ vs. ‘mistake’), and stop word removal (e.g. remove ‘a’, ‘the’, etc.). 

% Furthermore, being aware of the tokenizer's behavior can helo further de-duplication. For example, while \textit{\_hi} (with white-space) and \textit{hi} (without white-space) are two different tokens, they are semantically equivalent.
% }


% We choose to avoid affecting the accuracy of query results.

% Hard Rules: 
% \textbf{Solution} 
% \begin{itemize}
%     \item Exact deduplicate with \textit{DISTINCT} before input to external UDF 
%     \item Deduplicate with hard rules for entity resolution datasets 
%     \begin{itemize}
%         \item Case sensitivity: E.g. ‘Apple’ and ‘apple’ 
%         \item Noise removal: E.g., ‘<Apple>’ and ‘Apple’
%         \item Stemming and lemmatization: E.g., ‘running’, ‘ran’, ‘runs’ 
%         \item Normalization: E.g., ‘USA’ and ‘United States’ 
%         \item Word correction: E.g., ‘mitake’ vs. ‘mistake’
%         \item Stop word removal: E.g., remove ‘a’, ‘the’, etc.
%     \end{itemize}
%     \item TODO: token-level deduplication and workload shaping 
% \end{itemize}

% And semantic duplicates
% \textbf{Prefix Cache}
% \begin{enumerate}
%     \item Multi-tenant queries (even larger batch)
%     \item Distributed setup: caching and routing to different cores based on locality 
%     \item Optimize sub-hits 
%     \item Advanced cost estimation on different models as input to LLM 
% \end{enumerate}
% \textbf{Deduplication}
% \begin{enumerate}
%     \item Can say that provide an interface for users to specify those hard rules according to their heuristics 
%     \item Left semantic deduplication for future work 
% \end{enumerate}

% \textbf{More TODOs (relevant directions)}
% \begin{itemize}
%     \item When SQL queries involve joins with each side using different LLMs, understanding the computational costs of these models is crucial for optimizing query execution
%     \begin{itemize}
%         \item E.g. if we have two LLMs, LLM-A (less expensive) and LLM-B (more expensive), the join type (left-join, right-join, inner-join) impacts the efficiency and cost
%         \item Ideally, the more expensive model (LLM-B) should process fewer rows to minimize costs
%         \item The order of join and choice of model should be decided based on some factors like size of datasets, and complexity of tasks they perform 
%     \end{itemize}
%     \item Different model choices affect the cost model 
% \end{itemize}
% \section{Related Work}

Our optimizations build on recent work in LLM inference as well as prior work integrating machine learning and data management. We describe several major related areas below.

% \textbf{Outline}
% \begin{enumerate}
%     \item Existing pipeline tools
%     \begin{enumerate}
%         \item Langchain
%         \item Llamaindex
%     \end{enumerate}
%     \item Throughput optimized LLM inference
%     \begin{enumerate}
%         \item FlexGen
%         \item vLLM
%     \end{enumerate}
%     \item Vector DB papers
% \end{enumerate} 

% \noindent \textbf{Text2SQL.} LLM usage in Text2SQL tasks, where an LLM generates SQL queries given a natural language prompt, has been increasingly explored in recent work ~\cite{gao2023texttosqlempoweredlargelanguage}\cite{zhang2024benchmarkingtexttosqlcapabilitylarge}. While this task involves LLM inference on tabular data, it is distinct from our setting of optimizing LLM operators that are called within SQL queries.  Future work may explore Text2SQL where an LLM generates a SQL query that itself contains LLM operators. %\asim{shu/amog: review this}

% \noindent \textbf{Model pipeline tools.} LLM toolkits, which have grown rapidly in popularity, provide users with the ability to stitch model pipelines together from basic abstractions. Among these, LangChain \cite{langchain} and LlamaIndex \cite{llamaindex} have seen the most usage. LangChain's framework allows for convenient abstractions for different parts of the LLM serving stack and also enables users to ``batch'' multiple requests into a model. However, this is accomplished through basic thread parallelism and without any model optimizations applied to handle the series of queries found in a typical analytics workload.  \amog{Replace this with Text2SQL? Model pipeline tools should be combined with Inference-optimized system section. And I wouldn't realy consider LlamaIndex and LangChain to be model pipeline tools.} \asim{agreed this can be removed now I think}
% \accheng{so this is more basic than vLLM? what's the relation to our work?} \simon{This can be taken out, but I think these actually are more high level LLM workflow tool that can benefit from this work. Think an ORM to a DB.}
\vspace{-0.5em}
\noindent \textbf{Inference-optimized systems.} There has been a recent rise of dedicated systems for LLM inference, including FasterTransformer \cite{faster-transformers}, Orca \cite{orca-continous-batching}, vLLM \cite{vllm}, and FlexGen \cite{flexgen}. Our work builds upon prior work investigating high-throughput LLM inference and continuous batching for model serving. However, past systems focus on the online setting and make no assumptions about the requests sent to the LLM. In contrast, we leverage full workload information from batch queries to improve performance significantly.

% vLLM \cite{} introduces a system for efficient memory management during serving, which is relevant to our problem scenario with potentially large batches of queries. Specifically, our system relies on vLLM's scheduler and utilizes parallel decoding in a continuous batching strategy. Furthermore, the system processes requests in the order they arrive to the model engine. This was something we believe can be improved in an analytics setting, where the order of outputted responses is less critical, and smarter execution planning can be done within the model to reuse computation with common prefixes. 

% FlexGen \cite{} applies both offloading and quantization to efficiently serve large models on limited gpu space with high throughput. However, trade-offs are made for the system assuming a set of latency-insensitive tasks. Our goal was to create a unified abstraction to augment analytics with LLM capabilities while still preserving the ability to make fast, one-off queries on both tabular and unstructured data. 
\vspace{-0.5em}
\noindent \textbf{Prefix Sharing.} Recent work explores developing memory-efficient GPU kernels that perform inference while leveraging shared prefixes. %to compute LLM attention leveraging shared prefix. 
SGLang's RadixAttention \cite{sglang}, Hydragen \cite{hydragen}, and Cascade Inference \cite{cascade-inference} all implement optimized kernels. Our work heavily leverages these kernels to enable prefix sharing while delivering higher throughput as compared to traditional attention kernels \cite{flash-attention}. % \shu{try reducing this}
% \accheng{unclear how these kernels are different from other kernels. what are the other kernels? how does our work leverage these kernels?} \simon{I cited the baseline kernels. This section is to claim we are not on the same track as these work by building on top of the.} achieve memory saving

% \shu{DBML}
\vspace{-0.5em}
\noindent \textbf{LLMs in Relational Data Analytics} 
% \accheng{Ralf and Velox citation} \simon{done}
% There is extensive work on integrating machine learning models with analytics~\cite{noscope,blazeit,prob-pred}. MADLib~\cite{madlib} is one example of many works that have focused on designing systems to train complex models on large datasets. Recent works such as Velox explore online serving using data management systems~\cite{velox}, and Ralf optimizes machine learning feature maintenance in data pipeline ~\cite{ralf-feature-store}. \amog{The works above don't seem very related to our work at all. Systems for training models, or feature stores are unrelated.} 
% There has been prior work on frameworks to run ML models or LLMs as operators on relational data. Systems like Spark MLlib\cite{sparkmllib}, and SystemML\cite{systemml} However, these past works did not specifically address large language models with extremely high computational costs and unique architectural properties, such as the KV cache. As such, LLMs offer many new optimization opportunities in the context of analytics.
There are many systems that support calling LLMs as operators on relational data, spanning from production database vendors like Databricks \cite{databricks-ai-functions}, Google BigQuery \cite{google-bigquery-llm} and AWS Redshift \cite{aws-redshift-llm} to programming frameworks like LOTUS \cite{lotus}. While these works provide APIs for running LLMs over relational data, they do not specifically explore data reordering optimizations to maximize KV cache hits. NoScope~\cite{noscope}, BlazeIt~\cite{blazeit}, and Probabilistic Predicates~\cite{prob-pred} propose approximating expensive ML model calls with less expensive models for approximate query processing, but this can reduce query accuracy, and does not take advantage of the unique opportunities for KV cache reuse in LLM inference.

% 
% Blending advancement of machine learning with database technologies is not new. For example, both MADLib \cite{madlib} and Velox \cite{velox} adds model training and inference as part of analytic workflow. However, the past literature focus on classical machine learning algorithms such as linear regression and K-means clustering. Large language models are a lot more expensive to compute and have interesting properties related to prefix sharing. Therefore, LLMs open up new opportunities to for relational operator optimization such as sorting and better cost estimation. \simon{this paragraph is poorly written, someone plz rewrite it}

% \accheng{Joey/Matei, is there other related work we should be citing?} \matei{approximate queries with ML in SQL: NoScope, BlazeIt, Probabilistic Predicates}
%

\section{Related Work}

Our optimizations build on recent work in LLM inference as well as prior work integrating machine learning and data management. We describe several major related areas below.

\vspace{-0.5em}
\noindent \textbf{Inference-optimized systems.} There has been a recent rise of dedicated systems for LLM inference, including FasterTransformer \cite{faster-transformers}, Orca \cite{orca-continous-batching}, vLLM \cite{vllm}, and SGLang \cite{sglang}. 
Many systems already explore developing memory-efficient GPU kernels that perform inference while leveraging shared prefixes. 
SGLang's RadixAttention \cite{sglang}, Hydragen \cite{hydragen}, and Cascade Inference \cite{cascade-inference} all implement optimized kernels. 
Our work builds upon prior work investigating high-throughput LLM inference and prefix caching for model serving. In addition, we leverage full workload information from batch queries to further improve performance in relational workloads.


% \shu{DBML}
\vspace{-0.5em}
\noindent \textbf{LLMs in Relational Data Analytics} 
Many systems support calling LLMs as operators on relational data, spanning from production database vendors like Databricks \cite{databricks-ai-functions}, Google BigQuery \cite{google-bigquery-llm} and AWS Redshift \cite{aws-redshift-llm} to programming frameworks like LOTUS \cite{lotus}. While these works provide APIs for running LLMs over relational data, they do not explore how reordering data can optimize KV cache hits. 
There is also a line of work ~\cite{noscope, prob-pred} that explores using cheaper models for approximate query generation. This orthogonal direction is not considered in our paper scope, as our work specifically focuses on calling LLMs as functions from inside a regular, given SQL query.

% NoScope~\cite{noscope}, BlazeIt~\cite{blazeit}, and Probabilistic Predicates~\cite{prob-pred} propose approximating expensive ML model calls with less expensive models for approximate query processing, but this can reduce query accuracy, and does not take advantage of the unique opportunities for KV cache reuse in LLM inference.
% However, past systems focus on the online setting and make no assumptions about the requests sent to the LLM. 

% \vspace{-0.5em}
% \noindent \textbf{Prefix Sharing.} Recent work explores developing memory-efficient GPU kernels that perform inference while leveraging shared prefixes. %to compute LLM attention leveraging shared prefix. 
% SGLang's RadixAttention \cite{sglang}, Hydragen \cite{hydragen}, and Cascade Inference \cite{cascade-inference} all implement optimized kernels. Our work heavily leverages these kernels to enable prefix sharing while delivering higher throughput as compared to traditional attention kernels \cite{flash-attention}. 

%\section{Conclusion}
In this paper, we introduce techniques to optimize LLM invocations in relational data analytics workloads.
By leveraging workload information coupled with observations about the LLM inference process, we can significantly improve end-to-end query performance and reduce costs without affecting query semantics. 
Our technique achieves up to 3.4$\times$ decreases in end-to-end query latency with Llama-3-8B and Llama-3-70B and also achieves up to 32\% cost savings under OpenAI and Anthropic pricing models. 

% We first observe that the LLM computation is the dominant cost factor in LLM-enabled SQL workload. Each byte processed by LLMs is multiple orders of magnitude more expensive than bytes processed with traditional SQL operators.
% Our request reordering techniques, along with other standard optimizations, enhance prefix sharing of the LLM KV cache and reduce the number of LLM invocations needed. 
% In optimizing for LLM queries, we fill a void in existing inference systems, which only target the online setting. 
% Our results suggest that there is a wide design space to further enhance LLM inference in the relational analytics setting to greatly improve performance.

% In this paper, we introduce a range of techniques to optimize LLM invocations in relational workloads.
% We first observe that the LLM computation is the dominant cost factor in LLM-enabled SQL workload. Each byte processed by LLMs is multiple orders of magnitude more expensive than bytes processed with traditional SQL operators.
% By leveraging relational workload information in relational analytics, coupled with observations about the LLM inference process, we can significantly improve end-to-end query performance and reduce costs without affecting query semantics. 
% Our request reordering techniques, along with other standard optimizations, enhance prefix sharing of the LLM KV cache and reduce the number of LLM invocations needed. 
% We observe up to 3.4$\times$ decreases in end-to-end query latency. 
% In optimizing for LLM queries, we fill a void in existing inference systems, which only target the online setting. 
% Our results suggest that there is a wide design space to further enhance LLM inference in the relational analytics setting to greatly improve performance.
% \shu{fixed the formatting below on references, seem a bit off}
%\section*{Acknowledgement}
We thank Soujanya Ponnapalli for helpful discussions and feedback, and Jelani Nelson for academic advising. This research was supported by gifts from Accenture, AMD, Anyscale, Broadcom Inc., Google, IBM, Intel, Intesa Sanpaolo, Lambda, Mibura Inc, Samsung SDS, and SAP.

% \received{20 February 2007}
% \received[revised]{12 March 2009}
% \received[accepted]{5 June 2009}

\bibliography{reference}
\bibliographystyle{mlsys2025}

%% If your work has an appendix, this is the place to put it.
% \appendix
% 

\section{Query Examples}
\label{appendix:queries}
Our benchmark suite incorporates a broad range of query types. We show examples of each query type as follows.

% Our query benchmark suite is designed to explore the full spectrum of \sys's capabilities, incorporating a broad range of query types and use cases:
\vspace{-0.2em}

\textbf{\textit{LLM filter.}} This query type leverages LLM for filtering data within a \texttt{WHERE} clause. The LLM processes and analyzes information to meet some specified criteria, such as identifying whether a movie is suitable for kids. This query type illustrates typical use cases in sentiment analysis and content filtering, which are important for application tasks, such as customer feedback analysis and content moderation. 

\begin{mdframed}[linecolor=black, linewidth=.5pt]
\begin{minted}[fontsize=\small]{sql}
SELECT t.movietitle
FROM MOVIES
WHERE LLM(
    'Given the following fields, determine whether the movie is suitable for kids. Answer ONLY with "Yes" or "No".',
    movieinfo,
    reviewcontent,
    reviewtype,
    movietitle
) = 'Yes'
\end{minted}
\end{mdframed} 
\vspace{8pt}
\vspace{8pt}
\vspace{-2em}
\textbf{\textit{LLM projection.}} This query type makes calls to an LLM within a \texttt{SELECT} statement to process information from specified database column(s). It reflects common tasks in data analytics in which the LLM is used for summarization and interpretation based on certain data attributes.

\begin{mdframed}[linecolor=black, linewidth=.5pt]
\begin{minted}[fontsize=\small]{sql}
SELECT LLM(
    'Given the following information, summarize good qualities in this movie that led to a favorable rating.',
    reviewcontent, movieinfo
)
FROM MOVIES
\end{minted}
\end{mdframed} 

% SELECT LLM('Given information including movie descriptions and critic reviews, summarize the good qualities in this movie that led to a favorable rating.', mr.*)
% FROM ( SELECT r.*, m.* FROM reviews r JOIN movies m ON r.link = m.link ) AS mr

\textbf{\textit{Multi-LLM invocation.}} This query type involves multiple LLM calls in different parts of the query and addresses scenarios in which several layers of data processing or analysis are required. It represents advanced analytical tasks, such as combining different data insights.

\begin{mdframed}[linecolor=black, linewidth=.5pt]
\begin{minted}[fontsize=\small]{sql}
SELECT LLM(
    'Given the information about a movie, summarize the good qualities that led to a favorable rating.',
    reviewtype,
    reviewcontent,
    movieinfo,
    genres
)
FROM MOVIES
WHERE LLM(
    'Given the following review, answer whether the sentiment is "POSITIVE" or "NEGATIVE". Respond ONLY with "POSITIVE" or "NEGATIVE", in all caps.',
    reviewcontent
) = 'NEGATIVE'
\end{minted}
\end{mdframed} 

% WITH ( SELECT r.*, m.* FROM reviews r JOIN movies m ON r.link = m.link ) AS mr
% SELECT LLM('Given information including movie descriptions and critic reviews, summarize the good qualities in this movie that led to a favorable rating.', mr.*)
% WHERE LLM('Given the following review, answer whether the sentiment associated is 'POSITIVE' or 'NEGATIVE'. Answer in all caps with ONLY 'POSITIVE' or 'NEGATIVE': ', mr.review) == 'NEGATIVE'

\vspace{-1em}
\textbf{\textit{LLM aggregation.}} This query type incorporates an AVG operator that incorporates LLM outputs into further query processing. For example, one
could use LLMs to assign sentiment scores to individual reviews and then aggregate these scores to calculate an average sentiment for overall customer feedback.
This query type is essential for tasks that need to extract insights from complex textual data.

\begin{mdframed}[linecolor=black, linewidth=.5pt]
\begin{minted}[fontsize=\small]{sql}
SELECT AVG(
    LLM(
        'Rate sentiment in numerical values from 1 (bad) to 5 (good).',
        reviewcontent, movieinfo
    )
) AS AverageScore
FROM MOVIES
\end{minted}
\end{mdframed} 
\vspace{-1em}

\textbf{\textit{Retrieval-augmented generation (RAG)}.} This query type leverages external knowledge bases for enhanced LLM processing, enriching LLM queries with a broader context. It simulates use cases where queries need to pull in relevant information from external sources, such as document databases or knowledge graphs, to provide comprehensive answers. 

\begin{mdframed}[linecolor=black, linewidth=.5pt]
\begin{minted}[fontsize=\small]{sql}
SELECT LLM(
    'Given a question and four supporting contexts, answer the provided question.', VectorDB.search(question, k=4), question)
FROM FEVER
\end{minted}
\end{mdframed}

\section{Dataset Information}
\label{appendix:fields}
% \shu{@asim: add the dataset configuration here on column settings}
We detail the fields and functional dependencies 
(FDs) used for each dataset as follows. 
\begin{tcolorbox}[colback=gray!5!white, colframe=black!75!white, title=MOVIES]
\footnotesize
\begin{verbatim}
columns:
genres, movieinfo, movietitle, 
productioncompany, reviewcontent, 
reviewtype, rottentomatoeslink, 
topcritic

FDs: 
movieinfo, movietitle, 
rottentomatoeslink
\end{verbatim}
\end{tcolorbox}

\begin{tcolorbox}[colback=gray!5!white, colframe=black!75!white, title=PRODUCTS]
\footnotesize
\begin{verbatim}
columns: 
description, id, parent_asin, 
product_title, rating, review_title, 
text, verified_purchase


FDs: 
parent_asin, product_title
\end{verbatim}
\end{tcolorbox}

\begin{tcolorbox}[colback=gray!5!white, colframe=black!75!white, title=BIRD]
\footnotesize
\begin{verbatim}
columns:
Body, PostDate, PostId, Text

FDs: 
Body, PostId
\end{verbatim}
\end{tcolorbox}

\begin{tcolorbox}[colback=gray!5!white, colframe=black!75!white, title=PDMX]
\footnotesize
\begin{verbatim}
columns: 
artistname, bestarrangement, bestpath, 
bestuniquearrangement, composername,
complexity, genre, grooveconsistency, 
groups, hasannotations, hascustomaudio,
hascustomvideo, haslyrics, hasmetadata, 
haspaywall, id, isbestarrangement, 
isbestpath, isbestuniquearrangement, 
isdraft, isofficial, isoriginal, 
isuserpro, isuserpublisher, isuserstaff, 
license, licenseurl, metadata, 
nannotations, ncomments, nfavorites, 
nlyrics, notesperbar, nnotes, nratings, 
ntracks, ntokens, nviews, path, 
pitchclassentropy, postdate, postid, 
publisher, rating, scaleconsistency, 
songlength, songlengthbars, 
songlengthbeats, songlengthseconds, 
songname, subsetall, subsetdeduplicated, 
subsetrated, subsetrateddeduplicated, 
subtitle, tags, text, title, tracks, 
version


FDs:
[metadata, path], 
[hasannotations, hasmetadata, isdraft, 
isofficial, isuserpublisher, subsetall
]

\end{verbatim}
\end{tcolorbox}

\begin{tcolorbox}[colback=gray!5!white, colframe=black!75!white, title=BEER]
\footnotesize
\begin{verbatim}
columns: 
beer/beerId, beer/name, beer/style, 
review/appearance, review/overall, 
review/palate, review/profileName, 
review/taste, review/time

FDs: 
[beer/beerId, beer/name]
\end{verbatim}
\end{tcolorbox}

\begin{tcolorbox}[colback=gray!5!white, colframe=black!75!white, title=FEVER]
\footnotesize
\begin{verbatim}
-- FEVER --
columns: 
claim, evidence1, evidence2, 
evidence3, evidence4

FDs: []
\end{verbatim}
\end{tcolorbox}

\begin{tcolorbox}[colback=gray!5!white, colframe=black!75!white, title=SQuAD]
\footnotesize
\begin{verbatim}
columns: 
question, context1, context2,
context3, context4, context5

FDs: []
\end{verbatim}
\end{tcolorbox}

% \begin{lstlisting}
% -- MOVIES --
% columns: [reviewcontent, reviewtype, topcritic, rottentomatoeslink, movietitle, movieinfo, genres, productioncompany]
% functional_dependencies: [movieinfo, movietitle, rottentomatoeslink]

% -- PRODUCTS --
% columns: [text, description, parent_asin, review_title, verified_purchase, rating, product_title, id]
% functional_dependencies: [product_title, parent_asin]

% -- BIRD --
% columns: [Text, PostId, PostDate, Body]
% functional_dependencies: [PostId, Body]

% -- PDMX --
% columns: [path, metadata, hasmetadata, version, isuserpro, isuserpublisher, isuserstaff, haspaywall, israted, isofficial, isoriginal, isdraft, hascustomaudio, hascustomvideo, ncomments, nfavorites, nviews, nratings, rating, license, licenseurl, genres, groups, tags, songname, title, subtitle, artistname, composername, publisher, complexity, ntracks, tracks, songlength, songlengthseconds, songlengthbars, songlengthbeats, nnotes, notesperbar, nannotations, hasannotations, nlyrics, haslyrics, ntokens, pitchclassentropy, scaleconsistency, grooveconsistency, bestpath, isbestpath, bestarrangement, isbestarrangement, bestuniquearrangement, isbestuniquearrangement, subsetall, subsetrated, subsetdeduplicated, subsetrateddeduplicated]
% functional_dependencies: [[path, metadata], [hasmetadata, isofficial, isuserpublisher, isdraft, hasannotations, subsetall]]


% -- BEER -- 
% columns: [review/profileName, review/time, review/overall, review/taste, review/palate, review/appearance, beer/name, beer/style, beer/beerId]
% functional_dependencies: [beer/beerId, beer/name]


% -- FEVER --
% columns: [claim, evidence4, evidence3, evidence2, evidence1]
% functional_dependencies: []

% -- SQuAD -- 
% columns: [question, context5, context4, context3, context2, context1]
% functional_dependencies: []
% \end{lstlisting}
\vspace{-0.5em}
\section{Prompts}
\label{appendix:prompts}
We detail the system and user prompts for each query type and dataset as follows. 
\vspace{-0.5em}

% \subsection{LLM System Prompt}

\begin{tcolorbox}[colback=gray!5!white, colframe=black!75!white, title=System Prompt]
\scriptsize
\begin{verbatim}
You are a data analyst. Use the provided JSON data 
to answer the user query based on the specified 
fields. Respond with only the answer, 
no extra formatting. 

Answer the below query: 
{QUERY} 

Given the following data: 
{fields}
\end{verbatim}
\end{tcolorbox}


% \begin{lstlisting}
%   You are a data analyst. Use the provided JSON data to answer the user query based on the specified fields. Respond with only the answer, no extra formatting. Answer the below query: \n{QUERY} \n Given the following data: \n {fields}.
% \end{lstlisting}

% 
% -- EXAMPLE MOVIES PROMPT --
% 'Answer the below query:
% Given information including movie descriptions and a critic reviews for movies with a positive sentiment, summarize the good qualities in this movie that led to a favorable rating.
%  Given the following data:
%  {'reviewcontent': "There's over-the-top, and then there's just bafflingly insane.", 'reviewtype': 'Rotten', 'topcritic': 'False', 'rottentomatoeslink': 'm/300_rise_of_an_empire', 'movietitle': '300: Rise of an Empire', 'movieinfo': 'While King Leonidas and his 300 Spartans have their date with destiny at Thermopylae, another battle against the Persians is brewing, this time at sea. Themistocles (Sullivan Stapleton), a Greek general, sees the threat posed by the God-King Xerxes of Persia. He knows that he must unite all of Greece if he is to stand any chance of repelling the Persian invasion. Even if he accomplishes his mission, Themistocles must still face Artemisia (Eva Green), the ruthless leader of the Persian armada.', 'genres': 'Action & Adventure, Drama', 'productioncompany': 'Warner Bros. Pictures'}'
\vspace{-0.6em}


% \subsection{LLM User Prompt}
%\shu{@Asim: can we write out each SQL query for each, because we do filter selection as well}

\begin{tcolorbox}[colback=gray!5!white, colframe=black!75!white, title=User Prompt - LLM Aggregation]
\scriptsize
\begin{verbatim}
MOVIES: Given the following fields of a movie 
description and a user review, assign a sentiment 
score for the review out of 5. Answer with ONLY a 
single integer between 1 (bad) and 5 (good).

PRODUCTS: Given the following fields of a product 
description and a user review, assign a sentiment
score for the review out of 5. Answer with ONLY a
single integer between 1 (bad) and 5 (good).
\end{verbatim}
\end{tcolorbox}
\vspace{-0.6em}

\begin{tcolorbox}[colback=gray!5!white, colframe=black!75!white, title=User Prompt - Multi-LLM Invocation]
\scriptsize
\begin{verbatim}
MOVIES/PRODUCTS: Given the following review, answer 
whether the sentiment associated is 'POSITIVE' or 
'NEGATIVE'. Answer in all caps with ONLY 'POSITIVE' 
or 'NEGATIVE': 
\end{verbatim}
\end{tcolorbox}
\vspace{-0.6em}

\begin{tcolorbox}[colback=gray!5!white, colframe=black!75!white, title=User Prompt - LLM Filter]
\scriptsize
\begin{verbatim}
MOVIES: Given the following fields, answer in one 
word, 'Yes' or 'No', whether the movie would be 
suitable for kids.  Answer with ONLY 'Yes' or 'No'.

PRODUCTS: Given the following fields determine if 
the review speaks positively ('POSITIVE'), 
negatively ('NEGATIVE'), or netural ('NEUTRAL') 
about the product. Answer only 'POSITIVE', 
'NEGATIVE', or 'NEUTRAL', nothing else.

BIRD: Given the following fields related to posts 
in an online codebase community, answer whether the
post is related to statistics. Answer with only 
'YES' or 'NO'.

PDMX: Based on following fields, answer 'YES' or 
'NO' if any of the song information references a 
specific individual. Answer only 'YES' or 'NO', 
nothing else.

BEER: Based on the beer descriptions, does this 
beer have European origin? Answer 'YES' if it does 
or 'NO' if it doesn't.
\end{verbatim}
\end{tcolorbox}

\vspace{-0.5em}
\begin{tcolorbox}[colback=gray!5!white, colframe=black!75!white, title=User Prompt - LLM Projection]
\scriptsize
\begin{verbatim}
MOVIES: Given information including movie 
descriptions and critic reviews, summarize the good
qualities in this movie that led to a favorable 
rating. (also used in multi-invocation)

PRODUCTS: Given the following fields related to 
amazon products, summarize the product, then answer 
whether the product description is consistent with 
the quality expressed in the review. (also used 
in multi-invocation)

BIRD: Given the following fields related to posts 
in an online codebase community, summarize how the
comment Text related to the post body.

PDMX: Given the following fields, provide an 
overview on the music type, and analyze the given 
scores. Give exactly 50 words of summary.

BEER: Given the following fields, provide an 
high-level overview on the beer and review in a 
20 words paragraph.
\end{verbatim}
\end{tcolorbox}
\vspace{-0.6em}

\begin{tcolorbox}[colback=gray!5!white, colframe=black!75!white, title=User Prompt - RAG]
\scriptsize
\begin{verbatim}
FEVER: You are given 4 pieces of evidence as 
{evidence1}, {evidence2}, {evidence3}, and 
{evidence4}. You are also given a claim as {claim}. 
Answer SUPPORTS if the pieces of evidence support 
the given {claim}, REFUTES if the evidence refutes 
the given {claim}, or NOT ENOUGH INFO if there is
not enough information to answer. Your answer
should just be SUPPORTS, REFUTES, or NOT ENOUGH
INFO and nothing else.

SQuAD: Given a question and supporting contexts, 
answer the provided question.
\end{verbatim}
\end{tcolorbox}

% \begin{lstlisting}
% QUERIES

% -- FILTERING -- 
% MOVIES: 'Given the following fields, answer in ONE word, 'Yes' or 'No', whether the movie would be suitable for kids.  Answer with ONLY 'Yes' or 'No'.'

% PRODUCTS: 'Given the following fields determine if the review speaks positively ('POSITIVE'), negatively ('NEGATIVE'), or netural ('NEUTRAL') about the product. Answer only 'POSITIVE', 'NEGATIVE', or 'NEUTRAL', nothing else.'

% BIRD: 'Given the following fields related to posts in an online codebase community, answer whether the post is related to statistics. Answer with only 'YES' or 'NO'.'

% PDMX: 'Based on following fields, answer 'YES' or 'NO' if any of the song information references a specific individual. Answer only 'YES' or 'NO', nothing else.'

% BEER: 'Based on the beer descriptions, does this beer have European origin? Answer 'YES' if it does or 'NO' if it doesn't.'

% -- PROJECTION -- 
% MOVIES: 'Given information including movie descriptions and critic reviews, summarize the good qualities in this movie that led to a favorable rating.' (also used in multi-invocation)

% PRODUCTS: 'Given the following fields related to amazon products, summarize the product, then answer whether the product description is consistent with the quality expressed in the review.' (also used in multi-invocation)

% BIRD: 'Given the following fields related to posts in an online codebase community, summarize how the comment Text related to the post Body'

% PDMX: 'Given the following fields, provide an overview on the music type, and analyze the given scores. Give exactly 50 words of summary.'

% BEER: 'Given the following fields, provide an high-level overview on the beer and review in a 20 words paragraph.'

% -- MULTI-INVOCATION --
% MOVIES/PRODUCTS: 'Given the following review, answer whether the sentiment associated is 'POSITIVE' or 'NEGATIVE'. Answer in all caps with ONLY 'POSITIVE' or 'NEGATIVE': '

% -- AGGREGATION -- 
% MOVIES: 'Given the following fields of a movie description and a user review, assign a sentiment score for the review out of 5. Answer with ONLY a single integer between 1 (bad) and 5 (good): '

% PRODUCTS: 'Given the following fields of a product description and a user review, assign a sentiment score for the review out of 5. Answer with ONLY a single integer between 1 (bad) and 5 (good): '

% -- RAG -- 

% FEVER: 'You are given 4 pieces of evidence as {evidence1}, {evidence2}, {evidence3}, and {evidence4}. You are also given a claim as {claim}. Answer SUPPORTS if the pieces of evidence support the given {claim}, REFUTES if the evidence refutes the given {claim}, or NOT ENOUGH INFO if there is not enough information to answer. Your answer should just be SUPPORTS, REFUTES, or NOT ENOUGH INFO and nothing else.'

% SQuAD: 'Given a question and supporting contexts, answer the provided question.'
% \end{lstlisting}

% \section{LLM Prompt}


% \section{Fixed Reordering (FR)}

% \subsection{Algorithm}
% The Fixed Reordering (FR) algorithm assigns a score to each column based on two factors: the average length of the values in the column and the average group size of the values in the column, prioritizing columns with longer values and larger groups of repeated values to maximize PHC.

% Let $T$ be a table with $n$ rows and $m$ columns, where $T[r][c]$ denotes the value in row $r$ and column $c$. The \textit{average length} of the values in column $c$ is defined as:

% \begin{equation}
% \text{avg\_len}(c) = \frac{1}{n} \sum_{r=1}^{n} \text{len}(T[r][c])
% \end{equation}

% Let $G(c)$ be the set of unique values in column $c$, and let $|R_v|$ be the number of rows where the value in column $c$ equals $v$. The \textit{average group size} of column $c$ is given by:

% \begin{equation}
% \text{avg\_group\_size}(c) = \frac{1}{|G(c)|} \sum_{v \in G(c)} |R_v|
% \end{equation}

% The FR algorithm ranks and sorts the column in descending order of their scores. The \textbf{score} for column $c$ is the product of the average length and the average group size:

% \begin{equation}
% \text{score}(c) = \text{avg\_len}(c)^2 \times \text{avg\_group\_size}(c)
% \end{equation}




% \subsection{FR Worst-Case Analysis}
% We perform a formal worst-case analysis of the FR algorithm under the given assumptions. We aim to compute the maximum possible ratio between the Prefix Hit Counts (PHCs) achieved by the GGR algorithm and the FR algorithm, and prove that this ratio cannot exceed the number of columns $c$. 


% Consider a table $T$ with $r$ rows and $c$ columns. We make the following assumption: 
% \begin{enumerate}
%     \item \textit{Idenitcal Average String Length}: All columns have identical average string lengths, i.e., avg\_len($c$) is the same for all $c$.
%     \item \textit{Single Group per Column}: each column contains exactly one group of $x$ identical values, where $x \leq r$. We later show how this can be generalized to multiple groups. All other values in each column are unique.
%     \item \textit{Non-overlapping Rows between Consecutive Columns}: \shu{not sure if this is a proper assumption, or worst-case choice FR can make}
% \end{enumerate}

% \paragraph{Prefix Hit Count of FR and GGR}
% The FR algorithm assigns a score to each column based on the above formula. Since avg\_len($c$) and the average group size is identical across all columns, the FR algorithm assigns equal scores to all columns and will arbitrarily choose one column to prioritize. 
% The worst-case layout for FR is that there are non-overlapping rows between groups in consecutive columns. Thus, only one group can get prefix hit, as shown in Figure~\ref{}. 
% The prefix hit count (PHC) of the FR algorithm is 
% \begin{equation} \text{PHC}_{\text{FR}} = x - 1 \end{equation} 

% The GGR algorithm recursively selects the group with the largest group size at each step, allowing for different column orderings per row. Since the groups in different columns are in non-overlapping rows, the GGR algorithm can group together all $c$ groups independently. Therefore, the total PHC for the GGR algorithm is: 
% \begin{equation} \text{PHC}_{\text{GGR}} = c \times (x - 1) \end{equation}

% The ratio of the PHCs achieved by the GGR and FR algorithms is $c$. 

% \proof{
% Say we want to add $n$ more such groups with $x$ rows into the table, and see whether we can further increase the gap of $c$ between FR and GGR algorithm. Now it is possible that multiple groups appearing in one column. Say we add $n$ more groups into one arbitrary columns. When we add these $n$ groups, FR will prioritize this column first based on the average group size score, thus its PHC will become $(x-1)\times(n+1)$. While the GGR algorithm will still be able to detect these $n$ groups as usual, resulting in a PHC of $(x-1)\times(c+n)$. The ratio between GGR and FR now becomes $\frac{c+n}{n+1}$. Since $n > 0$ and $c >= 1$, we get $\frac{c+n}{n+1} - c = \frac{n(1-c)}{n+1} \leq 0$, as the numerator is non-positive, and denominator is always positive. Thus, $\frac{c+n}{n+1} \leq c$. Further adding $n$ more groups in an arbitrary columns will not make the gap between FR and GGR larger. 

% }
% \proof{
%     Assume, for contradiction, that there exists a table $T'$ where under the given assumptions, the ratio exceeds $c$. \shu{I was thinking to place another group somewhere, but FR will not stay the same, it will prioritize this column with additional groups}
% }
% For simplicity, we show the worst-case analysis of FR performance assuming each column has identical average string length. 

% \textit{Table Construction} Consider a table $T$ with $r$ rows and $c$ columns. Assume that each \textit{column} contains exactly one group of $x$ identical values, where $x \leq r$, and all other values in the column are distinct. There are $c$ such groups.

% % Now assume that these groups does not overlap the same rows as in the previous columns. Values other than the grouped values are all distinct. 

% \textit{Fixed Reordering} The FR algorithm assigns a fixed score to each column based on average group size of this column, ignoring avg\_len(c) since they are the same across columns. Given that all groups contain $x$ rows, the FR algorithm will break-tie and pick one column to prioritize. \shu{not sure how to write this: zigzag shape} If the groups are formed in zigzag patterns where groups in consecutive columsn do not overlap in rows, this will result in a total prefix hit count of 
% \begin{equation}
%     PHC_{FR} = x - 1
% \end{equation}

% \textit{GGR} The GGR algorithm recursively selects the group of values that has the largest group size at each step, allowing different column orderings per row. In this case, it can take advantage of these non-overlapping groups at each set of rows to achieve a much higher PHC, resulting in: 
% \begin{equation}
%     PHC_{GGR} = (x - 1) \times c
% \end{equation}

% % Comparing the two, the ratio of PHCs for GGR and FR is $min(\frac{r}{x}, c)$. For example, if $x = \frac{r}{c}$, then this ratio becomes $c$. 

% \proof{ 
%     \shu{How to prove this??}
%     We aim to prove that the worst-case performance ratio cannot exceed $c$. 
%     Assume, for the sake of contradiction, that the ratio could exceed $c$, given a table, say $c+n$ where $n > 0$. In this case, it means that the GGR algorithm is able to hit $n$ more groups of $x$ rows than the FR algorithm. Putting such group anywhere in the current setup where each column cotnains exactly one such group will make FR algorithm prioritize the column with one additional group. 
% }


% \textit{Variabe string length}


% \subsection{FR Best-Case Analysis}
% We will now prove that GGR achieves a PHC no worse than fixed reordering. 
% % TODO



% \textbf{Base Case}: Trivially, GGR and Fixed reordering perform identically for 1 row tables, as the entire row is a cache miss. We can now analyze the 2 row scenario:

% We consider a table \( T \) with 2 rows and \( m \) columns. For each column \( c \), we define an indicator variable \( I_c \) as follows:

% \[
% I_c = 
% \begin{cases} 
% 1 & \text{if } T[1][c] = T[2][c], \\
% 0 & \text{if } T[1][c] \neq T[2][c].
% \end{cases}
% \]

% \textit{Fixed Reordering} \\Fixed Reordering (FR) orders the columns based on a fixed score \( score(c) \). 

% In the two-row case, the score \( score(c) \) for column \( c \) reduces to:

% \[
% score(c) = 
% \begin{cases} 
% \text{len}(T[1][c]) & \text{if } I_c = 1, \\
% 0 & \text{if } I_c = 0,
% \end{cases}
% \]

% Thus, \( \sigma_{\text{FA}} \) orders the columns such that all columns with \( I_j = 1 \) appear before any columns with \( I_j = 0 \), in order of length. 

% \textit{GGR} \\The GGR algorithm works by greedily selecting the columna where the score \( score(c) \) is highest and recursing on the rest of the table. We note that the scores in each recursive step are identical to the scores at the beginnign of the algorithm, as each recursive call only removes 1 column and no rows from \( T \).

% For the two-row case, this means GGR will continue to select columns where \( I_c = 1 \) in order of the longest length until no such columns remain.

% \textit{Comparison of Hits for FR and GGR}

% We see that both algorithms will place all columns in order of length. The number of hits will be the sum of the length of each unique value in all columns \( c \) such that \( I_c = 1 \). 
   
% \textbf{Inductive Hypothesis}:
% The inductive hypothesis states that for all \( k \) such that \( 2 < k < n \), GGR performs no worse than FR. Formally, we assume that:

% \[
% \forall k, \ 2 < k < n, \ PHC_{GGR}(T_k) \geq PHC_{FR}(T_k)
% \]

% \textbf{Inductive Step}:
% We now analyze a table \( T \) with n rows. At the first stage, GGR selects the value group with the largest score. This divides \( T \) into 2 subtables, one with the rows containing this max value \( T_{top} \) and one without \( T_{bottom} \).

% First, we note that \( T_{bottom} \) must contain at least \( n - 1 \) rows, as by definition of GGR, one value is selected and all occurrences of that value are included in \( T_{top} \). In the event that all values tie for the same score, a value is selected arbitrarily. With this in mind, we can apply the inductive hypothesis to \( T_{bottom} \) to conclude that \( PHC_{GGR}(T_{bottom})\) \geq \(PHC_{FR}(T_{bottom})\). 
% \asim{is this fair though? the actual values in Tbottom might be different for FR vs GGR so the proof might break here...}

% We can now analyze \( T_{top} \). We can first consider the PHC of the first column in \( T_{top} \). By definition, in rows containing the max value, the PHC in this first column is maximized, as GGR selects the local optimum -- so more hits are contributed to overall PHC in these rows in this column than FR. We can also apply the inductive hypothesis to the subtable not containing the max value column in the rows containing the max value.

\section{Ablations}
We present two sets of ablation experiments: one comparing the prefix hit rate (PHR) between \greedy and an optimal oracle, and another examining the impact of using a smaller LLM model.

\subsection{PHR of GGR v.s. \optimal}
\label{appendix:hit-rate}
OPHR is a very expensive brute-force oracle algorithm that iterates through all possible combinations of value groups and calculates the prefix hit count. In our empirical evaluation, it is impractical to run on larger datasets.

Thus, we test the first (10, 25, 50, 100, 200) rows for each dataset and terminate OPHR runs exceeding 2 hours, reporting the result of the successful run with the most rows. For PDMX, we reduce 57 columns to 10 to enable runs on even 
as few as 10 rows. The PHR (prefix hit rate) and solver runtime in seconds across datasets are reported in Table~\ref{tab:phr_runtime_combined}, with the dataset labeled as \textit{\{dataset\}}-\textit{\{\#rows\}}.

% Thus, to analyze the difference between the greedy heuristic and OHPR, we test the first (10, 25, 50, 100, 200) rows for each dataset and terminate OPHR runs exceeding 2 hours, reporting the result of the successful run with the most rows. For PDMX, we reduce 57 columns to 10 to enable runs even on 10 rows. The PHR (prefix hit rate) and solver runtime results across datasets are reported in Table~\ref{tab:phr_runtime_combined}. 
% We show empirical results on the hit rate comparison 


\begin{table}[h]
\centering
\footnotesize
\begin{tabular}{lcccccc}
\toprule
\textbf{Dataset} & \multicolumn{3}{c}{\textbf{PHR (\%)}} & \multicolumn{2}{c}{\textbf{Solver Runtime (s)}} \\
\cmidrule(lr){2-4} \cmidrule(lr){5-6}
& \textbf{OPHR} & \textbf{GGR} & \textbf{Diff} & \textbf{OPHR} & \textbf{GGR} \\
\midrule
Movies-50   & 80.6 & 80.6 & 0\%   & 2556  & 0.05 \\
Products-25 & 19.7 & 18.5 & -1.2\% & 357   & 0.06 \\
BIRD-50     & 77.5 & 76.2 & -1.3\% & 0.43  & 0.05 \\
PDMX-25     & 29.4 & 28.6 & -0.8\% & 822   & 0.05 \\
Fever-50    & 7.3  & 6.9  & -0.4\% & 110   & 0.23 \\
Beer-10     & 25.7 & 25.6 & -0.1\% & 1269  & 0.08 \\
SQuAD-10    & 34.0 & 34.0 & 0\%   & 1.6   & 0.05 \\
\bottomrule
\end{tabular}
\caption{Comparison of Prefix Hit Rate (PHR) and solver runtime across datasets. GGR achieves near-optimal PHR while being orders of magnitude faster than OPHR.}
\label{tab:phr_runtime_combined}
\end{table}



% \begin{table}[h]
% \centering
% \begin{tabular}{lccccccc}
% \toprule
% \textbf{PHR} & \textbf{Movies-50} & \textbf{Products-25} & \textbf{BIRD-50} & \textbf{PDMX-25} & \textbf{Fever-50} & \textbf{Beer-10} & \textbf{SQuAD-10} \\
% \midrule
% OPHR & 80.6 & 19.7 & 77.5 & 29.4 & 7.3 & 25.7 & 34.0 \\
% GGR  & 80.6 (0\%) & 18.5 (-1.2\%) & 76.2 (-1.3\%) & 28.6 (-0.8\%) & 6.9 (-0.4\%) & 25.6 (-0.1\%) & 34.0 (0\%) \\
% \bottomrule
% \end{tabular}
% \caption{Prefix Hit Rate (PHR) across datasets for OHPR and GGR. GGR achieves hit rate within 2\% as compared to OPHR.}
% \label{tab:phr}
% \end{table}

% \begin{table}[h]
% \centering
% \begin{tabular}{lccccccc}
% \toprule
% \textbf{Solver Runtime (s)} & \textbf{Movies-50} & \textbf{Products-25} & \textbf{BIRD-50} & \textbf{PDMX-25} & \textbf{Fever-50} & \textbf{Beer-10} & \textbf{SQuAD-10} \\
% \midrule
% OPHR & 2556 & 357 & 0.43 & 822 & 110 & 1269 & 1.6 \\
% GGR  & 0.05 & 0.06 & 0.05 & 0.05 & 0.23 & 0.08 & 0.05 \\
% \bottomrule
% \end{tabular}
% \caption{Solver runtime (in seconds) across datasets. GGR demonstrates a significant speedup over OPHR.}
% \label{tab:runtime}
% \end{table}


We can see that on these small samples of the datasets, our algorithm (GGR) achieves within 2\% of the optimal, but can be up to \textit{hours faster} on solver runtime. 

\subsection{Results of Smaller Model}
\label{appendix:models}
To analyze the impact of using a smaller model, we run the Filter Query described in Fig.~\ref{fig:filter-q} with the Llama-3.2-1B model, using the same setup as with Llama-3 8B (i.e., single L4 instance), and compare the prefix hit rate and end-to-end query execution time of GGR with the default vLLM baseline (i.e. Cache Original). The results are reported in Table~\ref{tab:llama32results}. 

% \begin{table}[h]
% \centering
% \begin{tabular}{lccc}
% \toprule
% \textbf{Dataset} & \textbf{OPHR} & \textbf{GGR} & \textbf{Difference} \\
% \midrule
% Movies-50   & 80.6 & 80.6 & 0\% \\
% Products-25 & 19.7 & 18.5 & -1.2\% \\
% BIRD-50     & 77.5 & 76.2 & -1.3\% \\
% PDMX-25     & 29.4 & 28.6 & -0.8\% \\
% Fever-50    & 7.3  & 6.9 & -0.4\% \\
% Beer-10     & 25.7 & 25.6 & -0.1\% \\
% SQuAD-10    & 34.0 & 34.0 & 0\% \\
% \bottomrule
% \end{tabular}
% \caption{Prefix Hit Rate (PHR) comparison between OPHR and GGR across datasets. GGR achieves hit rates within 2\% of OPHR.}
% \label{tab:phr}
% \end{table}

\begin{table}[h]
\centering
\small
\setlength{\tabcolsep}{4pt}
\begin{tabular}{lccc}
\toprule
\textbf{Metric} & \textbf{BIRD} & \textbf{Movies} & \textbf{PDMX} \\
\midrule
Runtime (orig/GGR) & 1.5$\times$ & 1.3$\times$ & 1.3$\times$ \\
Orig PHR (\%) & 10.41 & 29.32 & 11.97 \\
GGR PHR (\%) & 83.99 & 82.10 & 56.00 \\
\midrule
\textbf{Metric} & \textbf{Products} & \textbf{BEER} & \\
\midrule
Runtime (orig/GGR) & 1.4$\times$ & 1.2$\times$ & \\
Orig PHR (\%) & 24.06 & 47.98 & \\
GGR PHR (\%) & 82.10 & 73.93 & \\
\bottomrule
\end{tabular}
\caption{Cache runtime ratio and prefix hit rate (PHR) (\%) comparison between original and GGR ordering for Llama-3.2-1B.}
\label{tab:llama32results}
\end{table}



% \begin{table}[h]
% \centering
% \begin{tabular}{lccccc}
% \toprule
% \textbf{Llama-3.2-1B} & \textbf{BIRD} & \textbf{Movies} & \textbf{PDMX} & \textbf{Products} & \textbf{BEER} \\
% \midrule
% Cache original / GGR (runtime) & 1.5x & 1.3x & 1.3x & 1.4x & 1.2x \\
% Cache original PHR & 10.41\% & 29.32\% & 11.97\% & 24.06\% & 47.98\% \\
% GGR PHR & 83.99\% & 82.10\% & 56.00\% & 82.10\% & 73.93\% \\
% \bottomrule
% \end{tabular}
% \caption{Cache performance and PHR comparison for Llama-3.2-1B across datasets.}
% \label{tab:llama32results}
% \end{table}


We observe similar prefix hit rates with Llama-3.2-1B compared to our previous 8B model runs. This consistency arises from the effectiveness of GGR field reordering, which converts non-reusable field contents (0 hits) into reusable prefixes within the cache.
We also observe that under the same GPU instance setup (e.g., L4 with 24 GB memory), larger models like Llama-8B (7.6 GB) exhibit larger relative performance gains from GGR compared to smaller models like Llama-1B (1.8 GB), despite seeing similar prefix hit rates. This is because prefix caching benefits from reducing computational overhead on shared prefixes and enabling larger batch sizes for LLM generation by reducing memory usage through sharing. For smaller models, the availability of ample GPU memory diminishes the relative impact of prefix caching, as larger batch sizes can be achieved without relying on caching. But for larger models, or when there is less available GPU space, prefix caching benefits become more pronounced.



\end{document}
\endinput
